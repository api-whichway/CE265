% Options for packages loaded elsewhere
\PassOptionsToPackage{unicode}{hyperref}
\PassOptionsToPackage{hyphens}{url}
%
\documentclass[
]{article}
\usepackage{lmodern}
\usepackage{amsmath}
\usepackage{ifxetex,ifluatex}
\ifnum 0\ifxetex 1\fi\ifluatex 1\fi=0 % if pdftex
  \usepackage[T1]{fontenc}
  \usepackage[utf8]{inputenc}
  \usepackage{textcomp} % provide euro and other symbols
  \usepackage{amssymb}
\else % if luatex or xetex
  \usepackage{unicode-math}
  \defaultfontfeatures{Scale=MatchLowercase}
  \defaultfontfeatures[\rmfamily]{Ligatures=TeX,Scale=1}
\fi
% Use upquote if available, for straight quotes in verbatim environments
\IfFileExists{upquote.sty}{\usepackage{upquote}}{}
\IfFileExists{microtype.sty}{% use microtype if available
  \usepackage[]{microtype}
  \UseMicrotypeSet[protrusion]{basicmath} % disable protrusion for tt fonts
}{}
\makeatletter
\@ifundefined{KOMAClassName}{% if non-KOMA class
  \IfFileExists{parskip.sty}{%
    \usepackage{parskip}
  }{% else
    \setlength{\parindent}{0pt}
    \setlength{\parskip}{6pt plus 2pt minus 1pt}}
}{% if KOMA class
  \KOMAoptions{parskip=half}}
\makeatother
\usepackage{xcolor}
\IfFileExists{xurl.sty}{\usepackage{xurl}}{} % add URL line breaks if available
\IfFileExists{bookmark.sty}{\usepackage{bookmark}}{\usepackage{hyperref}}
\hypersetup{
  hidelinks,
  pdfcreator={LaTeX via pandoc}}
\urlstyle{same} % disable monospaced font for URLs
\usepackage{color}
\usepackage{fancyvrb}
\newcommand{\VerbBar}{|}
\newcommand{\VERB}{\Verb[commandchars=\\\{\}]}
\DefineVerbatimEnvironment{Highlighting}{Verbatim}{commandchars=\\\{\}}
% Add ',fontsize=\small' for more characters per line
\newenvironment{Shaded}{}{}
\newcommand{\AlertTok}[1]{\textcolor[rgb]{1.00,0.00,0.00}{\textbf{#1}}}
\newcommand{\AnnotationTok}[1]{\textcolor[rgb]{0.38,0.63,0.69}{\textbf{\textit{#1}}}}
\newcommand{\AttributeTok}[1]{\textcolor[rgb]{0.49,0.56,0.16}{#1}}
\newcommand{\BaseNTok}[1]{\textcolor[rgb]{0.25,0.63,0.44}{#1}}
\newcommand{\BuiltInTok}[1]{#1}
\newcommand{\CharTok}[1]{\textcolor[rgb]{0.25,0.44,0.63}{#1}}
\newcommand{\CommentTok}[1]{\textcolor[rgb]{0.38,0.63,0.69}{\textit{#1}}}
\newcommand{\CommentVarTok}[1]{\textcolor[rgb]{0.38,0.63,0.69}{\textbf{\textit{#1}}}}
\newcommand{\ConstantTok}[1]{\textcolor[rgb]{0.53,0.00,0.00}{#1}}
\newcommand{\ControlFlowTok}[1]{\textcolor[rgb]{0.00,0.44,0.13}{\textbf{#1}}}
\newcommand{\DataTypeTok}[1]{\textcolor[rgb]{0.56,0.13,0.00}{#1}}
\newcommand{\DecValTok}[1]{\textcolor[rgb]{0.25,0.63,0.44}{#1}}
\newcommand{\DocumentationTok}[1]{\textcolor[rgb]{0.73,0.13,0.13}{\textit{#1}}}
\newcommand{\ErrorTok}[1]{\textcolor[rgb]{1.00,0.00,0.00}{\textbf{#1}}}
\newcommand{\ExtensionTok}[1]{#1}
\newcommand{\FloatTok}[1]{\textcolor[rgb]{0.25,0.63,0.44}{#1}}
\newcommand{\FunctionTok}[1]{\textcolor[rgb]{0.02,0.16,0.49}{#1}}
\newcommand{\ImportTok}[1]{#1}
\newcommand{\InformationTok}[1]{\textcolor[rgb]{0.38,0.63,0.69}{\textbf{\textit{#1}}}}
\newcommand{\KeywordTok}[1]{\textcolor[rgb]{0.00,0.44,0.13}{\textbf{#1}}}
\newcommand{\NormalTok}[1]{#1}
\newcommand{\OperatorTok}[1]{\textcolor[rgb]{0.40,0.40,0.40}{#1}}
\newcommand{\OtherTok}[1]{\textcolor[rgb]{0.00,0.44,0.13}{#1}}
\newcommand{\PreprocessorTok}[1]{\textcolor[rgb]{0.74,0.48,0.00}{#1}}
\newcommand{\RegionMarkerTok}[1]{#1}
\newcommand{\SpecialCharTok}[1]{\textcolor[rgb]{0.25,0.44,0.63}{#1}}
\newcommand{\SpecialStringTok}[1]{\textcolor[rgb]{0.73,0.40,0.53}{#1}}
\newcommand{\StringTok}[1]{\textcolor[rgb]{0.25,0.44,0.63}{#1}}
\newcommand{\VariableTok}[1]{\textcolor[rgb]{0.10,0.09,0.49}{#1}}
\newcommand{\VerbatimStringTok}[1]{\textcolor[rgb]{0.25,0.44,0.63}{#1}}
\newcommand{\WarningTok}[1]{\textcolor[rgb]{0.38,0.63,0.69}{\textbf{\textit{#1}}}}
\usepackage{longtable,booktabs}
\usepackage{calc} % for calculating minipage widths
% Correct order of tables after \paragraph or \subparagraph
\usepackage{etoolbox}
\makeatletter
\patchcmd\longtable{\par}{\if@noskipsec\mbox{}\fi\par}{}{}
\makeatother
% Allow footnotes in longtable head/foot
\IfFileExists{footnotehyper.sty}{\usepackage{footnotehyper}}{\usepackage{footnote}}
\makesavenoteenv{longtable}
\usepackage{graphicx}
\makeatletter
\def\maxwidth{\ifdim\Gin@nat@width>\linewidth\linewidth\else\Gin@nat@width\fi}
\def\maxheight{\ifdim\Gin@nat@height>\textheight\textheight\else\Gin@nat@height\fi}
\makeatother
% Scale images if necessary, so that they will not overflow the page
% margins by default, and it is still possible to overwrite the defaults
% using explicit options in \includegraphics[width, height, ...]{}
\setkeys{Gin}{width=\maxwidth,height=\maxheight,keepaspectratio}
% Set default figure placement to htbp
\makeatletter
\def\fps@figure{htbp}
\makeatother
\setlength{\emergencystretch}{3em} % prevent overfull lines
\providecommand{\tightlist}{%
  \setlength{\itemsep}{0pt}\setlength{\parskip}{0pt}}
\setcounter{secnumdepth}{-\maxdimen} % remove section numbering
\ifluatex
  \usepackage{selnolig}  % disable illegal ligatures
\fi

\author{}
\date{}

\begin{document}

The outline and scheme of 《Computer Network》

\begin{itemize}
\item
  Chapter 1 Computer Networks and the Internet
\item
  Chapter 2 Application Layer {[}应用层{]}
\item
  Chapter 3 Transport Layer {[}传输层{]}
\item
  Chapter 4 Network Layer {[}网络层{]}
\item
  Chapter 5 Link Layer: Links, Access Networks, and LANs {[}链路层{]}
\item
  Chapter 6 Wireless and Mobile Networks {[}无线网络{]}
\end{itemize}

\hypertarget{chapter-1-roadmap}{%
\section{Chapter 1: roadmap}\label{chapter-1-roadmap}}

\hypertarget{questions}{%
\subsection{Questions}\label{questions}}

\begin{enumerate}
\def\labelenumi{\arabic{enumi}.}
\item ~
  \hypertarget{ux4ec0ux4e48ux662finternet}{%
  \subparagraph{什么是internet?}\label{ux4ec0ux4e48ux662finternet}}
\end{enumerate}

\textbf{A.两个描述方式}

\textbf{a. 描述因特网的具体构成,即构成因特网的基本硬件和软件组件}

主机(端系统)、通信链路、分组交换机(路由器---用于网络核心、链路层交换机---用于接入网)、因特网服务提供商(ISP)、协议

\textbf{b.为分布式应用提供服务的联网基础设施:}

因特网为应用程序的平台,其API规定了运行在两个端系统上的软件交付数据的方式

\textbf{B.网络协议}

协议定义了在两个或多个通信实体之间交换的报文格式和次序,以及报文发送和/或接收一条报文或其他事件所采取的动作。

掌握计算机网络领域知识的过程就是理解网络协议的构成、原理和工作方式的过程。

\begin{enumerate}
\def\labelenumi{\arabic{enumi}.}
\item ~
  \hypertarget{ux4ec0ux4e48ux662f-protocol}{%
  \subparagraph{什么是 protocol?}\label{ux4ec0ux4e48ux662f-protocol}}
\end{enumerate}

\textbf{Answer:}

\begin{itemize}
\item
  一个协议定义了在两个或多个通信实体之间交换的报文格式和次序,以及报文发送和/或接收一条报文或其他事件所采取的动作。
\item
  \begin{quote}
  A protocol defines the format and the order of messages exchanged
  between two or more communicating entities, as well as the actions
  taken on the trans-mission and/or receipt of a message or other event.
  \end{quote}
\end{itemize}

\begin{enumerate}
\def\labelenumi{\arabic{enumi}.}
\item ~
  \hypertarget{adsl-ux548c-hfc-ux7684ux5f02ux540cux70b9-dsl-ux548c-hfc-ux7684ux5f02ux540cux70b9}{%
  \subparagraph{ADSL 和 HFC 的异同点? (DSL 和 HFC
  的异同点?)}\label{adsl-ux548c-hfc-ux7684ux5f02ux540cux70b9-dsl-ux548c-hfc-ux7684ux5f02ux540cux70b9}}
\end{enumerate}

\textbf{Answer:}

同:

\begin{enumerate}
\def\labelenumi{\arabic{enumi}.}
\item
  主干线路都是光纤
\end{enumerate}

异:

\begin{enumerate}
\def\labelenumi{\arabic{enumi}.}
\item
  ADSL用户接入仍使用电话线,HFC 用户接入使用同轴电缆
\item
  ADSL 上行信道256Kbps, 下行信道 1Mbps; HFC 带宽取决于用户的数量
\end{enumerate}

\begin{enumerate}
\def\labelenumi{\arabic{enumi}.}
\item ~
  \hypertarget{ux5206ux7ec4ux4ea4ux6362ux548cux7535ux8defux4ea4ux6362ux7684ux5f02ux540cux70b9}{%
  \subparagraph{分组交换和电路交换的异同点?}\label{ux5206ux7ec4ux4ea4ux6362ux548cux7535ux8defux4ea4ux6362ux7684ux5f02ux540cux70b9}}
\end{enumerate}

\begin{enumerate}
\def\labelenumi{\arabic{enumi}.}
\item ~
  \hypertarget{ux56dbux79cdux65f6ux5ef6ux662fux4ec0ux4e48}{%
  \subparagraph{四种时延是什么?}\label{ux56dbux79cdux65f6ux5ef6ux662fux4ec0ux4e48}}
\end{enumerate}

\begin{enumerate}
\def\labelenumi{\arabic{enumi}.}
\item ~
  \hypertarget{5ux5c42ux6a21ux578b}{%
  \subparagraph{5层模型}\label{5ux5c42ux6a21ux578b}}
\end{enumerate}

\textbf{Answer}

\begin{enumerate}
\def\labelenumi{\arabic{enumi}.}
\item
  应用层
\item
  传输层
\item
  网络层
\item
  链路层
\item
  物理层
\end{enumerate}

\begin{enumerate}
\def\labelenumi{\arabic{enumi}.}
\item ~
  \hypertarget{osi-7ux5c42ux6a21ux578b}{%
  \subparagraph{OSI 7层模型}\label{osi-7ux5c42ux6a21ux578b}}
\end{enumerate}

\textbf{Answer}

\begin{enumerate}
\def\labelenumi{\arabic{enumi}.}
\item
  应用层
\item
  表示层
\item
  会话层
\item
  运输层
\item
  网络层
\item
  链路层
\item
  物理层
\end{enumerate}

\hypertarget{11-what-is-the-internet}{%
\subsection{\texorpdfstring{1.1 what \emph{is} the
Internet?}{1.1 what is the Internet?}}\label{11-what-is-the-internet}}

\begin{quote}
2个角度, 架构:包含了什么设备边缘, 主机和系统 服务:什么是协议
\end{quote}

1.6

What's the Internet: ``nuts and bolts'' view

\begin{quote}
在本书中,我们使用一种特定的计算机网络,即公共因特网,作为讨论计算机网络及
其协议的主要载体:但什么是因特网?回答这个问题有两种方式:其一,我们能够描述因特网的具体构成,即构成因特网的基本硬件和软件组件;其二,我们能够根据为分布式应用
提 供 服 务 的 联 网 基 础 设 施 来 描 述 因 特 网 我 们 先 从 描 述 因
特 网 的 具 体 构 成 开 始 , 并 用图1-1举例说明我们的讨论、
\end{quote}

\begin{enumerate}
\def\labelenumi{\arabic{enumi}.}
\item
  millions of connected computing devices:
\end{enumerate}

\begin{itemize}
\item
  \begin{itemize}
  \item
    \emph{hosts{[}主机{]}} \emph{=} \emph{end systems{[}端系统{]}}
  \item
    running \emph{network apps}
  \end{itemize}
\end{itemize}

\begin{quote}
\end{quote}

\begin{figure}
\centering
\includegraphics{/Users/whichway/workspace/E_ESSEX/CE265/MD/CE265.assets/image-20210310084304952.png}
\caption{}
\end{figure}

\begin{quote}
因特网是一个世界范围的计算机网络,即它是一个互联了遍及全世界的数以亿计的计算设备的网络
在不久前,这些计算设备多数是传统的桌面PC、Linux工作站以及所谓的服务
器(它们用于存储和 传输Web页面和电子邮件报文等信息)。然而,
越来越多的非 传统的因特网端系统 (如便携机、智能手机 、平板电脑
、电视、游戏 机、Weh相机、汽年
、环境传感设备、数字相框、家用电器)和安全系统,正在与因特网 相连。的确
,在许多非传统设
备连接到因特网的情况下,计算机网络这个术语开始听起来有些过时了,用因特网术语來说.所有这些设备都称为
主机( host )或端系统( end system ) ,到2 0 1 1年7月为止,有大约5亿
台端系 统与因特网连接,这并未将 智能手机 、便 携机 和仅断断 续续
与因特网
连接的设备计算在内[ISC2011]。总体来说,因特网用户数估计超过20亿[HTI
2011]。
\end{quote}

\begin{enumerate}
\def\labelenumi{\arabic{enumi}.}
\item
  \emph{communication links}
\item
  2 fiber, copper, radio, satellite
\item
  3 transmission rate: \emph{bandwidth}
\end{enumerate}

\begin{figure}
\centering
\includegraphics{/Users/whichway/workspace/E_ESSEX/CE265/MD/CE265.assets/image-20210310084312021.png}
\caption{}
\end{figure}

\begin{quote}
端系统( end system )通过通信链路(communication
link)和分组交换机(packet switch)连接到一起。在1 .
2节中,我们将介绍许多类型的通信链路,它们由不
同类型的物理媒体组成。这些物理媒体包括同轴电缆、铜线、光纤和无线电频谱。不同的链路能够以不同的速率传输数据,链路的传输速率以比特/秒度量(b
i t / s,或b p s
)。当一台端系统要向另一台端系统发送数据时,发送端系统将数据分段,并为每段加上首部字节。由此形成的信息包用计算机网络的术语来说称为分组(p
a c k e t )。这些分组通过网络发送到目的端系统,在那里被装配成初始数据。
\end{quote}

\begin{enumerate}
\def\labelenumi{\arabic{enumi}.}
\item
  \emph{Packet switches:} forward packets (chunks of data)
\item
  1 \emph{routers} and \emph{switches}
\end{enumerate}

\begin{figure}
\centering
\includegraphics{/Users/whichway/workspace/E_ESSEX/CE265/MD/CE265.assets/image-20210310084317568.png}
\caption{}
\end{figure}

\begin{figure}
\centering
\includegraphics{/Users/whichway/workspace/E_ESSEX/CE265/MD/CE265.assets/image-20210310084328468.png}
\caption{}
\end{figure}

\begin{quote}
分组交换机( packet
switch)从它的一条人通信链路接收到达的分组,并从它的一条出通信链路转发该分组。市面上流行着各种类型、各具特色的分组交换机,但在当今的因特网中,两种最著名的类
型是 路由 器(rou ter)和链路层交换机(link -laye r s witch)。这两种类
型的交换机 朝着最终目的地转发
分组。链路层交换机通常用于接人网中,而路由器通常用于网络核心中。从发送端系统到接收端系统,一个分组所经历的一系列通信链路和分组交换机称为通过该网络的路径(route或path)。因特网所承载的精确通信量是难以估算的,不过思科公
司[Cisco VNI 2011]估计,全球因特网流量在2012年每月大约为40EB
(10"字节)。

用于传送分组的分组交换网络在许多方面类似于承载运输车辆的运输网络,该网络包
括了高速公路、公路和立交桥。例如,考虑下列情况,一个工厂需要将大量货物搬运到数千公里以外的某个目的地仓库。在工厂中,货物要分开并装上卡车车队。然后,每辆卡车独立地通过高速公路、公路和立交桥组成的网络向该仓库运送货物。在目的地仓库,卸下
这些货物,并且与一起装载的同一批货物的其余部分堆放在一起。因此,在许多方面,分
组类似于卡车,通信链路类似于高速公路和公路,分组交换机类似于立交桥,而端系统类似于建筑物。就像卡车选取运输网络的一条路径前行一样,分组则选取计算机网络的一条路径前行。

端系统通过因特网服务提供商(Internet Service Provider,
ISP)接人因特网,包括如本地电缆或电话公 司那样的住宅区ISP、公 司I S
P、大学I S P,以及那些在机场、旅馆、咖啡店和 其 他 公 共场所提供W i F
i接人的I S P。每个I S P是一个由多个分组交换机和多段通信 链路组成的网络。
各I S P为端系统提供了各种不同类型的网络接入,包括如线缆调制解调器或D S
L那样的住宅宽带接人 、高速局域网接人 、无线接人 和5 6 k b p
s拨号调制解调器接人 。I S P也 为 内容提供者提供因特网接人服务,将W e
b站点直接接人因特网。因特网就是将端系统彼此互联,因此为端系统提供接人的I
S P也必须互联。 低层的I S P通过国家的、 国际的髙层 IS P (如 Le ve l 3
Co mm un ic at io ns、AT \&T、Sp ri nt 和 NT T)互联起来。高层 ISP
是由通过高速光纤链路互联的高速路由器组成的。无论是高层还是低层I S
P网络,它们每个都是独立管理的,运行着IP协议(详情见后),遵从一定的命名和地址习惯。我们将在1.3节中更为详细地考察ISP及其互联的情况。
\end{quote}

1.7

``Fun'' internet appliances

\begin{figure}
\centering
\includegraphics{/Users/whichway/workspace/E_ESSEX/CE265/MD/CE265.assets/image-20210310084351698.png}
\caption{}
\end{figure}

1.8

What's the Internet: ``nuts and bolts'' view

\begin{enumerate}
\def\labelenumi{\arabic{enumi}.}
\item
  \emph{Internet:} ``network of networks''
\item
  \begin{itemize}
  \item
    Interconnected{[}相互连接的🔗{]} ISPs
  \end{itemize}
\item
  \emph{protocols} control sending, receiving of msgs
\item
  \begin{itemize}
  \item
    e.g., TCP, IP, HTTP, Skype, 802.11
  \end{itemize}
\item
  \emph{Internet standards}
\item
  \begin{itemize}
  \item
    \hl{RFC: Request for comments}
  \item
    \hl{IETF: Internet Engineering Task Force}
  \end{itemize}
\end{enumerate}

\begin{figure}
\centering
\includegraphics{/Users/whichway/workspace/E_ESSEX/CE265/MD/CE265.assets/image-20210310084454901.png}
\caption{}
\end{figure}

\begin{quote}
端系统、 分组交换机和 其 他因特网部件都要运行一系列 协议(p r o t o c o
l ),这些协议控制因特网中信息的接收和发送。

TCP(Transmission Control Protocol ,传输控制协议)和I P ( I n t e r n e
t P r o t o c o l ,网际协议)是因特网中两个最为重要的协议。I
P协议定义了在路 由 器 和 端 系 统 之 间 发 送 和 接 收 的 分 组格
式。因特 网 的主要协议 统 称为T C P / I P。我们 在这一章中就开始接触这些
协议。 但这仅 仅是个开始,本书的许多地方与计算机网络协议有关。
鉴于因特网协议的重要性,每个人就各个协议及其作用取得一致认识是很重要的,这样
人 们 就 能 够 创 造 协 同 工 作 的 系 统 和 产 品 。 这 正 是 标 准 发
挥 作 用 的 地 方 . 因 特 网 标 准 (Internet
standard)由因特网工程任务组(Internet Engineering Task Force, IETF) IETF
2 0 1 2 ]研发:

I E T F的标准文档称为请求评论( Request For Comment , R F C ) RFC最初是作
为 普 通 的 请 求 评 论 ( 因 此 而 得 名 ) , 以 解 决 因 特 网 先 驱 者
们 面 临 的 网 络 和 协 议 问 题[A ll ma n 20 11 ]RF C文档 往往 是技
术性 很强 并相当详细 的。它们定义了 TC P、IP、HT TP ( 用 于 W e b ) 和 S
M T P ( 用 于 电 子 邮 件 ) 等 协 议 。 目 前 已 经 有 将 近 6 0 0 0 个
R F C • 其 他 组
织也在制定用于网络组件的标准,最引人注目的是针对网络链路的标准 。
例如,I E E E 8 0 2 LAN/MAN标准化委员会[IEEE 802
202]制定了以太网和无线WiFi的标准
\end{quote}

1.9

What's the Internet: a service view

\begin{enumerate}
\def\labelenumi{\arabic{enumi}.}
\item
  \emph{Infrastructure that provides services to applications:}
\item
  \begin{itemize}
  \item
    Web, VoIP, email, games, e-commerce, social nets, \ldots{}
  \end{itemize}
\item
  \emph{provides programming interface to apps}
\item
  \begin{itemize}
  \item
    hooks that allow sending and receiving app programs to ``connect''
    to Internet
  \item
    provides service options, analogous to postal service
  \end{itemize}
\end{enumerate}

\begin{figure}
\centering
\includegraphics{/Users/whichway/workspace/E_ESSEX/CE265/MD/CE265.assets/image-20210310084800821.png}
\caption{}
\end{figure}

\begin{quote}
前面的讨论已经辨识了构成因特网的许多部件 。 但是我们 也能从 一 个完全 不
同的角度,即 从
为应用程序提供服务的基础设施的角度来描述因特网。这些应用程序包括电子邮
件 、 W e b 冲 浪 、 即 时 讯 息 、 社 交 网 络 、 I P 语 音 ( V o I P )
、 流 式 视 频 、 分 布 式 游 戏、对 等 ( p e e r - t o - p e e r , P 2
P ) 文 件 共 享 、 因 特 网 电 视 、 远 程 注 册 等 等 。 这 些 应 用
程序称为 分 布式应用程序( d i s t r i b u t e d a p p l i c a t i o n )
,因为它们涉及多台相互 交换数据的端系统。重 要
的是,因特网应用程序运行在端系统上,即它们并不运行在网络核心中的分组交换机中
。尽管分组交换机促进端系统之间的数据交换,但它们并不 关心作
为数据的源或宿的应用程序。

我们稍深人地探讨一下为应用程序提供服务的基础设施的含义。为此,假定你对分布式因特网应用程序有一个激动人心的新思想,它可能大大地造福于人类,或者它可能直接使你富有和出名。你将如何将这种思想转换成为一种实际的因特网应用程序呢?因为应用程序运行在端系统上,所以
你将需要编写运行在端系统上的一 些软件 。 例如,你 可能用 J a v a、C或
python 编写软件。此时,因为你在研发一种分布式因特网应用程序,运行在不
同端系统上的软件将需要互相发送数据在这里我们碰到 一 个核心问题,这导致了
另
一种描述因特网的方法,即将因特网描述为应用程序的平台。运行在一个端系统上的应用程序

怎样才能指令因特网向运行在另一个端系统上的软件发送数据呢?
与因特网相连的端系统提供了一个应用程序编程接口(Ap pl ic at io n Pr og ra
mming Interface, A P I ),该A P
I规定了运行在一个端系统上的软件请求因特网基础设施向运行在另一个端系统上的
特定目 的地软件 交 付数 据的方 式。因特网A P
I是一套发送软件必须遵循的规则集合,因 此 因特 网 能够 将 数 据交 付给 目
的地、我们将在第2章详细讨论因特网A P I ,此时,我们 做 一
个简单的类比,在本书 中我们将经常使用这个类比。假定A l i c
e使用邮政服务向 B o b 发 一 封 信 。 当 然 , A l i c e 不 能 只 是 写
了 这 封 信 ( 相 关 数 据 ) 然 后把该信 丢 出窗外。相反,邮政服务要求A l
i c e将信放人 一 个 信封中;在信封的中央写 上B o b的全 名
、地址和邮政编码;封上信封;在信封的右上角贴上邮票;最后将该信封丢进一个邮局的邮政服务邮箱中
。因此,该邮政服务有自己的``邮政服务A P I ''或一套规则,这是A l i c
e必须遵循的,

这样邮政服务才能将自己的信 件 交 付给B o b .
.同理,因特网也有一个发送数据的程序必须遵循的API,使因特网向接收数据的程序交付数据。
当然,邮政服务向顾客提供了多种服务,如特快专递、挂号、普通服务等。同样的,因特网向应用程序提供
了多种服务当你研发 一种因特网应用程序时,也必须为
你的应用程序选择其中的一种因特网服务。我们将在第2章中描述因特网服务.
\end{quote}

1.10

\emph{human protocols:}

\begin{itemize}
\item
  ``what's the time?''
\item
  ``I have a question''
\item
  introductions
\end{itemize}

\ldots{} specific msgs sent

\ldots{} specific actions taken when msgs received, or other events

\begin{quote}
也许理解计算机网络协议概念的一 个最容易办法是,先 与某些
人类活动进行类比,因为我们 人类无时无刻
不在执行协议。考虑当你想要向某人询问时间时将要怎样做 。图1 - 2中显示了
一种典型的 交 互过 程。 人类协议 ( 至少说是 好的 行为方 式)
要求一方首先 进 行 问 候 ( 图 1 - 2 中 的 第 一 个 `` 你 好 '' ) , 以
开 始 与 另 一 个 人 的 通 信 。 对 `` 你 好 '' 的 典型 响 应 是 返 回
一 个 " 你 好 '' 报 文 。 此 人 用 一 个 热 情 的 `` 你 好 '' 进 行 响
应 , 隐 含 着 一 种指 示 , 表 明 能 够 继 续 向 那 人 询 问 时 间 了
。 对 最 初 的 `` 你 好 '' 的 不 同 响 应 ( 例 如 `` 不 要烦 我 ! '',
或 `` 我 不 会 说 英 语 '', 或 某 些 不 合 时 宜 的 回 答 ) 也 许 表
明 了 一 个 勉 强的或不能进行的通信 。在此情况
下,按照人类协议,发话者也许将不能够询问时间了 。有时,问的问题根本得不
到 任 何回答,在此情况 下,发话者通常会放弃向这个 人询问时间。注意在我们
人类协议中,有我们 发送的特定报文,也有我们根据接收到的应答报文或其 他
事
件采取的动作(例如在某个给定的时间内没有回答)n显然,发送和接收的报文,以及这些报文发送和接收或其
他 事 件 出现时所采取的动 作,这些在一 个 人类协议中起到
了核心作用。如果人 们 使用不 同的协议(例如,如果一 个 人讲礼貌,而另 一
人 不讲礼貌,或一 个 人明白时间这个概念,而另 一 人 却
不知道),该协议就不能互 动,因而不能完成有用 的 工 作 。 在 网 络 中 这
个 道 理 同 样 成 立 。 即 为 了 完 成 一 项 工 作 , 要 求 两 个 ( 或
多 个 )通信实体运行相同的协议。

我们 再考虑第二 个 人类类比的例子。
假定你正在大学课堂里上课(例如上的是计算机 网 络 课 程 ) 。 教 师 正 在
唠 唠 叨 叨 地 讲 述 协 议 , 而 你 困 惑 不 解 。 这 名 教 师 停 下 来
问 :`` 同学们有什 么问题吗?''(教师发送出 一
个报文,该报文被所有没有睡觉的学生接收到 了 。 ) 你 举 起 了 手(向教
师发送了 一 个隐含的报 文) , 这位教 师面带 微笑地示意你说: `` 请
讲\ldots{} \ldots{} '' (教师发 出的这个报文鼓励
你提出问题,教师喜欢被问问题。)接着你就问了 问 题 ( 即 向 该 教 师 传
输 了 你 的 报 文 ) 。 教 师 听 取 了 你 的 问 题 ( 即 接 收 了 你 的
问 题 报文)并加 以回答(向 你 传输了回答报文)。我们 再 一次看到
了报文的发送和接收,以 及 这 些 报 文 发 送 和 接 收 时 所 采 取 的 一
系 列 约 定 俗 成 的 动 作 , 这 些 是 这 个 `` 提 问 与 回 答 ''
协议的核心。
\end{quote}

\emph{network protocols:}

\begin{itemize}
\item
  machines rather than humans
\item
  all communication activity in Internet governed by protocols
\end{itemize}

\emph{protocols} \emph{define} \emph{format**,} \emph{order} \emph{of}
\emph{msgs sent and received} \emph{among network entities, and}
\emph{actions taken} \emph{on msg transmission, receipt}

\begin{quote}
网络协议类似 于 人类协议,除了 交换报文和采取 动
作的实体是某些设备的硬件或软件组件(这些设备可
以是计算机、智能手机、平板电脑、路由器或其 他
具有网络能力的设备)。在因特网中,凡是涉及 两
个或多个远程通信实体的所有活动都受 协议的制约。 例 如 , 在 两 台 物 理
上 连 接 的 计 算 机 中 , 硬 件 实 现 的 协 议 控 制 了 在 两 块 网 络
接 口 卡 间 的" 线 上 '' 的 比 特 流 ; 在 端 系 统 中 , 拥 塞 控 制 协
议 控 制 了 在 发 送 方 和 接 收 方 之 间 传 输 的 分组 发 送 的 速 率
。 协 议 在 因 特 网 中 到 处 运 行 , 因 此 本 书 的 大 量 篇 幅 与 计
算 机 网 络 协 议有关。 以大家可能熟悉的一 个计算机网络协议为
例,考虑当你 向 一 个W e b服务 器 发 出请求(即 你在W e b浏览器
中键人一个W e b网页的U R L )时所发生的情况。图1 - 2右半部分显示
了这种情形。首先,你的计算机将向该Web服务器发送一条连接请求报文,并等待回答。
该 W e b 服 务 器 将 最 终 能 接 收 到 连 接 请 求 报 文 , 并 返 回 一
条 连 接 响 应 报 文 。 得 知 请 求 该 W e b文档正常以
后,计算机则在一条G E T报文中 发送要从这台W e b服务 器 上
取回的网页名字。最后,Web服务器向计算机返回该Web网页(文件)
\end{quote}

1.11 \#\#\# \hl{⭐️ What's a protocol?}

\begin{quote}
协议, 定义了报文和次序,报文发送和接收采取的动作
\end{quote}

\textbf{Answer:}

\begin{itemize}
\item
  {[} \hl{defines, format and the order exchanged between entities,
  actions on transmission and receopt} {]} A protocol defines the format
  and the order of messages exchanged between two or more communicating
  entities, as well as the actions taken on the trans-mission and/or
  receipt of a message or other event.
\end{itemize}

\begin{quote}
一个协议定义了在两个或多个通信实体之间交换的报文格式和次序,以及报文发送和/或接收一条报文或其他事件所采取的动作。
\end{quote}

a human protocol and a computer network protocol:

\begin{figure}
\centering
\includegraphics{/Users/whichway/workspace/E_ESSEX/CE265/MD/CE265.assets/image-20210310084619368.png}
\caption{}
\end{figure}

\begin{quote}
TCP 三次握手🤝
\end{quote}

\emph{Q:} other human protocols?

\hypertarget{12-network-edge-ux7f51ux7edcux8fb9ux7f18--end-systems-access-networks-links}{%
\subsection{1.2 network edge {[}网络边缘{]} : end systems, access
networks,
links}\label{12-network-edge-ux7f51ux7edcux8fb9ux7f18--end-systems-access-networks-links}}

\begin{quote}
\textbf{2.*}*网络边缘**

a.端系统:位于边缘的与因特网相连的计算机和其他设备

b.接入网:将端系统连接到边缘路由器的物理链路

\begin{figure}
\centering
\includegraphics{/Users/whichway/workspace/E_ESSEX/CE265/MD/CE265.assets/image-20210420083016169.png}
\caption{}
\end{figure}
\end{quote}

1.13

A closer look at network structure

\begin{itemize}
\item
  \emph{network edge:}
\item
  \begin{itemize}
  \item
    hosts: clients and servers
  \item
    servers often in data centers
  \end{itemize}
\item
  \emph{access networks, physical media:} wired, wireless communication
  links
\item
  \emph{network core:}
\item
  \begin{itemize}
  \item
    interconnected routers
  \item
    network of networks
  \end{itemize}
\end{itemize}

\begin{figure}
\centering
\includegraphics{/Users/whichway/workspace/E_ESSEX/CE265/MD/CE265.assets/image-20210310090054142.png}
\caption{}
\end{figure}

1.14

Access networks and physical media

\emph{Q: How to connect end systems to edge router?}

\begin{itemize}
\item
  residential access nets
\item
  institutional access networks (school, company)
\item
  mobile access networks
\end{itemize}

\emph{keep in mind:}

\begin{itemize}
\item
  bandwidth (bits per second) of access network?
\item
  shared or dedicated?
\end{itemize}

\begin{figure}
\centering
\includegraphics{/Users/whichway/workspace/E_ESSEX/CE265/MD/CE265.assets/image-20210310090116089.png}
\caption{}
\end{figure}

1.15

Access net: \hl{digital subscriber line (DSL)}

\begin{figure}
\centering
\includegraphics{/Users/whichway/workspace/E_ESSEX/CE265/MD/CE265.assets/image-20210310090147366.png}
\caption{}
\end{figure}

\begin{itemize}
\item
  \begin{enumerate}
  \def\labelenumi{\arabic{enumi}.}
  \item
    use \emph{existing} telephone line to central office DSLAM
  \item
    \begin{itemize}
    \item
      data over DSL phone line goes to Internet
    \item
      voice over DSL phone line goes to telephone net
    \end{itemize}
  \item
    \textless{} 2.5 Mbps upstream transmission rate (typically
    \textless{} 1 Mbps)
  \item
    \textless{} 24 Mbps downstream transmission rate (typically
    \textless{} 10 Mbps)
  \end{enumerate}
\end{itemize}

1.16

Access net: cable network

\begin{figure}
\centering
\includegraphics{/Users/whichway/workspace/E_ESSEX/CE265/MD/CE265.assets/image-20210310090356894.png}
\caption{}
\end{figure}

\emph{frequency division multiplexing:} different channels transmitted
in different frequency bands

1.17

Access net: cable network

\begin{figure}
\centering
\includegraphics{/Users/whichway/workspace/E_ESSEX/CE265/MD/CE265.assets/image-20210310090618165.png}
\caption{}
\end{figure}

\begin{itemize}
\item
  HFC: hybrid fiber coax
\item
  \begin{itemize}
  \item
    asymmetric: up to 30Mbps downstream transmission rate, 2 Mbps
    upstream transmission rate
  \end{itemize}
\item
  network of cable, fiber attaches homes to ISP router
\item
  \begin{itemize}
  \item
    homes \emph{share access network} to cable headend
  \item
    unlike DSL, which has dedicated access to central office
  \end{itemize}
\end{itemize}

1.18

Access net: home network

\begin{figure}
\centering
\includegraphics{/Users/whichway/workspace/E_ESSEX/CE265/MD/CE265.assets/image-20210310090751619.png}
\caption{}
\end{figure}

1.19

Enterprise access networks (Ethernet)

\begin{figure}
\centering
\includegraphics{/Users/whichway/workspace/E_ESSEX/CE265/MD/CE265.assets/image-20210310091109063.png}
\caption{}
\end{figure}

\begin{itemize}
\item
  typically used in companies, universities, etc
\item
  \begin{itemize}
  \item
    10 Mbps, 100Mbps, 1Gbps, 10Gbps transmission rates
  \item
    today, end systems typically connect into Ethernet switch
  \end{itemize}
\end{itemize}

1.20

Wireless access networks

\begin{itemize}
\item
  shared \emph{wireless} access network connects end system to router
\item
  \begin{itemize}
  \item
    via base station aka ``access point''
  \end{itemize}
\end{itemize}

\emph{wireless LANs:}

\begin{itemize}
\item
  \begin{itemize}
  \item
    within building (100 ft)
  \item
    802.11b/g (WiFi): 11, 54 Mbps transmission rate
  \end{itemize}
\end{itemize}

\begin{figure}
\centering
\includegraphics{/Users/whichway/workspace/E_ESSEX/CE265/MD/CE265.assets/image-20210310091217671.png}
\caption{}
\end{figure}

wide-area wireless access

\begin{itemize}
\item
  \begin{itemize}
  \item
    provided by telco (cellular) operator, 10's km
  \item
    between 1 and 10 Mbps
  \item
    3G, 4G: LTE
  \end{itemize}
\end{itemize}

\begin{figure}
\centering
\includegraphics{/Users/whichway/workspace/E_ESSEX/CE265/MD/CE265.assets/image-20210310091246399.png}
\caption{}
\end{figure}

1.21

Host: sends \emph{packets} of data

host sending function:

\begin{itemize}
\item
  ❖takes application message
\item
  ❖breaks into smaller chunks, known as \emph{packets}, of length
  \emph{L} bits
\item
  ❖transmits packet into access network at \emph{transmission rate R}
\item
  \begin{itemize}
  \item
    link transmission rate, aka link \emph{capacity, aka link bandwidth}
  \end{itemize}
\end{itemize}

\begin{figure}
\centering
\includegraphics{/Users/whichway/workspace/E_ESSEX/CE265/MD/CE265.assets/image-20210310091323260.png}
\caption{}
\end{figure}

\begin{figure}
\centering
\includegraphics{/Users/whichway/workspace/E_ESSEX/CE265/MD/CE265.assets/image-20210310091343333.png}
\caption{}
\end{figure}

\[packet
\ transmission\ delay
\\ = time\ needed\ to\
transmit\ L\ bit\
packet\ into\ link 
\\ = \frac{L(bits)}{R(bits/sec)}\]

1.22

Physical media

\begin{itemize}
\item
  bit: propagates between transmitter/receiver pairs
\item
  physical link: what lies between transmitter \& receiver
\item
  guided media:
\item
  \begin{itemize}
  \item
    signals propagate in solid media: copper, fiber, coax
  \end{itemize}
\item
  unguided media:
\item
  \begin{itemize}
  \item
    signals propagate freely, e.g., radio
  \end{itemize}
\end{itemize}

\emph{twisted pair (TP**)}

\begin{itemize}
\item
  two insulated copper wires
\item
  \begin{itemize}
  \item
    Category 5: 100 Mbps, 1 Gpbs Ethernet
  \item
    Category 6: 10Gbps
  \end{itemize}
\end{itemize}

\begin{figure}
\centering
\includegraphics{/Users/whichway/workspace/E_ESSEX/CE265/MD/CE265.assets/image-20210310091722716.png}
\caption{}
\end{figure}

1.23

\emph{coaxial cable:}

\begin{itemize}
\item
  two concentric copper conductors
\item
  bidirectional
\item
  broadband:
\item
  \begin{itemize}
  \item
    multiple channels on cable
  \item
    HFC ( hybrid fiber coax )
  \end{itemize}
\end{itemize}

\begin{figure}
\centering
\includegraphics{/Users/whichway/workspace/E_ESSEX/CE265/MD/CE265.assets/image-20210310091737496.png}
\caption{}
\end{figure}

\emph{fiber optic cable:}

\begin{itemize}
\item
  glass fiber carrying light pulses, each pulse a bit
\item
  high-speed operation:
\item
  \begin{itemize}
  \item
    high-speed point-to-point transmission (e.g., 10's-100's Gpbs
    transmission rate)
  \end{itemize}
\item
  low error rate:
\item
  \begin{itemize}
  \item
    repeaters spaced far apart
  \item
    immune to electromagnetic noise
  \end{itemize}
\end{itemize}

\begin{figure}
\centering
\includegraphics{/Users/whichway/workspace/E_ESSEX/CE265/MD/CE265.assets/image-20210310091829664.png}
\caption{}
\end{figure}

1.24

Physical media: radio

\begin{itemize}
\item
  signal carried in electromagnetic spectrum
\item
  no physical ``wire''
\item
  bidirectional
\item
  propagation environment effects:
\item
  \begin{itemize}
  \item
    reflection
  \item
    obstruction by objects
  \item
    interference
  \end{itemize}
\end{itemize}

\emph{radio link types:}

\begin{itemize}
\item
  terrestrial microwave
\item
  \begin{itemize}
  \item
    e.g. up to 45 Mbps channels
  \end{itemize}
\item
  LAN (e.g., WiFi)
\item
  \begin{itemize}
  \item
    11Mbps, 54 Mbps
  \end{itemize}
\item
  wide-area (e.g., cellular)
\item
  \begin{itemize}
  \item
    3G cellular: \textasciitilde{} few Mbps
  \end{itemize}
\item
  satellite
\item
  \begin{itemize}
  \item
    Kbps to 45Mbps channel (or multiple smaller channels)
  \item
    270 msec end-end delay
  \item
    geosynchronous versus low altitude
  \end{itemize}
\end{itemize}

\hypertarget{13-network-core--packet-switching-circuit-switching-network-structure}{%
\subsection{1.3 network core : packet switching, circuit switching,
network
structure}\label{13-network-core--packet-switching-circuit-switching-network-structure}}

\begin{quote}
\hl{分组交换,电路交换进行对比🆚} ISP
\end{quote}

1.26

The network core

\begin{enumerate}
\def\labelenumi{\arabic{enumi}.}
\item
  mesh of interconnected routers
\item
  packet-switching: hosts break application-layer messages into
  \emph{packets}
\item
  \begin{enumerate}
  \def\labelenumii{\arabic{enumii}.}
  \item
    forward packets from one router to the next, across links on path
    from source to destination
  \item
    each packet transmitted at full link capacity
  \end{enumerate}
\end{enumerate}

\begin{figure}
\centering
\includegraphics{/Users/whichway/workspace/E_ESSEX/CE265/MD/CE265.assets/image-20210308143143569.png}
\caption{}
\end{figure}

1.27 Packet-switching: store-and-forward

\begin{figure}
\centering
\includegraphics{/Users/whichway/workspace/E_ESSEX/CE265/MD/CE265.assets/image-20210302082939772.png}
\caption{}
\end{figure}

\begin{itemize}
\item
  takes \emph{L}/\emph{R} seconds to transmit (push out) \emph{L}-bit
  packet into link at \emph{R} bps
\item
  \emph{store and forward:} entire packet must arrive at router before
  it can be transmitted on next link
\item
  end-end delay = 2\emph{L}/\emph{R} (assuming zero propagation delay)

  \begin{itemize}
  \item
    more on delay shortly \ldots{}
  \end{itemize}
\end{itemize}

\emph{one-hop numerical example:}

\begin{quote}
传输延迟
\end{quote}

\begin{itemize}
\item
  \emph{L} = 7.5 Mbits
\item
  \emph{R} = 1.5 Mbps
\item
  one-hop transmission delay = 5 sec
\end{itemize}

1.28 Packet Switching: queueing delay, loss

\begin{quote}
queueing delay{[}排队延迟{]}
\end{quote}

\begin{figure}
\centering
\includegraphics{/Users/whichway/workspace/E_ESSEX/CE265/MD/CE265.assets/image-20210302083212827.png}
\caption{}
\end{figure}

queuing and loss:

\begin{itemize}
\item
  If arrival rate (in bits) to link exceeds transmission rate of link
  for a period of time:
\item
  \begin{itemize}
  \item
    packets will queue, wait to be transmitted on link
  \item
    packets can be dropped (lost) if memory (buffer) fills up
  \end{itemize}
\end{itemize}

1.29 Two key network-core functions

\emph{routing:} determines source-destination route taken by packets

\begin{quote}
routing 路由
\end{quote}

\begin{itemize}
\item
  \begin{itemize}
  \item
    \emph{routing algorithms}

    \begin{quote}
    路由算法

    通常是根据 ip 地址
    \end{quote}
  \end{itemize}
\end{itemize}

\emph{forwarding**:} move packets from router's input to appropriate
router output

\begin{quote}
forwarding 转发
\end{quote}

\begin{figure}
\centering
\includegraphics{/Users/whichway/workspace/E_ESSEX/CE265/MD/CE265.assets/image-20210302083448665.png}
\caption{}
\end{figure}

1.30 Alternative core: \hl{circuit switching}

\begin{quote}
电路交换, 类似电话网络
\end{quote}

end-end resources allocated to, reserved for ``call'' between source \&
dest:

\begin{enumerate}
\def\labelenumi{\arabic{enumi}.}
\item
  In diagram, each link has four circuits.
\item
  \begin{itemize}
  \item
    call gets 2nd circuit in top link and 1st circuit in right link.
  \end{itemize}
\item
  dedicated resources: no sharing
\item
  \begin{itemize}
  \item
    circuit-like (guaranteed) performance
  \end{itemize}
\item
  circuit segment idle if not used by call \emph{(no sharing)}
\item
  Commonly used in traditional telephone networks
\end{enumerate}

\begin{figure}
\centering
\includegraphics{/Users/whichway/workspace/E_ESSEX/CE265/MD/CE265.assets/image-20210302083838581.png}
\caption{}
\end{figure}

1.31 Circuit switching: FDM versus TDM

\begin{quote}
电路交换, 类似电话网络
\end{quote}

\begin{figure}
\centering
\includegraphics{/Users/whichway/workspace/E_ESSEX/CE265/MD/CE265.assets/image-20210302084035686.png}
\caption{}
\end{figure}

1.32

\emph{packet switching allows more users to use network!}

example:

\begin{itemize}
\item
  1 Mb/s link
\item
  each user:
\item
  \begin{itemize}
  \item
    100 kb/s when ``active''
  \item
    active 10\% of time
  \end{itemize}
\item
\item
  \emph{circuit-switching:}
\item
  \begin{itemize}
  \item
    10 users
  \end{itemize}
\item
  \emph{packet switching:}
\item
  \begin{itemize}
  \item
    with 35 users, probability \textgreater{} 10 active at same time is
    less than .0004
  \end{itemize}
\end{itemize}

\begin{figure}
\centering
\includegraphics{/Users/whichway/workspace/E_ESSEX/CE265/MD/CE265.assets/image-20210302084543471.png}
\caption{}
\end{figure}

\begin{figure}
\centering
\includegraphics{/Users/whichway/workspace/E_ESSEX/CE265/MD/CE265.assets/image-20210302084413569.png}
\caption{}
\end{figure}

1.33

Packet switching versus circuit switching

\textbf{Is packet switching a ``slam dunk winner?''}

\begin{enumerate}
\def\labelenumi{\arabic{enumi}.}
\item
  great for bursty data
\item
  \begin{itemize}
  \item
    resource sharing
  \item
    simpler, no call setup
  \end{itemize}
\item
  excessive congestion possible: packet delay and loss
\item
  \begin{itemize}
  \item
    protocols needed for reliable data transfer, congestion control
  \end{itemize}
\item
  \emph{Q:} How to provide circuit-like behavior?
\item
  \begin{itemize}
  \item
    bandwidth guarantees needed for audio/video apps
  \item
    still an unsolved problem (chapter 7)
  \end{itemize}
\end{enumerate}

\emph{Q:} human analogies of reserved resources (circuit switching)
versus on-demand allocation (packet-switching)?

1.34

Internet structure: network of networks

\begin{enumerate}
\def\labelenumi{\arabic{enumi}.}
\item
  End systems connect to Internet via access ISPs (Internet Service
  Providers)
\item
  \begin{itemize}
  \item
    Residential, company and university ISPs
  \end{itemize}
\item
  Access ISPs in turn must be interconnected.
\item
  \begin{itemize}
  \item
    So that any two hosts can send packets to each other
  \end{itemize}
\item
  Resulting network of networks is very complex
\item
  \begin{itemize}
  \item
    Evolution was driven by economics and national policies
  \end{itemize}
\item
  Let's take a stepwise approach to describe current Internet structure
\end{enumerate}

1.35

\emph{Question:} given \emph{millions} of access ISPs, how to connect
them together?

\begin{figure}
\centering
\includegraphics{/Users/whichway/workspace/E_ESSEX/CE265/MD/CE265.assets/image-20210302091208764.png}
\caption{}
\end{figure}

1.36

\emph{Option:} \emph{connect each access ISP to every other access ISP?}

\begin{figure}
\centering
\includegraphics{/Users/whichway/workspace/E_ESSEX/CE265/MD/CE265.assets/image-20210302091240563.png}
\caption{}
\end{figure}

1.37

\emph{Option:} \emph{connect each access ISP to a global transit ISP?}
\emph{Customer} \emph{and} \emph{provider} \emph{ISPs have economic
agreement.}

\begin{figure}
\centering
\includegraphics{/Users/whichway/workspace/E_ESSEX/CE265/MD/CE265.assets/image-20210302091539492.png}
\caption{}
\end{figure}

1.38

But if one global ISP is viable business, there will be competitors
\ldots.

\begin{figure}
\centering
\includegraphics{/Users/whichway/workspace/E_ESSEX/CE265/MD/CE265.assets/image-20210302091904964.png}
\caption{}
\end{figure}

1.39

But if one global ISP is viable business, there will be competitors
\ldots. which must be interconnected

\begin{figure}
\centering
\includegraphics{/Users/whichway/workspace/E_ESSEX/CE265/MD/CE265.assets/image-20210302091933057.png}
\caption{}
\end{figure}

1.40

\ldots{} and regional networks may arise to connect access nets to ISPS

\begin{figure}
\centering
\includegraphics{/Users/whichway/workspace/E_ESSEX/CE265/MD/CE265.assets/image-20210302092001225.png}
\caption{}
\end{figure}

1.41

\ldots{} and content provider networks (e.g., Google, Microsoft, Akamai
) may run their own network, to bring services, content close to end
users

\begin{quote}
CDN(Content Delivery
Network)是指内容分发网络,也称为内容传送网络,这个概念始于1996年,是美国麻省理工学院的一个研究小组为改善互联网的服务质量而提出
\end{quote}

\begin{figure}
\centering
\includegraphics{/Users/whichway/workspace/E_ESSEX/CE265/MD/CE265.assets/image-20210302092229822.png}
\caption{}
\end{figure}

1.42

\begin{figure}
\centering
\includegraphics{/Users/whichway/workspace/E_ESSEX/CE265/MD/CE265.assets/image-20210302092502255.png}
\caption{}
\end{figure}

\begin{itemize}
\item
  at center: small \# of well-connected large networks
\item
  \begin{itemize}
  \item
    ``tier-1'' commercial ISPs (e.g., Level 3, Sprint, AT\&T, NTT),
    national \& international coverage
  \item
    content provider network (e.g, Google): private network that
    connects it data centers to Internet, often bypassing tier-1,
    regional ISPs
  \end{itemize}
\end{itemize}

1.43

Tier-1 ISP: e.g., Sprint

\begin{figure}
\centering
\includegraphics{/Users/whichway/workspace/E_ESSEX/CE265/MD/CE265.assets/image-20210302092721023.png}
\caption{}
\end{figure}

\begin{figure}
\centering
\includegraphics{/Users/whichway/workspace/E_ESSEX/CE265/MD/CE265.assets/image-20210302092726778.png}
\caption{}
\end{figure}

\hypertarget{14-delay-loss-throughput-in-networks}{%
\subsection{⭐️1.4 delay, loss, throughput in
networks}\label{14-delay-loss-throughput-in-networks}}

\begin{quote}
分组交换网中的时延、丢包和吞吐量

A.结点总时延

\begin{quote}
结点总时延=结点处理时延+排队时延+传输时延+传播时延

a.结点处理时延

检查分组首部和决定将分组导向何处所需要的时间 b.排队时延
在输出队列中等待传输的时延,取决于到达队列的速率、链路的传输速率和到达流量的性质(周期性到达还是突发形式到达)
1.流量强度:
令a表示分组到达队列的平均速率(单位:pkt/s);R为链路传输速率,即从链路推出比特的速率(b/s);每个分组由比特组成。
则比率La/R为流量强度.

\begin{quote}
若La/R>1,排队队列趋向于无界增加,排队时延趋向于无穷大;
若La/R≦1,排队时延随着流量强度的增加而增加。 设计系统时流量强度不能大于1
2.丢包 队列容量有限。满队列情况下,新分组将会被丢弃。 c.传输时延
将所有分组的比特推(传输)向链路的时间。与分组长度和链路传输速率有关
d.传播时延
链路起点向目标路由器传播所需要的时间。与路由器之间的距离和传播速率有关
B.端到端时延
假设源主机和目的主机之间有N-1台路由器。每台路由器和源主机的处理时延d(proc)、路由器和源主机的传输时延d(trans)、链路的传播时延d(prop).其中d(trans)=L/R.L为分组长度、R为路由器和源主机的输出速率.
则端到端时延d(end-end)=N(d(proc)+d(trans)+d(prop)) C.计算机网络中的吞吐
文件传输过程中,接收方接收文件的速率。如果文件由F比特组成,接收文件需要T秒,则平均吞吐量为F/T
bps
假设服务器和文件之间有N条链路,其传输速率分别是R1、R2、R3\ldots\ldots RN。则文件传输的吞吐量为min\{R1,R2\ldots\ldots RN\}。即整条路径的最小传输速率。
吞吐量不仅取决于沿着路径的传输速率,而且取决于干扰流量
\end{quote}
\end{quote}
\end{quote}

1.45

How do loss and delay occur?

packets \emph{queue} in router buffers

\begin{itemize}
\item
  packet arrival rate to link (temporarily) exceeds output link capacity
\item
  packets queue, wait for turn
\end{itemize}

\begin{figure}
\centering
\includegraphics{/Users/whichway/workspace/E_ESSEX/CE265/MD/CE265.assets/image-20210302092930254.png}
\caption{}
\end{figure}

1.46

\hypertarget{four-sources-of-packet-delay}{%
\subsubsection{\texorpdfstring{\hl{⭐️Four sources of packet
delay}}{⭐️Four sources of packet delay}}\label{four-sources-of-packet-delay}}

\begin{figure}
\centering
\includegraphics{/Users/whichway/workspace/E_ESSEX/CE265/MD/CE265.assets/image-20210302093010818.png}
\caption{}
\end{figure}

\begin{enumerate}
\def\labelenumi{\arabic{enumi}.}
\item
  ==\(d_{proc}\): nodal processing ==
\end{enumerate}

\begin{quote}
1.节点时延
\end{quote}

\begin{enumerate}
\def\labelenumi{\arabic{enumi}.}
\item
  check bit errors
\item
  determine output link
\item
  typically \textless{} msec
\item
  \hl{\(d_{queue}\): queueing delay}
\end{enumerate}

\begin{quote}
排队时延
\end{quote}

\begin{enumerate}
\def\labelenumi{\arabic{enumi}.}
\item
  time waiting at output link for transmission
\item
  depends on congestion level of router
\end{enumerate}

\begin{figure}
\centering
\includegraphics{/Users/whichway/workspace/E_ESSEX/CE265/MD/CE265.assets/image-20210302093020605.png}
\caption{}
\end{figure}

1.47

\begin{enumerate}
\def\labelenumi{\arabic{enumi}.}
\item
  \hl{\emph{d}trans: transmission delay:}
\end{enumerate}

\begin{quote}
传输时延
\end{quote}

\begin{itemize}
\item
  \emph{L}: packet length (bits)
\item
  \emph{R}: link \emph{bandwidth (bps)}
\item
  \emph{d**trans} \emph{= L/R}
\end{itemize}

\begin{enumerate}
\def\labelenumi{\arabic{enumi}.}
\item
  \hl{\emph{d}prop: propagation delay:}
\end{enumerate}

\begin{quote}
传播时延
\end{quote}

\begin{itemize}
\item
  \emph{d}: length of physical link
\item
  \emph{s}: propagation speed in medium (\textasciitilde2x108 m/sec)
\item
  \emph{d}prop = \emph{d}/\emph{s}
\end{itemize}

\begin{figure}
\centering
\includegraphics{/Users/whichway/workspace/E_ESSEX/CE265/MD/CE265.assets/image-20210302093132131.png}
\caption{}
\end{figure}

* Check out the Java applet for an interactive animation on trans vs.
prop delay

1.48

Caravan analogy

\begin{quote}
卡车定理传输时延
\end{quote}

\begin{figure}
\centering
\includegraphics{/Users/whichway/workspace/E_ESSEX/CE265/MD/CE265.assets/image-20210302093259313.png}
\caption{}
\end{figure}

\begin{itemize}
\item
  cars ``propagate'' at 100 km/hr
\item
  toll booth takes 12 sec to service car (bit transmission time)
\item
  car\textasciitilde bit; caravan \textasciitilde{} packet
\end{itemize}

\emph{Q:} How long until caravan is lined up before 2nd toll booth?

\begin{itemize}
\item
  time to ``push'' entire caravan through toll booth onto highway =
  12*10 = 120 sec
\item
  time for last car to propagate from 1st to 2nd toll both:
  100km/(100km/hr)= 1 hr
\item
  \emph{A:} 62 minutes
\end{itemize}

1.49

\begin{quote}
改变速度
\end{quote}

\begin{itemize}
\item
  suppose cars now ``propagate'' at 1000 km/hr
\item
  and suppose toll booth now takes one min to service a car
\item
  \emph{Q:} Will cars arrive to 2nd booth before all cars serviced at
  first booth?
\end{itemize}

1.50

Queueing delay (revisited)

\begin{quote}
排队时延
\end{quote}

\begin{figure}
\centering
\includegraphics{/Users/whichway/workspace/E_ESSEX/CE265/MD/CE265.assets/image-20210302094320911.png}
\caption{}
\end{figure}

\begin{itemize}
\item
  \emph{R:} link bandwidth (bps)
\item
  \emph{L:} packet length (bits)
\item
  a: average packet arrival rate
\end{itemize}

\begin{itemize}
\item
  \emph{La/R} \textasciitilde{} 0: avg. queueing delay small
\item
  \emph{La/R} -\textgreater{} 1: avg. queueing delay large
\item
  \emph{La/R} \textgreater{} 1: more ``work'' arriving than can be
  serviced, average delay infinite!
\end{itemize}

1.51

win

\begin{verbatim}
tracert www.baidu.com	
\end{verbatim}

linux/unix

\begin{verbatim}
traceroute www.baidu.com
\end{verbatim}

``Real'' Internet delays and routes

\begin{quote}
跳数
\end{quote}

\begin{itemize}
\item
  what do ``real'' Internet delay \& loss look like?
\item
  traceroute program: provides delay measurement from source to router
  along end-end Internet path towards destination. For all \emph{i:}
\item
  \begin{itemize}
  \item
    sends three packets that will reach router \emph{i} on path towards
    destination
  \item
    router \emph{i} will return packets to sender
  \item
    sender times interval between transmission and reply.
  \end{itemize}
\end{itemize}

\begin{figure}
\centering
\includegraphics{/Users/whichway/workspace/E_ESSEX/CE265/MD/CE265.assets/image-20210302095005602.png}
\caption{}
\end{figure}

1.52

traceroute: gaia.cs.umass.edu to \url{www.eurecom.fr}

\begin{figure}
\centering
\includegraphics{/Users/whichway/workspace/E_ESSEX/CE265/MD/CE265.assets/image-20210308140533714.png}
\caption{}
\end{figure}

1.53

Packet loss

\begin{enumerate}
\def\labelenumi{\arabic{enumi}.}
\item
  queue (aka buffer) preceding link in buffer has finite capacity
\item
  packet arriving to full queue dropped (aka lost)
\item
  lost packet may be retransmitted by previous node, by source end
  system, or not at all
\end{enumerate}

\begin{figure}
\centering
\includegraphics{/Users/whichway/workspace/E_ESSEX/CE265/MD/CE265.assets/image-20210308140642598.png}
\caption{}
\end{figure}

1.54

Throughput

\begin{itemize}
\item
  \emph{throughput:} rate (bits/time unit) at which bits transferred
  between sender/receiver
\item
  \begin{enumerate}
  \def\labelenumi{\arabic{enumi}.}
  \item
    \emph{instantaneous:} rate at given point in time
  \item
    \emph{average:} rate over longer period of time
  \end{enumerate}
\end{itemize}

\begin{figure}
\centering
\includegraphics{/Users/whichway/workspace/E_ESSEX/CE265/MD/CE265.assets/image-20210308140739214.png}
\caption{}
\end{figure}

\begin{itemize}
\item
  \emph{R**s} \emph{\textless{} R**c} What is average end-end
  throughput?
\end{itemize}

\begin{figure}
\centering
\includegraphics{/Users/whichway/workspace/E_ESSEX/CE265/MD/CE265.assets/image-20210308140823962.png}
\caption{}
\end{figure}

\begin{itemize}
\item
  \emph{R**s} \emph{\textgreater{} R**c} What is average end-end
  throughput?
\end{itemize}

\begin{figure}
\centering
\includegraphics{/Users/whichway/workspace/E_ESSEX/CE265/MD/CE265.assets/image-20210308140836611.png}
\caption{}
\end{figure}

\begin{itemize}
\item
  \emph{bottleneck link}
\end{itemize}

link on end-end path that constrains end-end throughput

1.56

Throughput: Internet scenario

\begin{itemize}
\item
  per-connection end-end throughput: min(Rc,Rs,R/10)
\item
  in practice: Rc or Rs is often bottleneck
\end{itemize}

\begin{figure}
\centering
\includegraphics{/Users/whichway/workspace/E_ESSEX/CE265/MD/CE265.assets/image-20210308141312358.png}
\caption{}
\end{figure}

\hypertarget{15-protocol-layersux534fux8baeux5c42ux6b21-service-modelsux670dux52a1ux6a21ux578b}{%
\subsection{1.5 protocol layers{[}协议层次{]}, service
models{[}服务模型{]}}\label{15-protocol-layersux534fux8baeux5c42ux6b21-service-modelsux670dux52a1ux6a21ux578b}}

\begin{quote}
5.协议层次及其服务模型

A.分层的体系结构

a.协议分层(因特网五层协议栈)(概述)

1.应用层

\begin{itemize}
\item
  网络应用程序及他们的应用层协议存留于应用层中。不同端系统中的应用程序通过应用层协议交换信息分组(报文)。应用层协议有:HTTP、SMTP、FTP、DNS。
\end{itemize}

2.传输层

\begin{itemize}
\item
  传输层在应用程序端点之间传送报文段(应用层报文)。传输层协议有:TCP、UDP
\end{itemize}

3.网络层

\begin{itemize}
\item
  网络层将数据报(网络层分组)从一台主机移动到另一台主机。网络层协议:IP协议、路由选择协议
\end{itemize}

4.链路层

\begin{itemize}
\item
  将帧(链路层分组)从一个结点移动到路径的下一个结点。链路层协议:以太网、Wifi和电缆接入网的DOCSIS协议
\end{itemize}

5.物理层

将帧中的一个个比特一个结点移动到路径的下一个结点。物理层协议与链路传输媒体相关。

b.OSI模型

【7层】自上而下:应用层、表示层、会话层、传输层、网路层、链路层、物理层

B.封装 (encapsulated)

\begin{figure}
\centering
\includegraphics{/Users/whichway/workspace/E_ESSEX/CE265/MD/CE265.assets/image-20210420084241129.png}
\caption{}
\end{figure}

每一层,一个分组具有两种类型的字段:首部字段和有效载荷字段。有效字段通常来源于上一层的分组。
\end{quote}

1.58

Protocol ``layers''

\emph{Networks are complex,}

\emph{with many ``pieces'':}

\begin{itemize}
\item
  \begin{enumerate}
  \def\labelenumi{\arabic{enumi}.}
  \item
    hosts
  \item
    routers
  \item
    links of various media
  \item
    applications
  \item
    protocols
  \item
    hardware, software
  \end{enumerate}
\end{itemize}

\emph{Question:}

\begin{itemize}
\item
  is there any hope of \emph{organizing} structure of network?
\end{itemize}

\begin{itemize}
\item
  \ldots. or at least our discussion of networks?
\end{itemize}

1.59

Organization of air travel

\begin{figure}
\centering
\includegraphics{/Users/whichway/workspace/E_ESSEX/CE265/MD/CE265.assets/image-20210308143924197.png}
\caption{}
\end{figure}

\begin{itemize}
\item
  a series of steps
\end{itemize}

1.60

Layering of airline functionality

\begin{figure}
\centering
\includegraphics{/Users/whichway/workspace/E_ESSEX/CE265/MD/CE265.assets/image-20210308143943959.png}
\caption{}
\end{figure}

\emph{layers:} each layer implements a service

\begin{itemize}
\item
  \begin{enumerate}
  \def\labelenumi{\arabic{enumi}.}
  \item
    via its own internal-layer actions
  \item
    relying on services provided by layer below
  \end{enumerate}
\end{itemize}

1.61

Why layering?

dealing with complex systems:

\begin{enumerate}
\def\labelenumi{\arabic{enumi}.}
\item
  explicit structure allows identification, relationship of complex
  system's pieces
\item
  \begin{itemize}
  \item
    layered \emph{reference model} for discussion
  \end{itemize}
\item
  modularization eases maintenance, updating of system
\item
  \begin{itemize}
  \item
    change of implementation of layer's service transparent to rest of
    system
  \item
    e.g., change in gate procedure doesn't affect rest of system
  \end{itemize}
\item
  layering considered harmful?
\end{enumerate}

1.62

\hypertarget{internet-protocol-stack}{%
\subsubsection{\texorpdfstring{\hl{⭐️Internet protocol
stack}}{⭐️Internet protocol stack}}\label{internet-protocol-stack}}

\begin{enumerate}
\def\labelenumi{\arabic{enumi}.}
\item
  \emph{application:} supporting network applications

  \begin{quote}
  应用层
  \end{quote}
\item
  \begin{itemize}
  \item
    FTP, SMTP, HTTP
  \end{itemize}
\item
  \emph{transport:} process-process data transfer

  \begin{quote}
  传输层
  \end{quote}
\item
  \begin{itemize}
  \item
    TCP, UDP
  \end{itemize}
\item
  \emph{network:} routing of datagrams from source to destination

  \begin{quote}
  网络层
  \end{quote}
\item
  \begin{itemize}
  \item
    IP, routing protocols
  \end{itemize}
\item
  \emph{link:} data transfer between neighboring network elements

  \begin{quote}
  链路层
  \end{quote}
\item
  \begin{itemize}
  \item
    Ethernet, 802.11 (WiFi), PPP
  \end{itemize}
\item
  \emph{physical:} bits ``on the wire''

  \begin{quote}
  物理层
  \end{quote}
\end{enumerate}

1.63

ISO/OSI reference model

\begin{enumerate}
\def\labelenumi{\arabic{enumi}.}
\item
  \emph{presentation:} allow applications to interpret meaning of data,
  e.g., encryption, compression, machine-specific conventions

  \begin{quote}
  表示层
  \end{quote}
\item
  \emph{session:} synchronization, checkpointing, recovery of data
  exchange

  \begin{quote}
  会话层
  \end{quote}
\item
  Internet stack ``missing'' these layers!
\item
  \begin{itemize}
  \item
    these services, \emph{if needed,} must be implemented in application
  \item
    needed?
  \end{itemize}
\end{enumerate}

1.64

Encapsulation{[}封装📦{]}

\begin{figure}
\centering
\includegraphics{/Users/whichway/workspace/E_ESSEX/CE265/MD/CE265.assets/image-20210308145142499.png}
\caption{}
\end{figure}

\hypertarget{16-networks-under-attack-security}{%
\subsection{1.6 networks under attack:
security}\label{16-networks-under-attack-security}}

6.面对攻击的网络

网络安全领域主要探讨以下问题:hacker如何攻击计算机网络,以及我们如何防御以免受他们的攻击,或者是更好的是设计能够事先免除这样的攻击的新型体系结构。

A.恶意软件(自我复制)

病毒:是一种需要某种形式的用户交互来感染用户设备的恶意软件。

蠕虫:是一种无需任何明显用户交互就能进入设备的恶意软件

B.拒绝服务攻击(DDOS)

1.弱点攻击。向一台目标主机运行的易受攻击的应用程序或操作系统发送精细制作的报文

2.带宽洪泛。攻击者向目标主机发送大量的分组,使得目标的接入链路拥塞,合法分组无法到达服务器。

3.连接洪泛。攻击者在目标主机中创建大量的半开或全开TCP连接使主机停止接受合法的连接。

C.分组嗅探器

在无线传输设备的附近放置一台被动的接收机,就能得到传输的每个分组的副本

D.IP哄骗

将具有虚假源地址的分组注入因特网的能力被称为IP哄骗。

1.66

Network security

\begin{enumerate}
\def\labelenumi{\arabic{enumi}.}
\item
  field of network security:
\item
  \begin{enumerate}
  \def\labelenumii{\arabic{enumii}.}
  \item
    how bad guys can attack computer networks
  \item
    how we can defend networks against attacks
  \item
    how to design architectures that are immune to attacks
  \end{enumerate}
\item
  Internet not originally designed with (much) security in mind
\item
  \begin{enumerate}
  \def\labelenumii{\arabic{enumii}.}
  \item
    \emph{original vision:} ``a group of mutually trusting users
    attached to a transparent network'' ☺
  \item
    Internet protocol designers playing ``catch-up''
  \item
    security considerations in all layers!
  \end{enumerate}
\end{enumerate}

1.67

Bad guys: put malware into hosts via Internet

\begin{enumerate}
\def\labelenumi{\arabic{enumi}.}
\item
  malware can get in host from:
\item
  \begin{itemize}
  \item
    \emph{virus:} self-replicating infection by receiving/executing
    object (e.g., e-mail attachment)
  \item
    \emph{worm:} self-replicating infection by passively receiving
    object that gets itself executed
  \end{itemize}
\item
  spyware malware can record keystrokes, web sites visited, upload info
  to collection site
\item
  infected host can be enrolled in botnet, used for spam. DDoS attacks
\end{enumerate}

1.68

Bad guys: attack server, network infrastructure

\hypertarget{denial-of-service-dos}{%
\subsubsection{\texorpdfstring{\hl{⭐️Denial of Service
(DoS)}}{⭐️Denial of Service (DoS)}}\label{denial-of-service-dos}}

\hl{\emph{Denial of Service (DoS):} attackers make resources (server,
bandwidth) unavailable to legitimate traffic by overwhelming resource
with bogus traffic}

\begin{enumerate}
\def\labelenumi{\arabic{enumi}.}
\item
  select target
\item
  break into hosts around the network (see botnet)
\item
  send packets to target from compromised hosts
\end{enumerate}

\begin{figure}
\centering
\includegraphics{/Users/whichway/workspace/E_ESSEX/CE265/MD/CE265.assets/image-20210308150918715.png}
\caption{}
\end{figure}

1.69 Bad guys can sniff packets

\emph{packet ``sniffing'':}

\begin{itemize}
\item
  \begin{enumerate}
  \def\labelenumi{\arabic{enumi}.}
  \item
    broadcast media (shared ethernet, wireless)
  \item
    promiscuous network interface reads/records all packets (e.g.,
    including passwords!) passing by
  \end{enumerate}
\end{itemize}

\begin{figure}
\centering
\includegraphics{/Users/whichway/workspace/E_ESSEX/CE265/MD/CE265.assets/image-20210308151335887.png}
\caption{}
\end{figure}

\begin{itemize}
\item
  \begin{itemize}
  \item
    wireshark software used for end-of-chapter labs is a (free)
    packet-sniffer
  \end{itemize}
\end{itemize}

1.70

Bad guys can use fake addresses

\emph{IP spoofing:} send packet with false source address

\begin{figure}
\centering
\includegraphics{/Users/whichway/workspace/E_ESSEX/CE265/MD/CE265.assets/image-20210308151421362.png}
\caption{}
\end{figure}

\hypertarget{17-history}{%
\subsection{1.7 history}\label{17-history}}

1.72

Internet history

1961-1972: Early packet-switching principles

\begin{itemize}
\item
  1961: Kleinrock - queueing theory shows effectiveness of
  packet-switching
\item
  1964: Baran - packet-switching in military nets
\item
  1967: ARPAnet conceived by Advanced Research Projects Agency
\item
  1969: first ARPAnet node operational
\item
  1972:
\item
  \begin{itemize}
  \item
    ARPAnet public demo
  \item
    NCP (Network Control Protocol) first host-host protocol
  \item
    first e-mail program
  \item
    ARPAnet has 15 nodes
  \end{itemize}
\end{itemize}

\begin{figure}
\centering
\includegraphics{/Users/whichway/workspace/E_ESSEX/CE265/MD/CE265.assets/image-20210308151708257.png}
\caption{}
\end{figure}

1972-1980: Internetworking, new and proprietary nets

\begin{itemize}
\item
  1970: ALOHAnet satellite network in Hawaii
\item
  1974: Cerf and Kahn - architecture for interconnecting networks
\item
  1976: Ethernet at Xerox PARC
\item
  late70's: proprietary architectures: DECnet, SNA, XNA
\item
  late 70's: switching fixed length packets (ATM precursor)
\item
  1979: ARPAnet has 200 nodes
\end{itemize}

Cerf and Kahn's internetworking principles:

\begin{itemize}
\item
  \begin{enumerate}
  \def\labelenumi{\arabic{enumi}.}
  \item
    minimalism, autonomy - no internal changes required to interconnect
    networks
  \item
    best effort service model
  \item
    stateless routers
  \item
    decentralized control
  \end{enumerate}
\end{itemize}

define today's Internet architecture

1980-1990: new protocols, a proliferation of networks

\begin{itemize}
\item
  1983: deployment of TCP/IP
\item
  1982: smtp e-mail protocol defined
\item
  1983: DNS defined for name-to-IP-address translation
\item
  1985: ftp protocol defined
\item
  1988: TCP congestion control
\item
  new national networks: Csnet, BITnet, NSFnet, Minitel
\item
  100,000 hosts connected to confederation of networks
\end{itemize}

\emph{1990, 2000's: commercialization, the Web, new apps}

\begin{itemize}
\item
  early 1990's: ARPAnet decommissioned
\item
  1991: NSF lifts restrictions on commercial use of NSFnet
  (decommissioned, 1995)
\item
  early 1990s: Web
\item
  \begin{itemize}
  \item
    hypertext {[}Bush 1945, Nelson 1960's{]}
  \item
    HTML, HTTP: Berners-Lee
  \item
    1994: Mosaic, later Netscape
  \item
    late 1990's: commercialization of the Web
  \end{itemize}
\end{itemize}

late 1990's -- 2000's:

\begin{itemize}
\item
  more killer apps: instant messaging, P2P file sharing
\item
  network security to forefront
\item
  est. 50 million host, 100 million+ users
\item
  backbone links running at Gbps
\end{itemize}

\emph{2005-present}

\begin{itemize}
\item
  \textasciitilde750 million hosts
\item
  \begin{itemize}
  \item
    Smartphones and tablets
  \end{itemize}
\item
  Aggressive deployment of broadband access
\item
  Increasing ubiquity of high-speed wireless access
\item
  Emergence of online social networks:
\item
  \begin{itemize}
  \item
    Facebook: soon one billion users
  \end{itemize}
\item
  Service providers (Google, Microsoft) create their own networks
\item
  \begin{itemize}
  \item
    Bypass Internet, providing ``instantaneous'' access to search, emai,
    etc.
  \end{itemize}
\item
  E-commerce, universities, enterprises running their services in
  ``cloud'' (eg, Amazon EC2)
\end{itemize}

\hypertarget{18-introduction--summary}{%
\subsection{1.8 Introduction : summary}\label{18-introduction--summary}}

\emph{covered a ``ton'' of material!}

\begin{itemize}
\item
  Internet overview
\item
  what's a protocol?
\item
  network edge, core, access network
\item
  \begin{itemize}
  \item
    packet-switching versus circuit-switching
  \item
    Internet structure
  \end{itemize}
\item
  performance: loss, delay, throughput
\item
  layering, service models
\item
  security
\item
  history
\end{itemize}

\emph{you now have:}

\begin{itemize}
\item
  context, overview, ``feel'' of networking
\item
  more depth, detail \emph{to follow!}
\end{itemize}

\hypertarget{chapter-2-application-layer}{%
\section{Chapter 2 Application
Layer}\label{chapter-2-application-layer}}

2.1 principles of network applications

2.2 Web and HTTP

2.3 FTP

2.4 electronic mail : SMTP, POP3, IMAP

2.5 DNS

2.6 P2P applications

2.7 socket programming with UDP and TCP

\begin{quote}
应用层是我们学习协议非常好的起点,它最为我们所熟悉。我们熟悉的很多应用就是建立在这些将要学习的协议基础上的。通过对应用层的学习,将有助于我们认知协议有关知识,将使我们了解到很多问题,这些问题当我们学习运输层、网络层及数据链路层协议时也同样会碰到。

研发网 络应用 程序的核 心 是写 出能够运行在不
同的端系统和通过网络彼此通信的程序。 例如,在W e b应用程序中,有两 个
互相通信的不 同的程序:一
个是运行在用户主机(桌面机、膝上机、平板电脑、智能电话等)上的浏览器程序;另一个是运行在Web服务器主机上的W
e b 服务 器程序。 另 一 个 例子是P 2 P文件 共 享系统,在参 与文件 共
享的社区中的每台主机中都有一个程序.:,在这种情况下,在各台主
机中的这些程序可能都 是类似的或相同的。

因此,当研发新应用程序时,你需要编写将在多台端系统上运行的软件。例如,该软件能够用C、J
a v a或P y t h o
n来编写重要的是,你不需要写在网络核心设备如路由器或链路层交换机上运行的软件」即使你要为网络核心设备写应用程序软件,你也不能做到这一
点。如 我们在第1章所知 ,以 及如 图1 - 2 4所 显示 的
那样,网络核心设备并不在应用层上 起作用,而仅在较低层起作用,特别是位
于网络层及 下面层次这种基本设计,也 即将应用软件限制在端系统(如阁2 -
1所示)的方法,促进了大量的网络应用程序的迅速研发和部署。
\end{quote}

\hypertarget{21-principles-of-network-applications}{%
\subsection{2.1 principles of network
applications}\label{21-principles-of-network-applications}}

our goals:

\begin{itemize}
\item
  conceptual, implementation aspects of network application protocols
\item
  \begin{enumerate}
  \def\labelenumi{\arabic{enumi}.}
  \item
    transport-layer service models
  \item
    client-server paradigm
  \item
    peer-to-peer paradigm
  \end{enumerate}
\item
  learn about protocols by examining popular application-level protocols
\item
  \begin{enumerate}
  \def\labelenumi{\arabic{enumi}.}
  \item
    HTTP
  \item
    FTP
  \item
    SMTP / POP3 / IMAP
  \item
    DNS
  \end{enumerate}
\item
  creating network applications
\item
  \begin{itemize}
  \item
    socket API
  \end{itemize}
\end{itemize}

\begin{quote}
当进行软件编码之前,应当对应用程序有一个宽泛的体系结构计划。记住应用程序的体系结构明显不同于网络的体系结构(例如在第1章中所讨论的5层因特网体系结构)。

从应用程序研发者的角度看,网络体系结构是固定的,并为应用程序提供了特定的服务集合。在另一方面,应用程序体系结构(application
architectur e)由应用程 序研发者设
计,规定了如何在各种端系统上组织该应用程序„在选择应用程序体系结构时,应用程序研发者很可能利用现代网络应用程序中所使用的两种主流体系结构之一:

客户-服务器体系结构

对等(P2P)体系结构。

在客户-服务器体系结构(client-server architectur e)中,有一个总是打
开的主机称为 服 务 器 , 它 服 务 于 来 自 许 多 其 他 称 为 客 户 的 主
机 的 请 求 - 个 经 典 的 例 子 是 W e b 应 用 程序 , 其 中 总 是 打 开
的 W e b 服 务 器 服 务 于 来 自 浏 览 器 ( 运 行 在 客 户 主 机 h )
的 请 求 : , 当W e b服务
器接收到来自某客户对某对象的请求时,它向该客户发送所请求的对象作 为 响
应。 值得 注 意的是利用客 户- 服务 器 体系结构.客户相互
之间不直接通信;例如,在W e b 应用中 两 个浏览器并不直接通信 。客户-服务
器 体系结构的另一个特征是该服务器具有固 定 的 . 周 知 的地 址, 该地
址称为1 P地 址(我们将 很 快讨 论它〉 。因为该服务 器 具有 固 定的、
周知的地址,并且因为该服务 器总是打开的,客户总是能够通过向该服务
器的地址 发 送 分 组 来 与 其 联 系 。 具 有 客 户- 服务 器 体系 结构的
非常著名的应用 程序包括W e b 、 FTP、Telnet和电子邮件。
\end{quote}

2.4

Some network apps

\begin{itemize}
\item
  e-mail
\item
  web
\item
  text messaging
\item
  remote login
\item
  P2P file sharing
\item
  multi-user network games
\item
  streaming stored video (YouTube, Hulu, Netflix)
\item
  voice over IP (e.g., Skype)
\item
  real-time video conferencing
\item
  social networking
\item
  search
\item
  \ldots{}
\item
  \ldots{}
\end{itemize}

2.5

Creating a network app

write programs that:

\begin{itemize}
\item
  run on (different) \emph{end systems}
\item
  communicate over network
\item
  e.g., web server software communicates with browser software
\end{itemize}

no need to write software for network-core devices

\begin{itemize}
\item
  network-core devices do not run user applications
\item
  applications on end systems allows for rapid app development,
  propagation
\end{itemize}

\begin{figure}
\centering
\includegraphics{/Users/whichway/workspace/E_ESSEX/CE265/MD/CE265.assets/image-20210309093711875.png}
\caption{}
\end{figure}

2.6

\hypertarget{application-architectures}{%
\subsubsection{\texorpdfstring{\hl{⭐️Application
architectures}}{⭐️Application architectures}}\label{application-architectures}}

possible structure of applications:

\begin{enumerate}
\def\labelenumi{\arabic{enumi}.}
\item
  client-server
\item
  peer-to-peer (P2P)
\end{enumerate}

2.7

\hypertarget{client-server-architecture}{%
\subsubsection{\texorpdfstring{\hl{⭐️Client-server
architecture}}{⭐️Client-server architecture}}\label{client-server-architecture}}

server:

\begin{itemize}
\item
  always-on host
\item
  permanent IP address
\item
  data centers for scaling
\end{itemize}

clients:

\begin{itemize}
\item
  communicate with server
\item
  may be intermittently connected
\item
  may have dynamic IP addresses
\item
  do not communicate directly with each other
\end{itemize}

\begin{figure}
\centering
\includegraphics{/Users/whichway/workspace/E_ESSEX/CE265/MD/CE265.assets/image-20210309093859483.png}
\caption{}
\end{figure}

2.8

\hypertarget{p2p-architecture}{%
\subsubsection{\texorpdfstring{\hl{⭐️P2P
architecture}}{⭐️P2P architecture}}\label{p2p-architecture}}

\begin{itemize}
\item
  \emph{no} always-on server
\item
  arbitrary end systems directly communicate
\item
  peers request service from other peers, provide service in return to
  other peers
\item
  \begin{itemize}
  \item
    \emph{self scalability} -- new peers bring new service capacity, as
    well as new service demands
  \end{itemize}
\item
  peers are intermittently connected and change IP addresses
\item
  \begin{itemize}
  \item
    complex management
  \end{itemize}
\end{itemize}

\begin{figure}
\centering
\includegraphics{/Users/whichway/workspace/E_ESSEX/CE265/MD/CE265.assets/image-20210309093928890.png}
\caption{}
\end{figure}

2.9

Processes communicating \textbf{进程通信}

\emph{process:} program running within a host

\begin{itemize}
\item
  within same host, two processes communicate using inter-process
  communication (defined by OS)
\item
  processes in different hosts communicate by exchanging messages
\end{itemize}

clients, servers

\emph{client process:} process that initiates communication

\emph{server process:} process that waits to be contacted

\begin{itemize}
\item
  aside: applications with P2P architectures have client processes \&
  server processes
\end{itemize}

2.10

Sockets

\begin{itemize}
\item
  process sends/receives messages to/from its socket
\item
  socket analogous to door
\item
  \begin{itemize}
  \item
    sending process shoves message out door
  \item
    sending process relies on transport infrastructure on other side of
    door to deliver message to socket at receiving process
  \end{itemize}
\end{itemize}

\begin{figure}
\centering
\includegraphics{/Users/whichway/workspace/E_ESSEX/CE265/MD/CE265.assets/image-20210309094046064.png}
\caption{}
\end{figure}

2.11

Addressing processes

\begin{itemize}
\item
  to receive messages, process must have \emph{identifier}
\item
  host device has unique 32-bit IP address
\item
  \emph{Q:} does IP address of host on which process runs suffice for
  identifying the process?
\item
  \emph{A:} no, \emph{many} processes can be running on same host
\end{itemize}

\begin{enumerate}
\def\labelenumi{\arabic{enumi}.}
\item
  \emph{identifier} includes both IP address and port numbers associated
  with process on host.
\item
  example port numbers:

  \begin{itemize}
  \item
    HTTP server: 80
  \item
    mail server: 25
  \end{itemize}
\item
  to send HTTP message to gaia.cs.umass.edu web server:

  \begin{itemize}
  \item
    IP address: 128.119.245.12
  \item
    port number: 80
  \end{itemize}
\item
  more shortly\ldots{}
\end{enumerate}

2.12

App-layer protocol defines

\begin{itemize}
\item
  types of messages exchanged,
\item
  \begin{itemize}
  \item
    e.g., request, response
  \end{itemize}
\item
  message syntax:
\item
  \begin{itemize}
  \item
    what fields in messages \& how fields are delineated
  \end{itemize}
\item
  message semantics
\item
  \begin{itemize}
  \item
    meaning of information in fields
  \end{itemize}
\item
  rules for when and how processes send \& respond to messages
\end{itemize}

open protocols:

\begin{itemize}
\item
  defined in RFCs
\item
  allows for interoperability
\item
  e.g., HTTP, SMTP
\end{itemize}

proprietary protocols:

\begin{itemize}
\item
  e.g., Skype
\end{itemize}

2.13

What transport service does an app need?

\begin{enumerate}
\def\labelenumi{\arabic{enumi}.}
\item
  data integrity
\end{enumerate}

\begin{itemize}
\item
  some apps (e.g., file transfer, web transactions) require 100\%
  reliable data transfer
\item
  other apps (e.g., audio) can tolerate some loss
\end{itemize}

\begin{enumerate}
\def\labelenumi{\arabic{enumi}.}
\item
  timing
\end{enumerate}

\begin{itemize}
\item
  some apps (e.g., Internet telephony, interactive games) require low
  delay to be ``effective''
\end{itemize}

\begin{enumerate}
\def\labelenumi{\arabic{enumi}.}
\item
  throughput
\end{enumerate}

\begin{itemize}
\item
  some apps (e.g., multimedia) require minimum amount of throughput to
  be ``effective''
\item
  other apps (``elastic apps'') make use of whatever throughput they get
\end{itemize}

\begin{quote}
\begin{enumerate}
\def\labelenumi{\arabic{enumi}.}
\item
  定时 运 输层协议也能提供定时保证。如同 具有吞
  吐量保证那样,定时保证能够以多种形式实现。 一 个
  保证的例子如:发送方注人进套接字中的每个比特到达接收方的套接字不 迟于1
  0 0 m s。这种服务将对交
  互式实时应用程序有吸引力,如因特网电话、虚拟环境、 电 话 会 议 和 多
  方 游 戏 , 所 有 这 些服务 为 了有 效 性而 要求 数 据交
  付有严格的时间 限制 ( 参 见第7章,{[} G a u t h i e r 1 9 9 9 ; R a m
  j e e 1 9 9 4 {]})。 例如,在因特网电话中,较长的时延会导致会话中
  出现不自然的停顿;在多方游戏和虚拟互 动环境中,在做 出 动
  作并看到来自环境 ( 如 来 自 位 于 端 到 端 连 接 中 另 一 端 点 的 玩
  家 ) 的 响 应 之 间 , 较 长 的 时 延 使 得 它 失 去
  真实感。对于非实时的应用,较低的时延总比较高的时延好,但对端到端的时延没有严
  格的约束。
\end{enumerate}
\end{quote}

\begin{enumerate}
\def\labelenumi{\arabic{enumi}.}
\item
  security
\end{enumerate}

\begin{itemize}
\item
  encryption, data integrity, \ldots{}
\end{itemize}

\begin{quote}
\begin{enumerate}
\def\labelenumi{\arabic{enumi}.}
\item
  安全性 最 后 , 运 输 协 议 能 够 为 应 用 程 序 提 供 一 种 或 多 种
  安 全 性 服 务 , 例 如 , 在 发 送 主
  机中,运输协议能够加密由发送进程传输的所有数据,在接收主机中,运输层协议能够在将数据交
  付给接收进程之
  前解密这些数据。这种服务将在发送和接收进程之间提供机密性,以防该数据以某种方式在这两
  个进程之间被观察到 。运输协议还能提供除了机密性以外的其 他安全性
  服务,包括 数 据 完 整 性和端点鉴别,我们将 在第8章中详 细讨论这些
  主题。
\end{enumerate}
\end{quote}

2.14

Transport service requirements: common apps

\begin{longtable}[]{@{}llll@{}}
\toprule
\textbf{application} & \textbf{data loss} & \textbf{throughput} &
\textbf{time sensitive}\tabularnewline
\midrule
\endhead
& & &\tabularnewline
file transfer & no loss & elastic & no\tabularnewline
e-mail & no loss & elastic & no\tabularnewline
Web documents & no loss & elastic & no\tabularnewline
real-time audio/video & loss-tolerant & audio: 5kbps-1Mbps & yes, 100's
msec\tabularnewline
& & video:10kbps-5Mbps &\tabularnewline
stored audio/video & loss-tolerant & same as above & yes, few
secs\tabularnewline
interactive games & loss-tolerant & few kbps up & yes, 100's
msec\tabularnewline
text messaging & no loss & elastic & yes and no\tabularnewline
\bottomrule
\end{longtable}

\begin{quote}
至此,我们已经考虑了计算机网络能够一般性地提供的运输服务。现在我们要更为具
体地考察由因特网提供的运输服务类型。因特网(更一般的是T C P / I
P网络)为应用程序提供 两 个运输层协议,即U D P和T C P。当你(作 为 一
个软件开发者)为因特网创建一 个
新的应用时,首先要做出的决定是,选择UDP还是选择TCP。每个协议为调用它们的应用程序提供了不同的服务集合。图2-4显示了某些所选的应用程序的服务要求。
\end{quote}

2.15

Internet transport protocols services

\emph{TCP service:}

\begin{enumerate}
\def\labelenumi{\arabic{enumi}.}
\item
  \emph{reliable transport} between sending and receiving process
\item
  \emph{flow control:} sender won't overwhelm receiver
\item
  \emph{congestion control:} throttle sender when network overloaded
\item
  \emph{does not provide:} timing, minimum throughput guarantee,
  security
\item
  \emph{connection-oriented:} setup required between client and server
  processes
\end{enumerate}

\emph{UDP service:}

\begin{enumerate}
\def\labelenumi{\arabic{enumi}.}
\item
  \emph{unreliable data transfer} between sending and receiving process
\item
  \emph{does not provide:} reliability, flow control, congestion
  control, timing, throughput guarantee, security, orconnection setup,
\end{enumerate}

\begin{quote}
UDP是一种不提供不必要服务的轻量级运输协议,它仅提供最小服务。UDP是无连接的,因此在两个进程通信前没有握手过程。UDP协议提供一种不可靠数据传送服务,也
就是说,当进程将一个报文发送进UDP套接字时,UDP协议并不保证该报文将到达接收进程。不仅如此,到达接收进程的报文也可能是乱序到达的。
UDP没有包括拥塞控制机制,所以UDP的发送端可以用它选定的任何速率向其下层(网络层)注人数据。(然而,值得注意的是实际端到端吞吐量可能小于这种速率,这可能是因为中间链路的带宽受限或因为拥塞而造成的。)
\end{quote}

Q: why bother? Why is there a UDP?

2.16

Internet apps: application, transport protocols

\begin{longtable}[]{@{}lll@{}}
\toprule
\textbf{application} & \textbf{application layer protocol} &
\textbf{underlying transport protocol}\tabularnewline
\midrule
\endhead
& &\tabularnewline
e-mail & SMTP {[}RFC 2821{]} & TCP\tabularnewline
remote terminal access & Telnet {[}RFC 854{]} & TCP\tabularnewline
Web & HTTP {[}RFC 2616{]} & TCP\tabularnewline
file transfer & FTP {[}RFC 959{]} & TCP\tabularnewline
streaming multimedia & HTTP (e.g., YouTube), RTP {[}RFC 1889{]} & TCP or
UDP\tabularnewline
& &\tabularnewline
Internet telephony & SIP, RTP, proprietary (e.g., Skype) & TCP or
UDP\tabularnewline
\bottomrule
\end{longtable}

\begin{quote}
至此,我们已经考虑了计算机网络能够一般性地提供的运输服务。现在我们要更为具
体地考察由因特网提供的运输服务类型。因特网(更一般的是TCP
/IP网络)为应用程序提供两个运输层协议,即UDP和TCP。当你(作为一个软件开发者)为因特网创建一个
新的应用时,首先要做出的决定是,选择UDP还是选择TCP。每个协议为调用它们的应用程序提供了不同的服务集合。图2-4显示了某些所选的应用程序的服务要求。
\end{quote}

2.17

Securing TCP

TCP \& UDP

\begin{itemize}
\item
  no encryption
\item
  cleartext passwds sent into socket traverse Internet in cleartext
\end{itemize}

SSL

\begin{itemize}
\item
  provides encrypted TCP connection
\item
  data integrity
\item
  end-point authentication
\end{itemize}

SSL is at app layer

\begin{itemize}
\item
  Apps use SSL libraries, which ``talk'' to TCP
\end{itemize}

SSL socket API

\begin{itemize}
\item
  \begin{itemize}
  \item
    cleartext passwds sent into socket traverse Internet encrypted
  \item
    See Chapter 7
  \end{itemize}
\end{itemize}

\begin{quote}
无论TCP还是UDP都没有提供任何加密机制,这就是说发送进程传进其套接字的数据,与经网络传送到目的进程的数据相同。因此,举例来说如果某发送进程以明文方式(即没有加密)发送了一个口令进入它的套接字,该明文口令将经过发送方与接收方之间的所有链路传送,这就可能在任何中间链路被嗅探和发现,因为隐私和其他安全问题对许多应用而言已经成为至关重要的问题,所以因特网界已经研制了
TCP的加强版本,称为安全套接字层(Secure Sockets
Layer,SSL)。用SSL加强后的TCP不仅能够做传统的TCP所能做的一切,而且提供了关键的进程到进程的安全性服务,包括加密、数据完整性和端点鉴别。我们强调SSL不是与TCP和UD
P在相同层次上的第三种因特网运输协议,而是一种对TCP的加强,这种强化是在应用层上实现的。特别是,如果一个应用程序要使用SS
L的服务,它需要在该应用程序的客户端和服务器端包括SSL代码(利用现有的、高度优化
的库和类SS L有它 自己的套接 字API ,这类 似于 传统的TCP
套接字API,当一个应用使用SS L时,发送进程向SS
L套接字传递明文数据;在发送主机中的SSL则加密该数据并将加密的数据传递给TCP套接字。加密的数据经因特网传送到接收进程中的TCP套接字。该接收套接字将加密数据传递给SSL,由其进行解密。最后,SSL通过它的SSL套接字将明文数据传递给接收进程。我们将在第8章中更为详细地讨论SSL。
\end{quote}

\hypertarget{22-web-and-http-hypertext-transfer-protocol-}{%
\subsection{2.2 Web and HTTP( Hypertext Transfer Protocol
)}\label{22-web-and-http-hypertext-transfer-protocol-}}

2.19

Web and HTTP

\emph{First, a review\ldots{}}

\begin{itemize}
\item
  \emph{web page} consists of \emph{objects}
\item
  object can be HTML file, JPEG image, Java applet, audio file,\ldots{}
\item
  web page consists of \emph{base HTML-file} which includes
  \emph{several referenced objects}
\item
  each object is addressable by a \emph{URL,} e.g.,
\end{itemize}

\begin{figure}
\centering
\includegraphics{/Users/whichway/workspace/E_ESSEX/CE265/MD/CE265.assets/image-20210311105419270.png}
\caption{}
\end{figure}

2.20 HTTP{[}C/S model{]} overview

\begin{quote}
Http, 无状态的协议 持续链接和非持续链接的区别
\end{quote}

\hypertarget{http-hypertext-transfer-protocol}{%
\subsubsection{\texorpdfstring{\hl{⭐️HTTP: hypertext transfer
protocol}}{⭐️HTTP: hypertext transfer protocol}}\label{http-hypertext-transfer-protocol}}

\begin{itemize}
\item
  Web's application layer protocol
\item
  client/server model
\item
  \begin{itemize}
  \item
    \emph{client:} browser that requests, receives, (using HTTP
    protocol) and ``displays'' Web objects
  \item
    \emph{server:} Web server sends (using HTTP protocol) objects in
    response to requests
  \end{itemize}
\end{itemize}

\begin{figure}
\centering
\includegraphics{/Users/whichway/workspace/E_ESSEX/CE265/MD/CE265.assets/image-20210311105445603.png}
\caption{}
\end{figure}

2.21 HTTP overview (continued)

\emph{uses TCP:}

\begin{enumerate}
\def\labelenumi{\arabic{enumi}.}
\item
  client initiates TCP connection (creates socket) to server, port 80
\item
  server accepts TCP connection from client
\item
  HTTP messages (application-layer protocol messages) exchanged between
  browser (HTTP client) and Web server (HTTP server)
\item
  TCP connection closed
\end{enumerate}

\emph{HTTP is ``stateless''}

\begin{itemize}
\item
  server maintains no information about past client requests
\end{itemize}

\emph{aside}

protocols that maintain ``state'' are complex!

\begin{itemize}
\item
  past history (state) must be maintained
\item
  if server/client crashes, their views of ``state'' may be
  inconsistent, must be reconciled
\end{itemize}

2.22 HTTP connections

\emph{non-persistent HTTP}

\begin{itemize}
\item
  at most one object sent over TCP connection
\item
  \begin{itemize}
  \item
    connection then closed
  \end{itemize}
\item
  downloading multiple objects required multiple connections
\end{itemize}

\emph{persistent HTTP}

\begin{itemize}
\item
  multiple objects can be sent over single TCP connection between
  client, server
\end{itemize}

2.23 Non-persistent HTTP

\begin{quote}
非持续性的HTTP 链接 HTTP C 端 开始 对 80 端口的 TCP 链接 HTTP S
端开始等待 80 端口的 TCP 链接
\end{quote}

suppose user enters URL:
\url{www.someSchool.edu/someDepartment/home.index} (contains text,
references to 10 jpeg images)

1a. HTTP client initiates TCP connection to HTTP server (process) at
\url{www.someSchool.edu} on port 80

1b. HTTP server at host \url{www.someSchool.edu} waiting for TCP
connection at port 80. ``accepts'' connection, notifying client

\begin{enumerate}
\def\labelenumi{\arabic{enumi}.}
\item
  HTTP client sends HTTP \emph{request message} (containing URL) into
  TCP connection socket. Message indicates that client wants object
  someDepartment/home.index
\item
  HTTP server receives request message, forms \emph{response message}
  containing requested object, and sends message into its socket
\item
  HTTP server closes TCP connection.
\item
  HTTP client receives response message containing html file, displays
  html. Parsing html file, finds 10 referenced jpeg objects
\item
  Steps 1-5 repeated for each of 10 jpeg objects
\end{enumerate}

2.25

Non-persistent HTTP: response time

\hypertarget{rtt-round-trip-time-}{%
\subsubsection{\texorpdfstring{\hl{⭐️RTT( Round-Trip Time
)}}{⭐️RTT( Round-Trip Time )}}\label{rtt-round-trip-time-}}

(definition): time for a small packet to travel from client to server
and back

HTTP response time:

\begin{itemize}
\item
  one RTT to initiate TCP connection
\item
  one RTT for HTTP request and first few bytes of HTTP response to
  return
\item
  file transmission time
\item
  non-persistent HTTP response time = 2RTT+ file transmission time
\end{itemize}

\begin{figure}
\centering
\includegraphics{/Users/whichway/workspace/E_ESSEX/CE265/MD/CE265.assets/image-20210311105900039.png}
\caption{}
\end{figure}

2.26

Persistent HTTP

\emph{non-persistent HTTP issues:}

\begin{enumerate}
\def\labelenumi{\arabic{enumi}.}
\item
  requires 2 RTTs per object
\item
  OS overhead for \emph{each} TCP connection
\item
  browsers often open parallel TCP connections to fetch referenced
  objects
\end{enumerate}

\emph{persistent HTTP:}

\begin{enumerate}
\def\labelenumi{\arabic{enumi}.}
\item
  server leaves connection open after sending response
\item
  subsequent HTTP messages between same client/server sent over open
  connection
\item
  client sends requests as soon as it encounters a referenced object
\item
  as little as one RTT for all the referenced objects
\end{enumerate}

2.27

HTTP request message

\begin{itemize}
\item
  two types of HTTP messages: \emph{request}, \emph{response}
\item
  HTTP request message:
\item
  \begin{itemize}
  \item
    ASCII (human-readable format)
  \end{itemize}
\end{itemize}

\begin{figure}
\centering
\includegraphics{/Users/whichway/workspace/E_ESSEX/CE265/MD/CE265.assets/image-20210311110005934.png}
\caption{}
\end{figure}

2.28

HTTP request message: general format

\begin{figure}
\centering
\includegraphics{/Users/whichway/workspace/E_ESSEX/CE265/MD/CE265.assets/image-20210311110111435.png}
\caption{}
\end{figure}

2.29

Uploading form input

POST method:

\begin{itemize}
\item
  web page often includes form input
\item
  input is uploaded to server in entity body
\end{itemize}

URL method:

\begin{itemize}
\item
  uses GET method
\item
  input is uploaded in URL field of request line:
\end{itemize}

\begin{Shaded}
\begin{Highlighting}[]
\NormalTok{www.somesite.com/animalsearch?monkeys}\ErrorTok{\&}\NormalTok{banana}
\end{Highlighting}
\end{Shaded}

2.30

Method types

HTTP/1.0:

\begin{itemize}
\item
  GET
\item
  POST
\item
  HEAD
\item
  \begin{itemize}
  \item
    asks server to leave requested object out of response
  \end{itemize}
\end{itemize}

HTTP/1.1:

\begin{itemize}
\item
  GET, POST, HEAD
\item
  PUT
\item
  \begin{itemize}
  \item
    uploads file in entity body to path specified in URL field
  \end{itemize}
\item
  DELETE
\item
  \begin{itemize}
  \item
    deletes file specified in the URL field
  \end{itemize}
\end{itemize}

2.31

HTTP response message

\begin{figure}
\centering
\includegraphics{/Users/whichway/workspace/E_ESSEX/CE265/MD/CE265.assets/image-20210311110349765.png}
\caption{}
\end{figure}

2.32

HTTP response status codes

\begin{itemize}
\item
  status code appears in 1st line in server-to-client response message.
\item
  some sample codes:
\end{itemize}

\textbf{200 OK}

\begin{itemize}
\item
  \begin{itemize}
  \item
    request succeeded, requested object later in this msg
  \end{itemize}
\end{itemize}

\textbf{301 Moved Permanently}

\begin{itemize}
\item
  \begin{itemize}
  \item
    requested object moved, new location specified later in this msg
    (Location:)
  \end{itemize}
\end{itemize}

\textbf{400 Bad Request}

\begin{itemize}
\item
  \begin{itemize}
  \item
    request msg not understood by server
  \end{itemize}
\end{itemize}

\textbf{404 Not Found}

\begin{itemize}
\item
  \begin{itemize}
  \item
    requested document not found on this server
  \end{itemize}
\end{itemize}

\textbf{505 HTTP Version Not Supported}

2.33

Trying out HTTP (client side) for yourself

\begin{enumerate}
\def\labelenumi{\arabic{enumi}.}
\item
  Telnet to your favorite Web server:
\end{enumerate}

\textbf{telnet cis.poly.edu 80}

opens TCP connection to port 80

(default HTTP server port) at cis.poly.edu.

anything typed in sent

to port 80 at cis.poly.edu

\begin{enumerate}
\def\labelenumi{\arabic{enumi}.}
\item
  type in a GET HTTP request:
\end{enumerate}

\textbf{GET /\textasciitilde ross/ HTTP/1.1}

\textbf{Host: cis.poly.edu}

2.34 User-server state: cookies

many Web sites use cookies

\emph{four components:}

1) cookie header line of HTTP \emph{response} message

2) cookie header line in next HTTP \emph{request} message

3) cookie file kept on user's host, managed by user's browser

4) back-end database at Web site

example:

\begin{itemize}
\item
  Susan always access Internet from PC
\item
  visits specific e-commerce site for first time
\item
  when initial HTTP requests arrives at site, site creates:
\item
  \begin{itemize}
  \item
    unique ID
  \item
    entry in backend database for ID
  \end{itemize}
\end{itemize}

2.35

Cookies: keeping ``state'' (cont.)

\begin{figure}
\centering
\includegraphics{/Users/whichway/workspace/E_ESSEX/CE265/MD/CE265.assets/image-20210311141644663.png}
\caption{}
\end{figure}

2.36

\emph{what cookies can be used for:}

\begin{enumerate}
\def\labelenumi{\arabic{enumi}.}
\item
  authorization
\item
  shopping carts
\item
  recommendations
\item
  user session state (Web e-mail)
\end{enumerate}

\emph{how to keep ``state'':}

\begin{enumerate}
\def\labelenumi{\arabic{enumi}.}
\item
  protocol endpoints: maintain state at sender/receiver over multiple
  transactions
\item
  cookies: http messages carry state
\end{enumerate}

\emph{cookies and privacy:}

\begin{enumerate}
\def\labelenumi{\arabic{enumi}.}
\item
  cookies permit sites to learn a lot about you
\item
  you may supply name and e-mail to sites
\end{enumerate}

2.37

Web caches (proxy server)

\begin{quote}
web 缓存,
\end{quote}

\emph{goal:} satisfy client request without involving origin server

\begin{enumerate}
\def\labelenumi{\arabic{enumi}.}
\item
  user sets browser: Web accesses via cache
\item
  browser sends all HTTP requests to cache
\item
  \begin{itemize}
  \item
    object in cache: cache returns object
  \item
    else cache requests object from origin server, then returns object
    to client
  \end{itemize}
\end{enumerate}

\begin{figure}
\centering
\includegraphics{/Users/whichway/workspace/E_ESSEX/CE265/MD/CE265.assets/image-20210315154516806.png}
\caption{}
\end{figure}

2.38

More about Web caching

\begin{enumerate}
\def\labelenumi{\arabic{enumi}.}
\item
  cache acts as both client and server
\item
  \begin{itemize}
  \item
    server for original requesting client
  \item
    client to origin server
  \end{itemize}
\item
  typically cache is installed by ISP (university, company, residential
  ISP)
\end{enumerate}

\emph{why Web caching?}

\begin{enumerate}
\def\labelenumi{\arabic{enumi}.}
\item
  reduce response time for client request
\item
  reduce traffic on an institution's access link
\item
  Internet dense with caches: enables ``poor'' content providers to
  effectively deliver content (so too does P2P file sharing)
\end{enumerate}

2.39

Caching example:

\emph{assumptions:}

\begin{enumerate}
\def\labelenumi{\arabic{enumi}.}
\item
  avg object size: 100K bits
\item
  avg request rate from browsers to origin servers:15/sec
\item
  avg data rate to browsers: 1.50 Mbps
\item
  RTT from institutional router to any origin server: 2 sec
\item
  access link rate: 1.54 Mbps
\end{enumerate}

\emph{consequences:}

\begin{enumerate}
\def\labelenumi{\arabic{enumi}.}
\item
  LAN utilization: 15\%
\item
  access link utilization = 99\%
\item
  total delay = Internet delay
  {[}\includegraphics{/Users/whichway/workspace/E_ESSEX/CE265/MD/CE265.assets/image-20210316103014641.png}{]}
  access delay + LAN delay = 2 sec + minutes + usecs
\end{enumerate}

\begin{figure}
\centering
\includegraphics{/Users/whichway/workspace/E_ESSEX/CE265/MD/CE265.assets/image-20210315154842559.png}
\caption{}
\end{figure}

2.40

Caching example: fatter access link

\emph{assumptions:}

\begin{itemize}
\item
  avg object size: 100K bits
\item
  avg request rate from browsers to origin servers:15/sec
\item
  avg data rate to browsers: 1.50 Mbps
\item
  RTT from institutional router to any origin server: 2 sec
\item
  access link rate: 1.54
  Mbps\includegraphics{/Users/whichway/workspace/E_ESSEX/CE265/MD/CE265.assets/image-20210316103032810.png}
\end{itemize}

\emph{consequences:}

\begin{itemize}
\item
  LAN utilization: 15\%
\item
  access link utilization =
  99\%\includegraphics{/Users/whichway/workspace/E_ESSEX/CE265/MD/CE265.assets/image-20210316103042875.png}
\item
  total delay = Internet delay + access delay + LAN delay

  = 2 sec + minutes +
  usecs\includegraphics{/Users/whichway/workspace/E_ESSEX/CE265/MD/CE265.assets/image-20210316103055232.png}
\end{itemize}

Cost: increased access link speed (not cheap!)

\begin{figure}
\centering
\includegraphics{/Users/whichway/workspace/E_ESSEX/CE265/MD/CE265.assets/image-20210316102411065.png}
\caption{}
\end{figure}

2.41

\emph{assumptions:}

\begin{itemize}
\item
  avg object size: 100K bits
\item
  avg request rate from browsers to origin servers:15/sec
\item
  avg data rate to browsers: 1.50 Mbps
\item
  RTT from institutional router to any origin server: 2 sec
\item
  access link rate: 1.54 Mbps
\end{itemize}

\emph{consequences:}

\begin{itemize}
\item
  LAN utilization: 15\%
\item
  access link utilization = 100\%
\item
  total delay = Internet delay + access delay + LAN delay

  = 2 sec + minutes + usecs
\end{itemize}

\emph{How to compute link} utilization, delay?

Cost: web cache (cheap!)

\begin{figure}
\centering
\includegraphics{/Users/whichway/workspace/E_ESSEX/CE265/MD/CE265.assets/image-20210316103157522.png}
\caption{}
\end{figure}

2.42

Caching example: install local cache

\emph{Calculating access link utilization, delay with cache:}

\begin{itemize}
\item
  suppose cache hit rate is 0.4
\item
  \begin{itemize}
  \item
    40\% requests satisfied at cache, 60\% requests satisfied at origin
  \end{itemize}
\end{itemize}

\begin{itemize}
\item
  access link utilization:
\item
  \begin{itemize}
  \item
    60\% of requests use access link
  \end{itemize}
\item
  data rate to browsers over access link = 0.6*1.50 Mbps = .9 Mbps
\item
  \begin{itemize}
  \item
    utilization = 0.9/1.54 = .58
  \end{itemize}
\end{itemize}

\begin{itemize}
\item
  total delay
\item
  \begin{itemize}
  \item
    = 0.6 * (delay from origin servers) +0.4 * (delay when satisfied at
    cache)
  \item
    = 0.6 (2.01) + 0.4 (\textasciitilde msecs)
  \item
    = \textasciitilde{} 1.2 secs
  \item
    less than with 154 Mbps link (and cheaper too!)
  \end{itemize}
\end{itemize}

\includegraphics{/Users/whichway/workspace/E_ESSEX/CE265/MD/CE265.assets/image-20210316102901433.png}

2.43

Conditional GET

\begin{itemize}
\item
  \emph{Goal:} don't send object if cache has up-to-date cached version
\item
  \begin{itemize}
  \item
    no object transmission delay
  \item
    lower link utilization
  \end{itemize}
\item
  \emph{cache:} specify date of cached copy in HTTP request
\end{itemize}

\begin{Shaded}
\begin{Highlighting}[]
\NormalTok{If{-}modified{-}since: }\KeywordTok{\textless{}date\textgreater{}}
\end{Highlighting}
\end{Shaded}

\begin{itemize}
\item
  \emph{server:} response contains no object if cached copy is
  up-to-date:
\end{itemize}

\begin{Shaded}
\begin{Highlighting}[]
\NormalTok{HTTP/1.0 304 Not Modified}
\end{Highlighting}
\end{Shaded}

\begin{figure}
\centering
\includegraphics{/Users/whichway/workspace/E_ESSEX/CE265/MD/CE265.assets/image-20210316083302013.png}
\caption{}
\end{figure}

\hypertarget{23-ftp-fire-transfer-protocol-}{%
\subsection{2.3 FTP( Fire Transfer Protocol
)}\label{23-ftp-fire-transfer-protocol-}}

2.45 FTP: the file transfer protocol

\begin{quote}
FTP outband 带外传输, 2条链接 HTTP inband 带内传输
\end{quote}

\begin{figure}
\centering
\includegraphics{/Users/whichway/workspace/E_ESSEX/CE265/MD/CE265.assets/image-20210316083951508.png}
\caption{}
\end{figure}

\begin{enumerate}
\def\labelenumi{\arabic{enumi}.}
\item
  \hl{transfer file to/from remote host}
\item
  \hl{client/server model {[}CS 模式{]}}
\item
  \begin{itemize}
  \item
    \emph{client:} side that initiates transfer (either to/from remote)
  \item
    \emph{server:} remote host
  \end{itemize}
\item
  ftp: RFC 959
\item
  ftp server: port \hl{21}
\end{enumerate}

\begin{quote}
在一个典型的FTP会话中,用户坐在一台主机(本地主机)前面,向一台远程主机传输(或接收来自远程主机的)文件。为使用户能访问它的远程账户,用户必须提供一个用户标识和口令。在提供了这种授权信息后,用户就能从本地文件系统向远程主机文件系统传送文件,反之
亦然=如图2- 14所示,用户通过一个FT P用户代理与FTP交互
。该用户首先提供远程主机的主机名,使本地主机的FTP客户进程建立一个到远程主机FTP服务器进程的TCP连接。该用户接着提供用户标识和口令,作为FT
P命令的一部分在该TCP连接上传送。一旦该服务器向该用户授权,用户可以将存放在本地文件系统中的一个或者多个文件复制到远程文件系统(反之亦然)。
\end{quote}

2.46

FTP{[} File Transfer Protocol {]}: separate control, data connections

\begin{enumerate}
\def\labelenumi{\arabic{enumi}.}
\item
  FTP client contacts FTP server at port 21, using TCP
\item
  client authorized over control connection
\item
  client browses remote directory, sends commands over control
  connection
\item
  when server receives file transfer command, \emph{server} opens
  \emph{2**nd} TCP data connection (for file) \emph{to} client
\item
  after transferring one file, server closes data connection
\item
  server opens another TCP data connection to transfer another file
\item
  control connection: \emph{``out of band''}
\item
  FTP server maintains ``state'': current directory, earlier
  authentication
\end{enumerate}

\begin{figure}
\centering
\includegraphics{/Users/whichway/workspace/E_ESSEX/CE265/MD/CE265.assets/image-20210316084050007.png}
\caption{}
\end{figure}

\begin{quote}
HTTP和FTP都是文件传输协议,并且有很多共同的特点,例如,它们都运行在TCP
上、然而,这两个应用层协议也有一些重要的区别。其中最显著的就是FTP使用了两个并行的TC
P连接来传输文件,一个是控制连接(control
connection),一个是数据连接(data connection)
控制连接用于在两主机之间传输控制信息,如用户标识、口令、改变远程目
录的命令 以及 ``存放 (put)"和`` 获取(get)''文件的命令 。数据
连接用于实际发送一个 文件。 因为FT P协议使用一个 独立
的控制连接,所以我们也称FT P的控制信息是带外(ou t-o f-b and
)传送的。如你所知,HTT
P协议是在传输文件的同一个TCP连接中发送请求和响应首部行的。因此,HTTP也可以说是带内(in-band)发送控制信息的、FTP协议的控制连接和数据连接如图2-15所示。

当用户主机与远程主机开始一个FTP会 话时,FTP的客户(用户)端首先在服务器
21号端口与服务器(远程主机)端发起一个
用于控制的TCP连接。FTP的客户端也通过该控制连接发送用户的标识和口令,发送改变远程目录的命令。当FTP的服务器端从该连接上收到一个文件传输的命令后(无论是向
还是来自远程主机),就发起一个到客户端的TCP数据连接。FTP在该数据连接上准确地传送一个文件,然后关闭该连接。在同一个会话期间,如果用户还需要传输另一个文件,
FTP则打开另一个数据连接。因而对FTP传输而言,控制连接贯穿了整个用户会话期间,但是对会话中的每一次文件传输都需要建立一个新的数据连接(即数据连接是非持续的)。

FTP服务器必须在整个会话期间保留用户的状态(state)。特别是,服务器必须把特定的用户账户与控制连接联系起来,随着用户在远程目录树上徘徊,服务器必须追踪用户在远程目录树上的当前位置,对每个进行中的用户会话的状态信息进行追踪,大大限制了FTP同时维持的会话总数。而另一方面,前面讲过HTTP是无状态的,即它不必对任何用户状态进行追踪。
\end{quote}

2.47

FTP commands, responses

\emph{sample commands:}

\begin{itemize}
\item
  sent as ASCII text over control channel
\item
  \textbf{USER *username*}
\item
  \textbf{PASS *password*}
\item
  \textbf{LIST} return list of file in current directory
\item
  \textbf{RETR filename} retrieves (gets) file
\item
  \textbf{STOR filename} stores (puts) file onto remote host
\end{itemize}

\begin{quote}
我们通过简要地讨论几个常见的FTP命令和回答来结束本节。从客户到服务器的命令和从服务器到
客户 的回答, 都是以7比特ASC II格式 在控 制连接上传 送的 。因此 ,与
HTTP协议的命令类似,FTP协议的命令也是人可读的。为了区分连续的命令,每个命令后跟回车换行符。每个命令由4个大写字母ASCII字符组成,有些还具有可选参数。一些较为常见的命令如下:
• USER username:用于向服务器传送用户标识。 • PASS
password:用于向服务器发送用户口令、 •
UST:用于请求服务器回送当前远程目录中的所有文件列表。该文件表是经一
个(新建且非持续连接)数据连接传送的,而不是在控制TCP连接上传送。 • RETR
filename;用于从远程主机当前目录检索(即gel)文件。该命令引起远程主机发起一个数据连接,并经该数据连接发送所请求的文件。
• STOR filename:用于在远程主机的当前目录上存放(即put)文件。
\end{quote}

\emph{sample return codes}

\begin{itemize}
\item
  status code and phrase (as in HTTP)
\item
  \textbf{331 Username OK, password required}
\item
  \textbf{125 data connection already open; transfer starting}
\item
  \textbf{425 Can't open data connection}
\item
  \textbf{452 Error writing file}
\end{itemize}

\begin{quote}
贯穿控制连接,在用户发出的命令和FTP发送的命令之间通常有一一对应关系。每个命令都对应着一个从服务器发向客户的回答。回答是一个3位的数字,后跟一个可选信
息。这与HTTP响应报文状态行的状态码和状态信息的结构相同。一些典型的回答连同它们可能的报文如下所示:
• 331 Username OK, Password required (用户名 OK,需要口令)。 • 125
Data connection already open; transfer starting
(数据连接已经打开,开始传送) • 425 Can't open data connection
(无法打开数据连接) • 452 Error writing file (写文件差错)。
有兴趣学习其他FTP命令和回答的读者请阅读RFC 959。
\end{quote}

\hypertarget{24-electronic-mail--smtp-pop3-imap}{%
\subsection{2.4 electronic mail : SMTP, POP3,
IMAP}\label{24-electronic-mail--smtp-pop3-imap}}

\begin{quote}
留下的3个邮件访问协议 POP3, IMAP, HTTP
\end{quote}

\begin{quote}
自从有了因特网,电子邮件就在因特网上流行起来。当因特网还在襁褓中时,电子邮件已经成为最为流行的应用程序[Segaller
l998],年复一年,它变得越来越精细,越来越强大。它是当今因特网上最重要和实用的应用程序之一。
与普通邮件一样,电子邮件是一种异步通信媒介,即当人们方便时就可以收发邮件,不必与他人的计划进行协调。与普通邮件相比,电子邮件更为快速并且易于分发,而且价
格便宜。现代电子邮件具有许多强大的特性,包括具有附件、超链接、HTML格式文本和图片的报文。

Web电子邮件 1995年12月,在Web ``发 明''后 仅过了几年,Sabeer B hatia和Ja
ck S mith拜访了因特网风险投资人Dr ap er Fisher
Jurvetson,提出了研发一个免费的基于Web的电子邮件
系统的建议:其基本想法是为任何想要的人分配一个免费的电子邮件账户,并且使得这个账户可以在Web上使用。通过用公司的15\%份额作为交换,Dra
per Fi she r J urv ets on向 Bh at ia和Sm it
h提供了资金,后者组建了一家公司,叫做Hotmail。3个全职员工和14个
兼职人员为了自己拥有的股份而工作,1996年7月,他们研发并提供了该服务,:之后的一个月内,他们就拥有了
100 000名用户。199
7年12月,在启动该服务不到18个月内,Hotmail就拥有超过1200万个用户,并且以4亿美元的价格被微软公司收购,Hotmail公司的成功常被归结于它的``先行者优势(first-mover
advantage)"和它固有的电子邮件``病毒行销(viral marketing)
策略,:(也许正在阅读本书的某些学生将会成为这种新人
之-\/-\/-\/-\/-\/-\/-\/-构思并开发具有``先行者优势''和``病毒行销''策略特征的因特网服务,)Web电子邮件继续兴盛,每年都变得更为复杂和功能强大。当今最为流行的服务之
一是谷歌的Gmail,它提供了千兆字节的免费存储、先进的垃圾邮件过滤和病毒检测、
电子邮件加密(使用SSL)、对第三方电子邮件服务的邮件接纳和面向搜索的界面。社
交网络如脸谱中的异步信息转发在近年来也已经变得流行。
\end{quote}

2.49

Electronic mail

\emph{Three major components:}

\begin{enumerate}
\def\labelenumi{\arabic{enumi}.}
\item
  user agents
\item
  mail servers
\item
  simple mail transfer protocol: SMTP
\end{enumerate}

\emph{User Agent}

\begin{enumerate}
\def\labelenumi{\arabic{enumi}.}
\item
  a.k.a. ``mail reader''
\item
  composing, editing, reading mail messages
\item
  e.g., Outlook, Thunderbird, iPhone mail client
\item
  outgoing, incoming messages stored on server
\end{enumerate}

\begin{figure}
\centering
\includegraphics{/Users/whichway/workspace/E_ESSEX/CE265/MD/CE265.assets/image-20210316084931696.png}
\caption{}
\end{figure}

\begin{quote}
图2- 16给出了因特网电子邮件系统的总体情况。从该图中我们可以看到它有3个主
要组 成部分:用 户代理(use ragent)、邮件服务器 (mailse rv er
)和简单邮件传输协议(Si mp le M ail Tr an sf er P ro to co l, S MT P)
o下面我们结合发送方Al ic e发电子邮件给接收方
Bob的例子,对每个组成部分进行描述。用户代理允许用户阅读、回复、转发、保存和撰写报文
微软的Ou tl oo k和Ap pl e Ma il是电子邮件用户代理的例子。当Al ic
e完成邮件撰写 时.她的邮 件代 理向其 邮件服务器 发送邮
件,此时邮件放在邮件服务器的外出报文队列中。邮件服务器形成了电子邮件体系结构的核心。每个接收方(如Bob)在其中的某个邮件服务器上有一个邮箱(mailbox)。Bob的邮箱管理和维护着发送给他的报文。一个典型的邮件发送过程是:从发送方的用户代理开始,传输到发送方的邮件服务器,再传输到接收方的邮件服务器,然后在这里被分发到接收方的邮箱中。当Boh要在他的邮箱中读取该报文时,包含他邮箱的邮件服务器(使用用户名和口令)来鉴别Bob。Alice的邮箱也必须能处理Bob
的邮件服务器的故障 如果Alice的服务器不能将邮件交付给Bob的服务器,Al ic
e的邮件服务器在一个报文队列(message
queue)中保持该报文并在以后尝试再次发
送。通常每30分钟左右进行一次尝试;如果几天后仍不能成功,眼务器就删除该报文并以电子邮件的形式通知发送方(Alice),
\end{quote}

2.50

Electronic mail: mail servers

mail servers:

\begin{itemize}
\item
  \emph{mailbox} contains incoming messages for user
\item
  \emph{message queue} of outgoing (to be sent) mail messages
\item
  \emph{SMTP protocol} between mail servers to send email messages
\item
  \begin{itemize}
  \item
    client: sending mail server
  \item
    ``server'': receiving mail server
  \end{itemize}
\end{itemize}

\begin{figure}
\centering
\includegraphics{/Users/whichway/workspace/E_ESSEX/CE265/MD/CE265.assets/image-20210316090302112.png}
\caption{}
\end{figure}

\begin{quote}
SMT P是因特网电子邮件中主要的应用层协议。它使用TCP可靠数据传输服务,从发
送方的邮件服务器向接收方的邮件服务器发送邮件。像大多数应用层协议一样,SMTP也
有两个部分:运行在发送方邮件服务器的客户端和运行在接收方邮件服务器的服务器端。

每台邮件服务器上既运行SMT P的客户端也运行SMT
P的服务器端。当一个邮件服务器向
其他邮件服务器发送邮件时,它就表现为SMTP的客户;当邮件服务器从其他邮件服务器上接收邮件时,它就表现为一个SMTP的服务器
\end{quote}

2.51

Electronic Mail: SMTP {[}RFC 2821{]}

\begin{enumerate}
\def\labelenumi{\arabic{enumi}.}
\item
  uses TCP to reliably transfer email message from client to server,
  port 25
\item
  direct transfer: sending server to receiving server
\item
  three phases of transfer
\item
  \begin{itemize}
  \item
    handshaking (greeting)
  \item
    transfer of messages
  \item
    closures
  \end{itemize}
\item
  command/response interaction (like HTTP, FTP)
\item
  \begin{itemize}
  \item
    commands: ASCII text
  \item
    response: status code and phrase
  \end{itemize}
\item
  messages must be in 7-bit ASCI
\end{enumerate}

\begin{quote}
RF C 53 21给出了 SM TP的定义。SM
TP是因特网电子邮件应用的核心,如前所述,SMTP用于从发送方的邮件服务器发送报文到接收方的邮件服务器。SMT
P问世的时间比HT TP要长得多(初始的SM TP协议的RF C可追溯到19 82年,而SM
TP在此之前很长一段时间就已经出现了)。尽管电子邮件应用在因特网上的独特地位可以证明SMTP有着众多非常出色的性质,但它所具有的某种陈旧特征表明它仍然是一种继承的技术。例如,它限制所有邮件报文的体部分(不只是其首部)只能采用简单的7比特AS
CI I表示。在20世
纪80年代早期,这种限制是明智的,因为当时传输能力不足,没有人会通过电子邮件发
送大的附件或是大的图片、声音或者视频文件。然而,在今天的多媒体时代,7位ASC
II 的限 制的确 有点痛苦,即在用SM TP传送邮件之
前,需要将二进制多媒体数据编码为 AS CI I码,并且在使用SM
TP传输后要求将相应的ASCII码邮件解码还原为多媒体数据。
2.2节讲过,使用HTTP传送前不需要将多媒体数据编码为ASCII码。
\end{quote}

2.52

Scenario: Alice sends message to Bob

1) Alice uses UA to compose message ``to''
\href{mailto:bob@someschool.edu}{\nolinkurl{bob@someschool.edu}}

\begin{quote}
@ is the sign that never exists in person's name
\end{quote}

2) Alice's UA sends message to her mail server; message placed in
message queue

3) client side of SMTP opens TCP connection with Bob's mail server

4) SMTP client sends Alice's message over the TCP connection

5) Bob's mail server places the message in Bob's mailbox

6) Bob invokes his user agent to read message

\begin{figure}
\centering
\includegraphics{/Users/whichway/workspace/E_ESSEX/CE265/MD/CE265.assets/image-20210316090929202.png}
\caption{}
\end{figure}

\begin{quote}
为了 描述SM TP的基本操作.我们观察一种常见的情景。假设Al ic e想给Bo
b发送一 封简单的ASCII报文- •
Alice调用她的邮件代理程序并提供Bob的邮件地址(例如bob@ som eschool .
edu), 撰写报文.然后指示用户代理发送该报文。 •
Alire的用户代理把报文发给她的邮件服务器,在那里该报文被放在报文队列中,
•运行在Al ic e的邮件服务器 上的SM
TP客户端发现了报文队列中的这个报文,它就
创建一个到运行在Bob的邮件服务器上的SMTP服务器的TCP连接。
•在经过一些初始SMTP握手后,SMTP客户通过该TCP连接发送Alice的报文。
・在Boh的邮件服务器上,SMTP的服务器端接收该报文。Bob的邮件服务器然后将
该报文放人Bob的邮箱中- ・在Boh方便的时候,他调用用户代理阅读该报文。
图2-17总结了上述这个情况。 观察到下述现象是重要的:SMTP
一般不使用中间邮件服务器发送邮件,即使这两个
邮件服务器位于地球的两端也是这样。假设Alice的邮件服务器在中国香港,而Bob的服
务器在美国圣路易斯,那么这个TCP连接也是从香港服务器到圣路易斯服务器之间的直接
相连。特别是,如果Bob的邮件服务器没有开机,该报文会保留在Ali
ce的邮件服务器上
并等待进行新的尝试,这意味着邮件并不在中间的某个邮件服务器存留。
\end{quote}

2.53

Sample SMTP interaction

\begin{Shaded}
\begin{Highlighting}[]
\NormalTok{     S: 220 hamburger.edu }
\NormalTok{     C: HELO crepes.fr }
\NormalTok{     S: 250  Hello crepes.fr, pleased to meet you }
\NormalTok{     C: MAIL FROM: }\KeywordTok{\textless{}alice}\ErrorTok{@}\OtherTok{crepes.fr}\ErrorTok{\textgreater{}} 
     \ErrorTok{S:} \ErrorTok{250} \ErrorTok{alice@crepes.fr...} \ErrorTok{Sender} \ErrorTok{ok} 
     \ErrorTok{C:} \ErrorTok{RCPT} \ErrorTok{TO:} \ErrorTok{\textless{}bob@hamburger.edu\textgreater{}} 
     \ErrorTok{S:} \ErrorTok{250} \ErrorTok{bob@hamburger.edu} \ErrorTok{...} \ErrorTok{Recipient} \ErrorTok{ok} 
     \ErrorTok{C:} \ErrorTok{DATA} 
     \ErrorTok{S:} \ErrorTok{354} \ErrorTok{Enter} \ErrorTok{mail,} \ErrorTok{end} \ErrorTok{with} \ErrorTok{"."} \ErrorTok{on} \ErrorTok{a} \ErrorTok{line} \ErrorTok{by} \ErrorTok{itself} 
     \ErrorTok{C:} \ErrorTok{Do} \ErrorTok{you} \ErrorTok{like} \ErrorTok{ketchup?} 
     \ErrorTok{C:} \ErrorTok{How} \ErrorTok{about} \ErrorTok{pickles?} 
     \ErrorTok{C:} \ErrorTok{.} 
     \ErrorTok{S:} \ErrorTok{250} \ErrorTok{Message} \ErrorTok{accepted} \ErrorTok{for} \ErrorTok{delivery} 
     \ErrorTok{C:} \ErrorTok{QUIT} 
     \ErrorTok{S:} \ErrorTok{221} \ErrorTok{hamburger.edu} \ErrorTok{closing} \ErrorTok{connection}
\end{Highlighting}
\end{Shaded}

\begin{quote}
我们现在仔细观察一下,SMTP是如何将一个报文从发送邮件服务器传送到接收邮件
服务器的。我们将看到,SMTP与人类面对面交往的行为方式有许多类似性、首先,客户
SMT P (运行在发送邮件服务器上)在25号端口建立一个到服务器SMTP
(运行在接收邮
件服务器上)的TCP连接。如果服务器没有开机,客户会在稍后继续尝试连接。一旦连
接建立,服务器和客户执行某些应用层的握手,就像人们在互相交流前先进行自我介绍一
样,SMT
P的客户和服务器在传输信息前先相互介绍。在SMTP握手的阶段,SMTP客户指
示发送方的邮件地址(产生报文的那个人)和接收方的邮件地址。一旦该SMTP客户和服
务器彼此介绍之后,客户发送该报文。SMTP能依赖TCP提供的可靠数据传输无差错地将
邮件投递到接收服务器。该客户如果有另外的报文要发送到该服务器,就在该相同的TCP
连接上重复这种处理;否则,它指示TCP关闭连接。
接下来我们分析一个在SMTP客户(C)和SMTP服务器(S)之间交换报文脚本的例
子。客户 的主机名为 crepes, fr,服务器 的主机名为 ha mbur ge r. e du
o以C:开头的AS CI I码
文本行正是客户交给其TCP套接字的那些行,以S:开头的ASCII码则是服务器发送给其
TCP套接字的那些行。一旦创建了 TCP连接,就开始了下列过程

在上例中,客户从邮件服务器crepes, fr向邮件服务器hamburger, ed
u发送了一个报文 (**D o y ou lik e k etc hup ? H ow abo ut pic kle s?"
) o 作为对话的一部分,该客户发送了 5 条命 令:HELO (是 HELLO
的缩写)、MAIL FROM , RCPTTO、DATA 以及 QUIT。这些命令都
是自解释的。该客户通过发送一个只包含一个句点的行,向服务器指示该报文结束了。
220 hamburger.edu HELO crepes.fr 250 Hello crepes.fr, pleased to meet
you MAIL FROM: \textless alicegcrepes.fr\textgreater{} 250
alice@crepes.fr ... Sender ok RCPT TO:
\textless bob0hamburger.edu\textgreater{} 250 bob®hamburger.edu ...
Recipient ok DATA 354 Enter mail, end with "・"on a line by itself Do
you like ketchup? How about pickles?
(按照ASCII码的表示方法,每个报文以CRLF.
CRLF结束,其中的CR和LF分别表示回
车和换行。)服务器对每条命令做出回答,其中每个回答含有一个回答码和一些(可选的)英文解释,我们在这里指出SMTP用的是持续连接:如果发送邮件服务器有几个报文
发往同一个接收邮件服务器,它可以通过同一个TCP连接发送这些所有的报文。对每个报
文,该客户用一个新的MAI L F ROM: cr epe s,
fr开始,用一个独立的句点指示该邮件的结
束,并且仅当所有邮件发送完后才发送QUIT.
\end{quote}

2.54

Try SMTP interaction for yourself:

\begin{enumerate}
\def\labelenumi{\arabic{enumi}.}
\item
  '''shell telnet servername 25 '''
\item
  see 220 reply from server
\item
  enter HELO, MAIL FROM, RCPT TO, DATA, QUIT commands
\end{enumerate}

above lets you send email without using email client (reader)

\begin{quote}
我们强烈推荐你使用Telnet与一个SMTP服务器进行一次直接对话。使用的命令是

'''shell telnet serverName 25 '''

其中serverName是本地邮件服务器的名称。当你这么做时,就直接在本地主机与邮件服务
器之间建立一个TC P连接。输完上述命令后,你立即会从该服务器收到22
0回答。接下来,在适当的时机发出 HEL O、MAI L F ROM, RCP T T O、DAT
A、CRL F. CRL F 以及 QUIT等SM TP命令。
\end{quote}

2.55

SMTP: final words

\begin{enumerate}
\def\labelenumi{\arabic{enumi}.}
\item
  SMTP uses persistent connections
\item
  SMTP requires message (header \& body) to be in 7-bit ASCII
\item
  SMTP server uses CRLF.CRLF to determine end of message
\end{enumerate}

\emph{comparison with HTTP:}

\begin{enumerate}
\def\labelenumi{\arabic{enumi}.}
\item
  HTTP: pull
\item
  SMTP: push
\item
  both have ASCII command/response interaction, status codes
\item
  HTTP: each object encapsulated in its own response msg
\item
  SMTP: multiple objects sent in multipart msg
\end{enumerate}

\begin{quote}
我们简要地比较一下SM TP和HT TP .这两个 协议都用于从 一台
主机向另一台主机传 送文件:HTTP从Web服务器向Web客户(通常是一个
浏览器)传送文件(也称为对象);SMTP从一个邮件服务器向另一个邮件服务器传送文件(即电子邮件报文)。当进行文件传送时,持续的HTTP和SMTP都使用持续连接。因此,这两个协议有一些共同特征。
然而,两者之间也有一些重要的区别。首先,HT TP主要是一个拉协议(pull
protocol),
即在方便的时候,某些人在Web服务器上装载信息,用户使用HTTP从该服务器拉取这些信息。特别是TCP连接是由想接收文件的机器发起的。另一方面,SMTP基本上是一个推协议(push
protocol),即发送邮件服务器把文件推向接收邮件服务器。特别是,这个TCP
连接是由要发送该文件的机器发起的。

第二个区别就是我们前面间接地提到过的,SMTP要求每个报文(包括它们的体)使
用7比特ASCII码格式 。如果 某报 文包含 了非7比特ASC II字符( 如具有重音
的法文字符)或二进制数据(如图形文件),则该报文必须按照7比特ASCII码进行编码、HTTP数据则不受这种限制。

第三个重要区别是如何处理一个既包含文本又包含图形(也可能是其他媒体类型)的文档。如我们在2.
2节知道的那样,HTTP把每个对象封装到它自己的HTTP响应报文中,而SMTP则把所有报文对象放在一个报文之中:
\end{quote}

2.56

Mail message format

SMTP: protocol for exchanging email msgs

RFC 822: standard for text message format:

\begin{itemize}
\item
  header lines, e.g.,
\item
  \begin{itemize}
  \item
    To:
  \item
    From:
  \item
    Subject:
  \end{itemize}
\end{itemize}

\emph{different} \emph{from} SMTP MAIL FROM, RCPT TO: commands!

\begin{itemize}
\item
  Body: the ``message''
\item
  \begin{itemize}
  \item
    ASCII characters only
  \end{itemize}
\end{itemize}

\begin{figure}
\centering
\includegraphics{/Users/whichway/workspace/E_ESSEX/CE265/MD/CE265.assets/image-20210316092842579.png}
\caption{}
\end{figure}

2.57

Mail access protocols

\begin{figure}
\centering
\includegraphics{/Users/whichway/workspace/E_ESSEX/CE265/MD/CE265.assets/image-20210316093559563.png}
\caption{}
\end{figure}

\begin{itemize}
\item
  SMTP: delivery/storage to receiver's server
\item
  mail access protocol: retrieval from server
\item
  \begin{itemize}
  \item
    POP: Post Office Protocol {[}RFC 1939{]}: authorization, download
  \item
    IMAP: Internet Mail Access Protocol {[}RFC 1730{]}: more features,
    including manipulation of stored msgs on server
  \item
    HTTP: gmail, Hotmail, Yahoo! Mail, etc.
  \end{itemize}
\end{itemize}

\begin{quote}
当Alice给Bob写一封邮寄时间很长的普通信件时,她可能要在信的上部包含各种各
样的环境首部信息,如Bob的地址、她自己的回复地址以及日期等。同样,当一个人给另一个人发送电子邮件时,一个包含环境信息的首部位于报文体前面。这些环境信息包括在一系列首部行中,这些行由RFC
532 2定义。首部行和该报文的体用空行(即回车换行)进行分隔。RFC
5322定义了邮件首部行和它们的语义解释的精确格式。如同HTTP协议,每个首部行包含了可读的文本,是由关键词后跟冒号及其值组成的。某些关键词是必需的,另一
些则是可选的。每个首部必须含有一个Fr om :首部行和一 个To :首部行;一个
首部也许包含一个Subject:首部行以及其他可选的首部行。注意到下列事实是重要的:这些首部行不同
于我们在2. 4 . 1节所学到的SM
TP命令(即使那里也包含了某些相同的词汇,如fr om和to ).
.那节中的命令是SM
TP握手协议的一部分;本节中研究的首部行则是邮件报文自身的一部分。
一个典型的报文首部看起来如下:

'''html From: alice@crepes.fr To:
\href{mailto:bob@hamburger.edu}{\nolinkurl{bob@hamburger.edu}} Subject:
Searching for the meaning of life. '''

在报文首部之后,紧接着一个空白行,然后是以ACSII格式表示的报文体你应当用
Te ln
et向邮件服务器发送包含一些首部行的报文,包括Subject:首部行„为此,输人命令
telnet serverName 25 ,如在2. 4. 1节中讨论的那样.
\end{quote}

2.58

POP3 protocol

\emph{authorization phase}

\begin{itemize}
\item
  client commands:
\item
  \begin{itemize}
  \item
    \textbf{user:} declare username
  \item
    \textbf{pass:} password
  \end{itemize}
\item
  server responses
\item
  \begin{itemize}
  \item
    \textbf{+OK}
  \item
    \textbf{-ERR}
  \end{itemize}
\end{itemize}

'''html S: +OK POP3 server ready C: user bob S: +OK C: pass hungry S:
+OK user successfully logged on '''

\emph{transaction phase,} client:

\begin{itemize}
\item
  \textbf{list:} list message numbers
\item
  \textbf{retr:} retrieve message by number
\item
  \textbf{dele:} delete
\item
  \textbf{quit}
\end{itemize}

'''html C: list S: 1 498 S: 2 912 S: . C: retr 1 S: \textless message 1
contents\textgreater{} S: . C: dele 1 C: retr 2 S: \textless message 1
contents\textgreater{} S: . C: dele 2 C: quit S: +OK POP3 server signing
off '''

2.59

POP3 (more) and IMAP

\emph{more about POP3}

\begin{itemize}
\item
  previous example uses POP3 ``download and delete'' mode
\item
  \begin{itemize}
  \item
    Bob cannot re-read e-mail if he changes client
  \end{itemize}
\item
  POP3 ``download-and-keep'': copies of messages on different clients
\item
  POP3 is stateless across sessions
\end{itemize}

\emph{IMAP}

\begin{itemize}
\item
  keeps all messages in one place: at server
\item
  allows user to organize messages in folders
\item
  keeps user state across sessions:
\item
  \begin{itemize}
  \item
    names of folders and mappings between message IDs and folder name
  \end{itemize}
\end{itemize}

\hypertarget{25-dns}{%
\subsection{2.5 DNS}\label{25-dns}}

\begin{quote}
\begin{enumerate}
\def\labelenumi{\arabic{enumi}.}
\item
  主机 ip
\item
  主机 别名-规范名
\item
  邮件 别名-规范名
\item
  负载分配
\end{enumerate}

分布式: 单点故障, 通信容量
\end{quote}

2.61 DNS: domain name system

\emph{people:} many identifiers:

\begin{itemize}
\item
  \begin{itemize}
  \item
    SSN, name, passport \#
  \end{itemize}
\end{itemize}

\emph{Internet hosts, routers:}

\begin{itemize}
\item
  \begin{itemize}
  \item
    IP address (32 bit) - used for addressing datagrams
  \item
    ``name'', e.g., \url{www.yahoo.com} - used by humans
  \end{itemize}
\end{itemize}

\emph{Q:} how to map between IP address and name, and vice versa ?

\emph{Domain Name System:}

\begin{itemize}
\item
  \emph{distributed database} implemented in hierarchy of many
  \emph{name servers}
\item
  \emph{application-layer protocol:} hosts, name servers communicate to
  \emph{resolve} names (address/name translation)
\item
  \begin{itemize}
  \item
    note: core Internet function, implemented as application-layer
    protocol
  \item
    complexity at network's ``edge''
  \end{itemize}
\end{itemize}

2.62

DNS: services, structure

\emph{DNS services}

\begin{itemize}
\item
  hostname to IP address translation
\item
  host aliasing
\item
  \begin{itemize}
  \item
    canonical, alias names
  \end{itemize}
\item
  mail server aliasing
\item
  load distribution
\item
  \begin{itemize}
  \item
    replicated Web servers: many IP addresses correspond to one name
  \end{itemize}
\end{itemize}

\emph{why not centralize DNS?}

\begin{itemize}
\item
  single point of failure
\item
  traffic volume
\item
  distant centralized database
\item
  maintenance
\end{itemize}

A: doesn't scale!

2.63 DNS: a distributed, hierarchical database

\begin{quote}
根DNS服务器, 工作量大,像 baidu 这种要解析的是一群服务器
\end{quote}

\begin{figure}
\centering
\includegraphics{/Users/whichway/workspace/E_ESSEX/CE265/MD/CE265.assets/image-20210316104946417.png}
\caption{}
\end{figure}

\emph{client wants IP for \url{www.amazon.com}; 1**st} \emph{approx:}

\begin{itemize}
\item
  client queries root server to find com DNS server
\item
  client queries .com DNS server to get amazon.com DNS server
\item
  client queries amazon.com DNS server to get IP address for
  \url{www.amazon.com}
\end{itemize}

2.64

DNS: root name servers

\begin{itemize}
\item
  contacted by local name server that can not resolve name
\item
  root name server:
\item
  \begin{itemize}
  \item
    contacts authoritative name server if name mapping not known
  \item
    gets mapping
  \item
    returns mapping to local name server
  \end{itemize}
\end{itemize}

\begin{figure}
\centering
\includegraphics{/Users/whichway/workspace/E_ESSEX/CE265/MD/CE265.assets/image-20210316105711580.png}
\caption{}
\end{figure}

2.65

TLD, authoritative servers

\emph{top-level domain (TLD) servers:}

\begin{itemize}
\item
  \begin{itemize}
  \item
    responsible for com, org, net, edu, aero, jobs, museums, and all
    top-level country domains, e.g.: uk, fr, ca, jp
  \item
    Network Solutions maintains servers for .com TLD
  \item
    Educause for .edu TLD
  \end{itemize}
\end{itemize}

\emph{authoritative DNS servers:}

\begin{itemize}
\item
  \begin{itemize}
  \item
    organization's own DNS server(s), providing authoritative hostname
    to IP mappings for organization's named hosts
  \item
    can be maintained by organization or service provider
  \end{itemize}
\end{itemize}

2.66

Local DNS name server

\begin{itemize}
\item
  does not strictly belong to hierarchy
\item
  each ISP (residential ISP, company, university) has one
\item
  \begin{itemize}
  \item
    also called ``default name server''
  \end{itemize}
\item
  when host makes DNS query, query is sent to its local DNS server
\item
  \begin{itemize}
  \item
    has local cache of recent name-to-address translation pairs (but may
    be out of date!)
  \item
    acts as proxy, forwards query into hierarchy
  \end{itemize}
\end{itemize}

2.67

DNS name resolution example

\begin{itemize}
\item
  host at cis.poly.edu wants IP address for gaia.cs.umass.edu
\end{itemize}

\emph{iterated query:}

\begin{itemize}
\item
  contacted server replies with name of server to contact
\item
  ``I don't know this name, but ask this server''
\end{itemize}

\begin{figure}
\centering
\includegraphics{/Users/whichway/workspace/E_ESSEX/CE265/MD/CE265.assets/image-20210325190242499.png}
\caption{}
\end{figure}

2.68

DNS name resolution example

recursive query:

\begin{itemize}
\item
  puts burden of name resolution on contacted name server
\item
  heavy load at upper levels of hierarchy?
\end{itemize}

\begin{figure}
\centering
\includegraphics{/Users/whichway/workspace/E_ESSEX/CE265/MD/CE265.assets/image-20210325190404638.png}
\caption{}
\end{figure}

2.69

DNS: caching, updating records

\begin{itemize}
\item
  once (any) name server learns mapping, it \emph{caches} mapping
\item
  \begin{itemize}
  \item
    cache entries timeout (disappear) after some time (TTL)
  \item
    TLD servers typically cached in local name servers
  \end{itemize}
\item
  \begin{itemize}
  \item
    \begin{itemize}
    \item
      thus root name servers not often visited
    \end{itemize}
  \end{itemize}
\item
  cached entries may be \emph{out-of-date} (best effort name-to-address
  translation!)
\item
  \begin{itemize}
  \item
    if name host changes IP address, may not be known Internet-wide
    until all TTLs expire
  \end{itemize}
\item
  update/notify mechanisms proposed IETF standard
\item
  \begin{itemize}
  \item
    RFC 2136
  \end{itemize}
\end{itemize}

2.70

DNS records

\emph{DNS:} distributed db storing resource records (RR)

\[RR\ format: (name, value, type, ttl)\]

type=A

\begin{itemize}
\item
  \begin{itemize}
  \item
    \textbf{name} is hostname
  \item
    \textbf{value} is IP address
  \end{itemize}
\end{itemize}

type=NS

\begin{itemize}
\item
  \begin{itemize}
  \item
    \textbf{name} is domain (e.g., foo.com)
  \item
    \textbf{value} is hostname of authoritative name server for this
    domain
  \end{itemize}
\end{itemize}

type=CNAME

\begin{itemize}
\item
  \begin{itemize}
  \item
    \textbf{name} is alias name for some ``canonical'' (the real) name
  \item
    \textbf{\url{www.ibm.com}} is really
    \textbf{servereast.backup2.ibm.com}
  \end{itemize}
\item
  \begin{itemize}
  \item
    \textbf{value} is canonical name
  \end{itemize}
\end{itemize}

type=MX

\begin{itemize}
\item
  \begin{itemize}
  \item
    \textbf{value} is name of mailserver associated with \textbf{name}
  \end{itemize}
\end{itemize}

2.71

DNS protocol, messages

\begin{itemize}
\item
  \emph{query} and \emph{reply} messages, both with same \emph{message
  format}
\end{itemize}

msg header

\begin{itemize}
\item
  identification: 16 bit \# for query, reply to query uses same \#
\item
  flags:
\item
  \begin{itemize}
  \item
    query or reply
  \item
    recursion desired
  \item
    recursion available
  \item
    reply is authoritative
  \end{itemize}
\end{itemize}

\begin{figure}
\centering
\includegraphics{/Users/whichway/workspace/E_ESSEX/CE265/MD/CE265.assets/image-20210325190823371.png}
\caption{}
\end{figure}

2.72

\begin{figure}
\centering
\includegraphics{/Users/whichway/workspace/E_ESSEX/CE265/MD/CE265.assets/image-20210325190843660.png}
\caption{}
\end{figure}

2.73

Inserting records into DNS

\begin{itemize}
\item
  example: new startup ``Network Utopia''
\item
  register name networkuptopia.com at \emph{DNS registrar} (e.g.,
  Network Solutions)
\item
  \begin{itemize}
  \item
    provide names, IP addresses of authoritative name server (primary
    and secondary)
  \item
    registrar inserts two RRs into .com TLD server:
    \textbf{(networkutopia.com, dns1.networkutopia.com, NS)}
  \end{itemize}
\end{itemize}

\textbf{(dns1.networkutopia.com, 212.212.212.1, A)}

\begin{itemize}
\item
  create authoritative server type A record for
  \url{www.networkuptopia.com}; type MX record for networkutopia.com
\end{itemize}

2.74

Attacking DNS

DDoS attacks

\begin{itemize}
\item
  Bombard root servers with traffic
\item
  \begin{itemize}
  \item
    Not successful to date
  \item
    Traffic Filtering
  \item
    Local DNS servers cache IPs of TLD servers, allowing root server
    bypass
  \end{itemize}
\item
  Bombard TLD servers
\item
  \begin{itemize}
  \item
    Potentially more dangerous
  \end{itemize}
\end{itemize}

Redirect attacks

\begin{itemize}
\item
  Man-in-middle
\item
  \begin{itemize}
  \item
    Intercept queries
  \end{itemize}
\item
  DNS poisoning
\item
  \begin{itemize}
  \item
    Send bogus relies to DNS server, which caches
  \end{itemize}
\end{itemize}

Exploit DNS for DDoS

\begin{itemize}
\item
  Send queries with spoofed source address: target IP
\item
  Requires amplification
\end{itemize}

\hypertarget{26-p2p-applications}{%
\subsection{2.6 P2P applications}\label{26-p2p-applications}}

\begin{quote}
DNS

CS/P2P 下的最短分发时间

流行的P2P协议
\end{quote}

2.76 Pure P2P architecture

\begin{enumerate}
\def\labelenumi{\arabic{enumi}.}
\item
  {[} \hl{no always-on} {]} \emph{no} always-on server
\item
  \hl{{[} end-systems driectly} {]} arbitrary end systems directly
  communicate
\item
  \hl{{[} intermittently, change IP} {]} peers are intermittently
  connected and change IP addresses
\end{enumerate}

\emph{examples:}

\begin{itemize}
\item
  file distribution (BitTorrent)

  Streaming (KanKan)

  VoIP (Skype)
\end{itemize}

\begin{figure}
\centering
\includegraphics{/Users/whichway/workspace/E_ESSEX/CE265/MD/CE265.assets/image-20210325191120618.png}
\caption{}
\end{figure}

2.77

File distribution: client-server vs P2P

\emph{Question**:} how much time to distribute file (size \emph{F}) from
one server to \emph{N peers}?

\begin{itemize}
\item
  peer upload/download capacity is limited resource
\end{itemize}

\begin{figure}
\centering
\includegraphics{/Users/whichway/workspace/E_ESSEX/CE265/MD/CE265.assets/image-20210325191311485.png}
\caption{}
\end{figure}

2.78

File distribution time: client-server

\begin{itemize}
\item
  \emph{server transmission:} must sequentially send (upload) \emph{N}
  file copies:
\item
  \begin{itemize}
  \item
    time to send one copy: \emph{F/u**s}
  \item
    time to send N copies: \emph{NF/u**s}
  \end{itemize}
\item
  \emph{client:} each client must download file copy
\item
  \begin{itemize}
  \item
    \(d_{min}\) = min client download rate
  \item
    min client download time: \(F/dmin\)
  \end{itemize}
\end{itemize}

time to distribute F to N clients using client-server approach

\[D_{c-s} \geq max\{N\frac{F}{u_s},\frac{F}{d_{min}}\}\]

\begin{figure}
\centering
\includegraphics{/Users/whichway/workspace/E_ESSEX/CE265/MD/CE265.assets/image-20210325192251650.png}
\caption{}
\end{figure}

2.79

File distribution time: P2P

\begin{itemize}
\item
  \emph{server transmission:} must upload at least one copy
\item
  \begin{itemize}
  \item
    time to send one copy: \emph{F/u**s}
  \end{itemize}
\item
  \emph{client:} each client must download file copy
\item
  \begin{itemize}
  \item
    min client download time: F/dmin
  \end{itemize}
\item
  \emph{clients:} as aggregate must download \emph{NF} bits
\item
  \begin{itemize}
  \item
    max upload rate (limting max download rate) is us + Σui
  \end{itemize}
\end{itemize}

time to distribute F to N clients using P2P approach

\[D_{P2P} > max\{\frac{F}{u_s},\frac{F}{d_{min}},N \frac{F}{(u_s + \sum_{i=1}^{N} {u_i})}\}\]

increases linearly in N \ldots{}

\ldots{} but so does this, as each peer brings service capacity

2.80

Client-server vs. P2P: example

client upload rate = u, F/u = 1 hour, us = 10u, dmin ≥ us

\begin{figure}
\centering
\includegraphics{/Users/whichway/workspace/E_ESSEX/CE265/MD/CE265.assets/image-20210325193331899.png}
\caption{}
\end{figure}

2.81

P2P file distribution: BitTorrent

\begin{itemize}
\item
  file divided into 256Kb chunks{[}片{]}
\item
  peers in torrent send/receive file chunks
\end{itemize}

\emph{tracker:} tracks peers participating in torrent

\emph{torrent:} group of peers exchanging chunks of a file

\begin{figure}
\centering
\includegraphics{/Users/whichway/workspace/E_ESSEX/CE265/MD/CE265.assets/image-20210325210412848.png}
\caption{}
\end{figure}

2.82

P2P file distribution: BitTorrent

\begin{itemize}
\item
  peer joining torrent:
\item
  \begin{itemize}
  \item
    has no chunks, but will accumulate them over time from other peers
  \item
    registers with tracker to get list of peers, connects to subset of
    peers (``neighbors'')
  \end{itemize}
\end{itemize}

\begin{figure}
\centering
\includegraphics{/Users/whichway/workspace/E_ESSEX/CE265/MD/CE265.assets/image-20210325210442235.png}
\caption{}
\end{figure}

\begin{itemize}
\item
  while downloading, peer uploads chunks to other peers
\item
  peer may change peers with whom it exchanges chunks
\item
  \emph{churn:} peers may come and go
\item
  once peer has entire file, it may (selfishly) leave or
  (altruistically) remain in torrent
\end{itemize}

2.83

BitTorrent: requesting, sending file chunks

\emph{requesting chunks{[}厚片{]}:}

\begin{itemize}
\item
  at any given time, different peers have different subsets of file
  chunks
\item
  periodically, Alice asks each peer for list of chunks that they have
\item
  Alice requests missing chunks from peers, rarest first
\end{itemize}

\emph{sending chunks: tit-for-tat}

\begin{itemize}
\item
  Alice sends chunks to those four peers currently sending her chunks
  \emph{at highest rate}
\item
  \begin{itemize}
  \item
    other peers are choked by Alice (do not receive chunks from her)
  \item
    re-evaluate top 4 every10 secs
  \end{itemize}
\item
  every 30 secs: randomly select another peer, starts sending chunks
\item
  \begin{itemize}
  \item
    ``optimistically unchoke'' this peer
  \item
    newly chosen peer may join top 4
  \end{itemize}
\end{itemize}

2.84

BitTorrent: tit-for-tat

(1) Alice ``optimistically unchokes'' Bob

(2) Alice becomes one of Bob's top-four providers; Bob reciprocates

(3) Bob becomes one of Alice's top-four providers

\begin{figure}
\centering
\includegraphics{/Users/whichway/workspace/E_ESSEX/CE265/MD/CE265.assets/image-20210325210622318.png}
\caption{}
\end{figure}

\emph{higher upload rate:} find better trading partners, get file faster
!

2.85

\hypertarget{distributed-hash-table-dht-ux5206ux5e03ux5f0fux54c8ux5e0cux8868}{%
\subsubsection{\texorpdfstring{\hl{⭐️Distributed Hash Table (DHT)
{[}分布式哈希表{]}}}{⭐️Distributed Hash Table (DHT) {[}分布式哈希表{]}}}\label{distributed-hash-table-dht-ux5206ux5e03ux5f0fux54c8ux5e0cux8868}}

\begin{enumerate}
\def\labelenumi{\arabic{enumi}.}
\item
  Hash table
\item
  DHT paradigm{[}范例{]}
\item
  Circular DHT and overlay networks
\item
  Peer churn
\end{enumerate}

2.86

Simple Database

Simple database with(key, value) pairs:

\begin{itemize}
\item
  key: human name; value: social security \#
\end{itemize}

\begin{longtable}[]{@{}ll@{}}
\toprule
\textbf{Key} & \textbf{Value}\tabularnewline
\midrule
\endhead
John Washington & 132-54-3570\tabularnewline
Diana Louise Jones & 761-55-3791\tabularnewline
Xiaoming Liu & 385-41-0902\tabularnewline
Rakesh Gopal & 441-89-1956\tabularnewline
Linda Cohen & 217-66-5609\tabularnewline
\ldots\ldots. & \ldots\ldots\ldots{}\tabularnewline
Lisa Kobayashi & 177-23-0199\tabularnewline
\bottomrule
\end{longtable}

\begin{itemize}
\item
  key: movie title; value: IP address
\end{itemize}

2.87

Hash Table

\begin{itemize}
\item
  More convenient to store and search on numerical representation of key
\item
  key = hash(original key)
\end{itemize}

\begin{longtable}[]{@{}lll@{}}
\toprule
\textbf{Original Key} & \textbf{Key} & \textbf{Value}\tabularnewline
\midrule
\endhead
John Washington & 8962458 & 132-54-3570\tabularnewline
Diana Louise Jones & 7800356 & 761-55-3791\tabularnewline
Xiaoming Liu & 1567109 & 385-41-0902\tabularnewline
Rakesh Gopal & 2360012 & 441-89-1956\tabularnewline
Linda Cohen & 5430938 & 217-66-5609\tabularnewline
\ldots\ldots. & & \ldots\ldots\ldots{}\tabularnewline
Lisa Kobayashi & 9290124 & 177-23-0199\tabularnewline
\bottomrule
\end{longtable}

2.88

Distributed Hash Table (DHT)

\begin{enumerate}
\def\labelenumi{\arabic{enumi}.}
\item
  Distribute (key, value) pairs over millions of peers

  pairs are evenly distributed over peers
\item
  Any peer can query database with a key

  database returns value for the key

  To resolve query, small number of messages exchanged among peers
\item
  Each peer only knows about a small number of other peers
\item
  Robust to peers coming and going (churn)
\end{enumerate}

2.89

Assign key-value pairs to peers

\begin{enumerate}
\def\labelenumi{\arabic{enumi}.}
\item
  rule: assign key-value pair to the peer that has the \emph{closest}
  ID.
\item
  convention: closest is the \emph{immediate successor} of the key.
\item
  e.g., ID space \{0,1,2,3,\ldots,63\}
\item
  suppose 8 peers: 1,12,13,25,32,40,48,60
\item
  \begin{itemize}
  \item
    If key = 51, then assigned to peer 60
  \item
    If key = 60, then assigned to peer 60
  \item
    If key = 61, then assigned to peer 1
  \end{itemize}
\end{enumerate}

2.90

Circular DHT

\begin{itemize}
\item
  each peer \emph{only} aware of immediate successor and predecessor.
\end{itemize}

\begin{figure}
\centering
\includegraphics{/Users/whichway/workspace/E_ESSEX/CE265/MD/CE265.assets/image-20210325211314583.png}
\caption{}
\end{figure}

\begin{figure}
\centering
\includegraphics{/Users/whichway/workspace/E_ESSEX/CE265/MD/CE265.assets/image-20210325211320381.png}
\caption{}
\end{figure}

2.91

Resolving a query

\emph{O(N)} messages

on avgerage to resolve

query, when there

are \emph{N} peers

\begin{figure}
\centering
\includegraphics{/Users/whichway/workspace/E_ESSEX/CE265/MD/CE265.assets/image-20210325211459589.png}
\caption{}
\end{figure}

2.92

Circular DHT( Distributed Hash Table ) with shortcuts

\begin{figure}
\centering
\includegraphics{/Users/whichway/workspace/E_ESSEX/CE265/MD/CE265.assets/image-20210325211514050.png}
\caption{}
\end{figure}

\begin{itemize}
\item
  each peer keeps track of IP addresses of predecessor, successor, short
  cuts.
\item
  reduced from 6 to 3 messages.
\item
  possible to design shortcuts with \emph{O(log N)} neighbors,
  \emph{O(log N)} messages in query
\end{itemize}

2.93

Peer churn

handling peer churn:

\begin{enumerate}
\def\labelenumi{\arabic{enumi}.}
\item
  peers may come and go (churn)
\item
  each peer knows address of its two successors
\item
  each peer periodically pings its two successors to check aliveness
\item
  if immediate successor leaves, choose next successor as new immediate
  successor
\end{enumerate}

\emph{example: peer 5 abruptly leaves}

\begin{figure}
\centering
\includegraphics{/Users/whichway/workspace/E_ESSEX/CE265/MD/CE265.assets/image-20210420093346916.png}
\caption{}
\end{figure}

2.94

Peer churn

handling peer churn:

\begin{itemize}
\item
  peers may come and go (churn)
\item
  each peer knows address of its two successors
\item
  each peer periodically pings its two successors to check aliveness
\item
  if immediate successor leaves, choose next successor as new immediate
  successor
\end{itemize}

\emph{example: peer 5 abruptly leaves}

\begin{itemize}
\item
  peer 4 detects peer 5's departure; makes 8 its immediate successor
\item
  4 asks 8 who its immediate successor is; makes 8's immediate successor
  its second successor.
\end{itemize}

\begin{figure}
\centering
\includegraphics{/Users/whichway/workspace/E_ESSEX/CE265/MD/CE265.assets/image-20210420093414335.png}
\caption{}
\end{figure}

\hypertarget{27-socket-programming-with-udp-and-tcp}{%
\subsection{2.7 socket programming with UDP and
TCP}\label{27-socket-programming-with-udp-and-tcp}}

\begin{quote}
TCP 套接字的建立 Review Quesation 和 一些判断题
\end{quote}

2.96

\emph{goal:} learn how to build client/server applications that
communicate using sockets

\emph{socket:} door between application process and end-end-transport
protocol

\begin{figure}
\centering
\includegraphics{/Users/whichway/workspace/E_ESSEX/CE265/MD/CE265.assets/image-20210419013504222.png}
\caption{}
\end{figure}

2.97 Socket programming

\emph{Two socket types for two transport services:}

\begin{itemize}
\item
  \begin{itemize}
  \item
    \emph{UDP:} unreliable datagram
  \item
    \emph{TCP:} reliable, byte stream-oriented
  \end{itemize}
\end{itemize}

\emph{Application Example:}

\begin{enumerate}
\def\labelenumi{\arabic{enumi}.}
\item
  Client reads a line of characters (data) from its keyboard and sends
  the data to the server.
\item
  The server receives the data and converts characters to uppercase.
\item
  The server sends the modified data to the client.
\item
  The client receives the modified data and displays the line on its
  screen.
\end{enumerate}

2.98

Socket programming \emph{with UDP}

UDP: no ``connection'' between client \& server

\begin{itemize}
\item
  no handshaking before sending data
\item
  sender explicitly attaches IP destination address and port \# to each
  packet
\item
  rcvr extracts sender IP address and port\# from received packet
\end{itemize}

UDP: transmitted data may be lost or received out-of-order

Application viewpoint:

\begin{itemize}
\item
  UDP provides \emph{unreliable} transfer of groups of bytes
  (``datagrams'') between client and server
\end{itemize}

2.99

Client/server socket interaction: UDP

\begin{figure}
\centering
\includegraphics{/Users/whichway/workspace/E_ESSEX/CE265/MD/CE265.assets/image-20210419013801158.png}
\caption{}
\end{figure}

2.100

Example app: Python UDP client

\begin{Shaded}
\begin{Highlighting}[]
\ImportTok{from}\NormalTok{ socket }\ImportTok{import} \OperatorTok{*}   
\CommentTok{\# include Python’s socket library}
\NormalTok{serverName }\OperatorTok{=}\NormalTok{ ‘hostname’}
\NormalTok{serverPort }\OperatorTok{=} \DecValTok{12000}
\NormalTok{clientSocket }\OperatorTok{=}\NormalTok{ socket(socket.AF\_INET, socket.SOCK\_DGRAM) }
\CommentTok{\# create UDP socket for server}
\NormalTok{message }\OperatorTok{=} \BuiltInTok{raw\_input}\NormalTok{(’Input lowercase sentence:’) }
\CommentTok{\# get user keyboard input }
\NormalTok{clientSocket.sendto(message,(serverName, serverPort)) }
\CommentTok{\# Attach server name, port to message; send into socket}
\NormalTok{modifiedMessage, serverAddress }\OperatorTok{=}\NormalTok{ clientSocket.recvfrom(}\DecValTok{2048}\NormalTok{) }
\CommentTok{\# read reply characters from socket into string}
\BuiltInTok{print}\NormalTok{ modifiedMessage }
\CommentTok{\# print out received string and close socket}
\NormalTok{clientSocket.close()}
\end{Highlighting}
\end{Shaded}

2.101

Python UDPServer

\begin{Shaded}
\begin{Highlighting}[]
\ImportTok{from}\NormalTok{ socket }\ImportTok{import} \OperatorTok{*}
\NormalTok{serverPort }\OperatorTok{=} \DecValTok{12000}
\NormalTok{serverSocket }\OperatorTok{=}\NormalTok{ socket(AF\_INET, SOCK\_DGRAM)}
\CommentTok{\# create UDP socket}
\NormalTok{serverSocket.bind((}\StringTok{\textquotesingle{}\textquotesingle{}}\NormalTok{, serverPort))}
\CommentTok{\# bind socket to local port number 12000}
\BuiltInTok{print}\NormalTok{ “The server }\KeywordTok{is}\NormalTok{ ready to receive”}
\ControlFlowTok{while} \DecValTok{1}\NormalTok{:}
\CommentTok{\# loop forever}
\NormalTok{    message, clientAddress }\OperatorTok{=}\NormalTok{ serverSocket.recvfrom(}\DecValTok{2048}\NormalTok{)}
    \CommentTok{\# Read from UDP socket into message, getting client’s address (client IP and port)}
    
\NormalTok{    modifiedMessage }\OperatorTok{=}\NormalTok{ message.upper()}
\NormalTok{    serverSocket.sendto(modifiedMessage, clientAddress)}
    \CommentTok{\# send upper case string back to this client}
\end{Highlighting}
\end{Shaded}

2.102

Socket programming with TCP

\begin{enumerate}
\def\labelenumi{\arabic{enumi}.}
\item
  \hl{client must contact server}
\end{enumerate}

\begin{itemize}
\item
  \hl{{[} server, first running {]}} server process must \hl{first be
  running}
\item
  \hl{{[} server, created socket(door) {]}} server must \hl{have created
  socket (door)} that \hl{welcomes} client's contact
\end{itemize}

\begin{enumerate}
\def\labelenumi{\arabic{enumi}.}
\item
  \hl{client contacts server by:}
\end{enumerate}

\begin{itemize}
\item
  \hl{{[} client creates TCP socket, specifies IP, port number {]}}
  Creating \hl{TCP socket, specifying IP address, port number} of server
  process
\item
  \hl{{[}client TCP connects to server TCP{]}} \emph{when client creates
  socket:} \hl{client TCP establishes connection to server TCP}
\end{itemize}

when contacted by client, \emph{server TCP creates new socket} for
server process to communicate with that particular client

\begin{itemize}
\item
  allows server to talk with multiple clients
\item
  source port numbers used to distinguish clients (more in Chap 3)
\end{itemize}

application viewpoint:

TCP provides reliable, in-order byte-stream transfer (``pipe'') between
client and server

2.103

Client/server socket interaction: TCP

\begin{figure}
\centering
\includegraphics{/Users/whichway/workspace/E_ESSEX/CE265/MD/CE265.assets/image-20210419113258893.png}
\caption{}
\end{figure}

2.104

Python TCP Client

\begin{Shaded}
\begin{Highlighting}[]
\ImportTok{from}\NormalTok{ socket }\ImportTok{import} \OperatorTok{*}
\NormalTok{serverName }\OperatorTok{=}\NormalTok{ ’servername’}
\NormalTok{serverPort }\OperatorTok{=} \DecValTok{12000}
\NormalTok{clientSocket }\OperatorTok{=}\NormalTok{ socket(AF\_INET, SOCK\_STREAM)}
\CommentTok{\# create TCP socket for server, remote port 12000}
\NormalTok{clientSocket.}\ExtensionTok{connect}\NormalTok{((serverName,serverPort))}
\NormalTok{sentence }\OperatorTok{=} \BuiltInTok{raw\_input}\NormalTok{(‘Input lowercase sentence:’)}
\NormalTok{clientSocket.send(sentence)}
\CommentTok{\# No need to attach server name, port }
\NormalTok{modifiedSentence }\OperatorTok{=}\NormalTok{ clientSocket.recv(}\DecValTok{1024}\NormalTok{)}
\BuiltInTok{print}\NormalTok{ ‘From Server:’, modifiedSentence}
\NormalTok{clientSocket.close()}

\end{Highlighting}
\end{Shaded}

2.105

Python TCP Server

\begin{Shaded}
\begin{Highlighting}[]
 \ImportTok{from}\NormalTok{ socket }\ImportTok{import} \OperatorTok{*}
\NormalTok{serverPort }\OperatorTok{=} \DecValTok{12000}
\NormalTok{serverSocket }\OperatorTok{=}\NormalTok{ socket(AF\_INET,SOCK\_STREAM)}
\CommentTok{\# create TCP welcoming socket}
\NormalTok{serverSocket.bind((‘’,serverPort))}
\NormalTok{serverSocket.listen(}\DecValTok{1}\NormalTok{)}
\CommentTok{\# server begins listening for  incoming TCP requests}
\BuiltInTok{print}\NormalTok{ ‘The server }\KeywordTok{is}\NormalTok{ ready to receive’}
\ControlFlowTok{while} \DecValTok{1}\NormalTok{:}
\CommentTok{\# loop forever}
\NormalTok{     connectionSocket, addr }\OperatorTok{=}\NormalTok{ serverSocket.accept()}
     \CommentTok{\# server waits on accept() for incoming requests, new socket created on return}
\NormalTok{     sentence }\OperatorTok{=}\NormalTok{ connectionSocket.recv(}\DecValTok{1024}\NormalTok{)}
     \CommentTok{\# read bytes from socket (but not address as in UDP)}
\NormalTok{     capitalizedSentence }\OperatorTok{=}\NormalTok{ sentence.upper()}
\NormalTok{     connectionSocket.send(capitalizedSentence)}
     \CommentTok{\# close connection to this client (but not welcoming socket)s}
\NormalTok{     connectionSocket.close()}

\end{Highlighting}
\end{Shaded}

\hypertarget{chapter-2-summary}{%
\subsection{Chapter 2: summary}\label{chapter-2-summary}}

\emph{our study of network apps now complete!}

\begin{enumerate}
\def\labelenumi{\arabic{enumi}.}
\item
  \hl{application architectures}
\item
  \begin{itemize}
  \item
    \hl{client-server}
  \item
    \hl{P2P}
  \end{itemize}
\item
  \hl{application service requirements:}
\item
  \begin{itemize}
  \item
    \hl{reliability, bandwidth, delay}
  \end{itemize}
\item
  \hl{Internet transport service model}
\item
  \begin{itemize}
  \item
    \hl{connection-oriented{[}面向链接的{]}, reliable{[}可靠的{]}: TCP}
  \item
    \hl{unreliable{[}不可靠的{]}, datagrams{[}面向数据报的{]}: UDP}
  \end{itemize}
\end{enumerate}

\begin{itemize}
\item
  \hl{specific protocols:}
\item
  \begin{enumerate}
  \def\labelenumi{\arabic{enumi}.}
  \item
    \hl{HTTP}
  \item
    \hl{FTP}
  \item
    \hl{SMTP, POP, IMAP}
  \item
    \hl{DNS}
  \item
    \hl{P2P: BitTorrent, DHT}
  \end{enumerate}
\item
  \hl{socket programming: TCP, UDP sockets}
\end{itemize}

\emph{most importantly: learned about protocols!}

\begin{enumerate}
\def\labelenumi{\arabic{enumi}.}
\item
  \hl{typical request/reply message exchange:}
\item
  \begin{enumerate}
  \def\labelenumii{\arabic{enumii}.}
  \item
    \hl{client requests info or service}
  \item
    \hl{server responds with data, status code}
  \end{enumerate}
\item
  \hl{message formats:}
\item
  \begin{enumerate}
  \def\labelenumii{\arabic{enumii}.}
  \item
    \hl{headers: fields giving info about data}
  \item
    \hl{data: info being communicated}
  \end{enumerate}
\end{enumerate}

\emph{important themes:}

\begin{enumerate}
\def\labelenumi{\arabic{enumi}.}
\item
  \hl{control vs. data msgs}
\item
  \begin{itemize}
  \item
    \hl{in-band, out-of-band}
  \end{itemize}
\item
  \hl{centralized vs. decentralized}
\item
  \hl{stateless vs. stateful}
\item
  \hl{reliable vs. unreliable msg transfer}
\item
  \hl{``complexity at network edge''}
\end{enumerate}

\textbf{Chapter 1 Additional Slides}

2.109

\begin{figure}
\centering
\includegraphics{/Users/whichway/workspace/E_ESSEX/CE265/MD/CE265.assets/image-20210419014202950.png}
\caption{}
\end{figure}

\hypertarget{chapter-3-transport-layer}{%
\section{Chapter 3 Transport Layer}\label{chapter-3-transport-layer}}

3.1 transport-layer services

3.2 multiplexing and demultiplexing

3.3 connectionless transport: UDP

3.4 principles of reliable data transfer

3.5 connection-oriented transport: TCP

\begin{itemize}
\item
  \begin{enumerate}
  \def\labelenumi{\arabic{enumi}.}
  \item
    segment structure
  \item
    reliable data transfer
  \item
    flow control
  \item
    connection management
  \end{enumerate}
\end{itemize}

3.6 principles of congestion control

3.7 TCP congestion control

\begin{quote}
都需要大家很认真的去掌握,没有特别简单的
\end{quote}

our goals:

\begin{itemize}
\item
  understand principles behind transport layer services:
\item
  \begin{enumerate}
  \def\labelenumi{\arabic{enumi}.}
  \item
    multiplexing, demultiplexing
  \item
    reliable data transfer
  \item
    flow control
  \item
    congestion control
  \end{enumerate}
\item
  learn about Internet transport layer protocols:
\item
  \begin{enumerate}
  \def\labelenumi{\arabic{enumi}.}
  \item
    UDP: connectionless transport
  \item
    TCP: connection-oriented reliable transport
  \item
    TCP congestion control
  \end{enumerate}
\end{itemize}

\begin{quote}
我们在本章采用的教学方法是,交替地讨论运输层的原理和这些
原理在现有的协议中是如何实现的。 与往常一
样,我们将特別关注因特网协议,即TCP和UDP运输层协议。

我们将从讨论运输层和网络层的关系开始。这就进人了研究运输层第一个关键功能的阶段,即将网络层的在两个端系统之间的交付服务扩展到运行在两个不同端系统上的应用层进程之间的交付服务。我们将在讨论因特网的无连接运输协议UDP时阐述这个功能。

然后我们重新回到原理学习上,面对计算机网络中最为基础性的问题之一,即两个实体
怎 样 才 能 在 一 种 会 丢 失 或 损 坏 数 据 的 媒 体 上 可 靠 地 通 信
.通 过一系列复 杂 性不断 增加 (从而更真实!)的情况,我们将逐步建 立起
一套被运输 协议用来解决这些问题 的技术.
然后,我们将说明这些原理是如何体现在因特网面向连接的运输协议TCP中的。
\end{quote}

3.4

Transport services and protocols

\begin{itemize}
\item
  provide \emph{logical communication} between app processes running on
  different hosts
\item
  transport protocols run in end systems
\item
  \begin{itemize}
  \item
    send side: breaks app messages into \emph{segments}, passes to
    network layer
  \item
    rcv side: reassembles segments into messages, passes to app layer
  \end{itemize}
\item
  more than one transport protocol available to apps
\item
  \begin{itemize}
  \item
    Internet: TCP and UDP
  \end{itemize}
\end{itemize}

\begin{figure}
\centering
\includegraphics{/Users/whichway/workspace/E_ESSEX/CE265/MD/CE265.assets/image-20210329141135912.png}
\caption{}
\end{figure}

\hypertarget{31-transport-layer-services}{%
\subsection{3.1 transport-layer
services}\label{31-transport-layer-services}}

3.5

Transport vs. network layer

\begin{itemize}
\item
  \emph{network layer:} logical communication between hosts{[}主机{]}
\item
  \emph{transport layer:} logical communication between processes
\item
  \begin{itemize}
  \item
    relies on, enhances, network layer services
  \end{itemize}
\end{itemize}

\emph{household{[}家庭{]} analogy{[}类推{]}:}

\emph{12 kids in Ann's house sending letters to 12 kids in Bill's
house:}

\begin{itemize}
\item
  hosts = houses
\item
  processes = kids
\item
  app messages = letters in envelopes
\item
  transport protocol = Ann and Bill who demux to in-house siblings
\item
  network-layer protocol = postal service
\end{itemize}

3.6

Internet transport-layer protocols

\begin{itemize}
\item
  reliable, in-order delivery (TCP)
\item
  \begin{itemize}
  \item
    congestion control
  \item
    flow control
  \item
    connection setup
  \end{itemize}
\item
  unreliable, unordered delivery: UDP
\item
  \begin{itemize}
  \item
    no-frills extension of ``best-effort'' IP
  \end{itemize}
\item
  services not available:
\item
  \begin{itemize}
  \item
    delay guarantees
  \item
    bandwidth guarantees
  \end{itemize}
\end{itemize}

\begin{figure}
\centering
\includegraphics{/Users/whichway/workspace/E_ESSEX/CE265/MD/CE265.assets/image-20210329141449598.png}
\caption{}
\end{figure}

\hypertarget{32-multiplexingux591aux8defux590dux7528-and-demultiplexingux591aux8defux5206ux89e3}{%
\subsection{3.2 multiplexing{[}多路复用{]} and
demultiplexing{[}多路分解{]}}\label{32-multiplexingux591aux8defux590dux7528-and-demultiplexingux591aux8defux5206ux89e3}}

3.8

Multiplexing/demultiplexing

\begin{quote}
网络层提供的主机到
主机交付服务延伸到为运行在主机上的应用程序提供进程到进程的交付服务。
\end{quote}

\emph{multiplexing at sender:}

handle data from multiple sockets, add transport header (later used for
demultiplexing)

\emph{demultiplexing at receiver:}

use header info to deliver received segments to correct socket

\begin{figure}
\centering
\includegraphics{/Users/whichway/workspace/E_ESSEX/CE265/MD/CE265.assets/image-20210329143100950.png}
\caption{}
\end{figure}

3.9

How demultiplexing works

\begin{itemize}
\item
  host receives IP datagrams
\item
  \begin{itemize}
  \item
    each datagram has source IP address, destination IP address
  \item
    each datagram carries one transport-layer segment
  \item
    each segment has source, destination port number
  \end{itemize}
\item
  host uses \emph{IP addresses \& port numbers} to direct segment to
  appropriate socket
\end{itemize}

\begin{figure}
\centering
\includegraphics{/Users/whichway/workspace/E_ESSEX/CE265/MD/CE265.assets/image-20210329141823375.png}
\caption{}
\end{figure}

3.10 Connectionless demultiplexing

\begin{quote}
UDP 目的端口号一样,来源不一样,在同一端口号
\end{quote}

\begin{itemize}
\item
  \emph{recall:} created socket has host-local port \#:
\end{itemize}

\textbf{DatagramSocket mySocket1 = new
DatagramSocket(*}*12534\textbf{});**

\begin{itemize}
\item
  when host receives UDP segment:
\item
  \begin{itemize}
  \item
    checks destination port \# in segment
  \item
    directs UDP segment to socket with that port \#
  \end{itemize}
\item
  \emph{recall:} when creating datagram to send into UDP socket, must
  specify
\item
  \begin{itemize}
  \item
    destination IP address
  \item
    destination port \#
  \end{itemize}
\end{itemize}

IP datagrams with \emph{same dest. port \#,} but different source IP
addresses and/or source port numbers will be directed to \emph{same
socket} at dest

3.11

Connectionless demux: example

\begin{figure}
\centering
\includegraphics{/Users/whichway/workspace/E_ESSEX/CE265/MD/CE265.assets/image-20210329141951449.png}
\caption{}
\end{figure}

3.12

Connection-oriented demux

\begin{itemize}
\item
  TCP socket identified by 4-tuple:
\item
  \begin{itemize}
  \item
    source IP address
  \item
    source port number
  \item
    dest IP address
  \item
    dest port number
  \end{itemize}
\item
  demux: receiver uses all four values to direct segment to appropriate
  socket
\item
  server host may support many simultaneous TCP sockets:
\item
  \begin{itemize}
  \item
    each socket identified by its own 4-tuple
  \end{itemize}
\item
  web servers have different sockets for each connecting client
\item
  \begin{itemize}
  \item
    non-persistent HTTP will have different socket for each request
  \end{itemize}
\end{itemize}

3.13

Connection-oriented demux: example

\begin{figure}
\centering
\includegraphics{/Users/whichway/workspace/E_ESSEX/CE265/MD/CE265.assets/image-20210329142244164.png}
\caption{}
\end{figure}

three segments, all destined to IP address: B, dest port: 80 are
demultiplexed to \emph{different} sockets

3.14

Connection-oriented demux: example

\begin{figure}
\centering
\includegraphics{/Users/whichway/workspace/E_ESSEX/CE265/MD/CE265.assets/image-20210329142346192.png}
\caption{}
\end{figure}

\hypertarget{33-connectionless-transport-udp}{%
\subsection{3.3 connectionless transport:
UDP}\label{33-connectionless-transport-udp}}

3.16

UDP: User Datagram Protocol {[}RFC 768{]}

\begin{itemize}
\item
  ``no frills,'' ``bare bones'' Internet transport protocol
\item
  ``best effort'' service, UDP segments may be:
\item
  \begin{itemize}
  \item
    lost
  \item
    delivered out-of-order to app
  \end{itemize}
\item
  \emph{connectionless:}
\item
  \begin{itemize}
  \item
    no handshaking between UDP sender, receiver
  \item
    each UDP segment handled independently of others
  \end{itemize}
\item
  UDP use:
\item
  \begin{itemize}
  \item
    streaming multimedia apps (loss tolerant, rate sensitive)
  \item
    DNS
  \item
    SNMP
  \end{itemize}
\item
  reliable transfer over UDP:
\item
  \begin{itemize}
  \item
    add reliability at application layer
  \item
    application-specific error recovery!
  \end{itemize}
\end{itemize}

3.17

Internet apps: application, transport protocols

\begin{longtable}[]{@{}lll@{}}
\toprule
\textbf{Application} & \textbf{Application} &
\textbf{Underlying}\tabularnewline
\midrule
\endhead
& \textbf{layer protocol} & \textbf{transport protocol}\tabularnewline
e-mail & &\tabularnewline
remote terminal access & SMTP {[}RFC 2821{]} & TCP\tabularnewline
Web & Telnet {[}RFC 854{]} & TCP\tabularnewline
file transfer & HTTP {[}RFC 2616{]} & TCP\tabularnewline
streaming multimedia & FTP {[}RFC 959{]} & TCP\tabularnewline
& proprietary & TCP or UDP\tabularnewline
Internet telephony & (e.g. RealNetworks) &\tabularnewline
\bottomrule
\end{longtable}

3.18

UDP: segment header

\begin{figure}
\centering
\includegraphics{/Users/whichway/workspace/E_ESSEX/CE265/MD/CE265.assets/image-20210329142622056.png}
\caption{}
\end{figure}

why is there a UDP?

\begin{enumerate}
\def\labelenumi{\arabic{enumi}.}
\item
  no connection establishment (which can add delay)
\item
  simple: no connection state at sender, receiver
\item
  small header size
\item
  no congestion control: UDP can blast away as fast as desired
\end{enumerate}

'''http 0632 0035 001C E217 '''

发送端口号 0632

\begin{verbatim}
0000 0110 0011 0010
\end{verbatim}

目的端口号 0035 == 53

\begin{verbatim}
0000 0000 0011 0101
\end{verbatim}

\[=2^5+2^4+2^2+2^0=32+16+5=53\]

端口号是53

payload 28 bytes

3.19 UDP checksum

\begin{quote}
UDP 检查和

最后检查和补上1
\end{quote}

\emph{Goal:} detect ``errors'' (e.g., flipped bits) in transmitted
segment

sender:

\begin{itemize}
\item
  treat segment contents, including header fields, as sequence of 16-bit
  integers
\item
  checksum: addition (one's complement sum) of segment contents
\item
  sender puts checksum value into UDP checksum field
\end{itemize}

receiver:

\begin{itemize}
\item
  compute checksum of received segment
\item
  check if computed checksum equals checksum field value:
\item
  \begin{itemize}
  \item
    NO - error detected
  \item
    YES - no error detected. \emph{But maybe errors nonetheless?} More
    later \ldots.
  \end{itemize}
\end{itemize}

3.20 Internet checksum: example

example: add two 16-bit integers

\begin{figure}
\centering
\includegraphics{/Users/whichway/workspace/E_ESSEX/CE265/MD/CE265.assets/image-20210329143236295.png}
\caption{}
\end{figure}

\emph{Note:} when adding numbers, a carryout from the most significant
bit needs to be added to the result

\hypertarget{34-principles-of-reliable-data-transfer}{%
\subsection{3.4 principles of reliable data
transfer}\label{34-principles-of-reliable-data-transfer}}

3.22

Principles of reliable data transfer

\begin{itemize}
\item
  important in application, transport, link layers
\item
  \begin{itemize}
  \item
    top-10 list of important networking topics!
  \end{itemize}
\end{itemize}

\begin{figure}
\centering
\includegraphics{/Users/whichway/workspace/E_ESSEX/CE265/MD/CE265.assets/image-20210329143407780.png}
\caption{}
\end{figure}

\begin{itemize}
\item
  characteristics of unreliable channel will determine complexity of
  reliable data transfer protocol (rdt)
\end{itemize}

3.23

Principles of reliable data transfer

\begin{itemize}
\item
  important in application, transport, link layers
\item
  \begin{itemize}
  \item
    top-10 list of important networking topics!
  \end{itemize}
\end{itemize}

\begin{figure}
\centering
\includegraphics{/Users/whichway/workspace/E_ESSEX/CE265/MD/CE265.assets/image-20210329143600411.png}
\caption{}
\end{figure}

\begin{itemize}
\item
  characteristics of unreliable channel will determine complexity of
  reliable data transfer protocol (rdt)
\end{itemize}

3.24

Principles of reliable data transfer

\begin{itemize}
\item
  important in application, transport, link layers
\item
  \begin{itemize}
  \item
    top-10 list of important networking topics!
  \end{itemize}
\end{itemize}

\begin{figure}
\centering
\includegraphics{/Users/whichway/workspace/E_ESSEX/CE265/MD/CE265.assets/image-20210329143600411.png}
\caption{}
\end{figure}

\begin{itemize}
\item
  characteristics of unreliable channel will determine complexity of
  reliable data transfer protocol (rdt)
\end{itemize}

3.25

Reliable data transfer: getting started

send side

\textbf{rdt\_send():} called from above, (e.g., by app.). Passed data to
deliver to receiver upper layer

\textbf{udt\_send():} called by rdt, to transfer packet over unreliable
channel to receiver

\begin{figure}
\centering
\includegraphics{/Users/whichway/workspace/E_ESSEX/CE265/MD/CE265.assets/image-20210329144147741.png}
\caption{}
\end{figure}

receive side

\textbf{deliver\_data():} called by \textbf{rdt} to deliver data to
upper

\textbf{rdt\_rcv():} called when packet arrives on rcv-side of channel

3.26

Reliable data transfer: getting started

we'll:

\begin{itemize}
\item
  incrementally develop sender, receiver sides of reliable data transfer
  protocol (rdt)
\item
  consider only unidirectional data transfer
\item
  \begin{itemize}
  \item
    but control info will flow on both directions!
  \end{itemize}
\item
  use finite state machines (FSM) to specify sender, receiver
\end{itemize}

state: when in this ``state'' next state uniquely determined by next
event

\begin{figure}
\centering
\includegraphics{/Users/whichway/workspace/E_ESSEX/CE265/MD/CE265.assets/image-20210329144336666.png}
\caption{}
\end{figure}

3.27

rdt1.0: reliable transfer over a reliable channel

\begin{itemize}
\item
  underlying channel perfectly reliable
\item
  \begin{itemize}
  \item
    no bit errors
  \item
    no loss of packets
  \end{itemize}
\item
  separate FSMs for sender, receiver:
\item
  \begin{itemize}
  \item
    sender sends data into underlying channel
  \item
    receiver reads data from underlying channel
  \end{itemize}
\end{itemize}

\begin{figure}
\centering
\includegraphics{/Users/whichway/workspace/E_ESSEX/CE265/MD/CE265.assets/image-20210329144556542.png}
\caption{}
\end{figure}

3.28

rdt2.0: channel with bit errors

\begin{itemize}
\item
  underlying channel may flip bits in packet
\item
  \begin{itemize}
  \item
    checksum to detect bit errors
  \end{itemize}
\item
  \emph{the} question: how to recover from errors:
\end{itemize}

\textbf{\emph{How do humans recover from ``errors''} during
conversation?}

3.29

rdt2.0: channel with bit errors

\begin{itemize}
\item
  underlying channel may flip bits in packet
\item
  \begin{itemize}
  \item
    checksum to detect bit errors
  \end{itemize}
\item
  \emph{the} question: how to recover from errors:
\item
  \begin{itemize}
  \item
    \emph{acknowledgements (ACKs):} receiver explicitly tells sender
    that pkt received OK
  \item
    \emph{negative acknowledgements (NAKs):} receiver explicitly tells
    sender that pkt had errors
  \item
    sender retransmits pkt on receipt of NAK
  \end{itemize}
\item
  new mechanisms in \textbf{rdt2.0} (beyond \textbf{rdt1.0}):
\item
  \begin{itemize}
  \item
    error detection
  \item
    feedback: control msgs (ACK,NAK) from receiver to sender
  \end{itemize}
\end{itemize}

3.30

rdt2.0: FSM specification

\begin{figure}
\centering
\includegraphics{/Users/whichway/workspace/E_ESSEX/CE265/MD/CE265.assets/image-20210329151945438.png}
\caption{}
\end{figure}

3.31

rdt2.0: operation with no errors

\begin{figure}
\centering
\includegraphics{/Users/whichway/workspace/E_ESSEX/CE265/MD/CE265.assets/image-20210329152340735.png}
\caption{}
\end{figure}

3.32

rdt2.0: error scenario

\begin{figure}
\centering
\includegraphics{/Users/whichway/workspace/E_ESSEX/CE265/MD/CE265.assets/image-20210329153836051.png}
\caption{}
\end{figure}

3.33

rdt2.0 has a fatal flaw!

what happens if ACK/NAK corrupted?

\begin{itemize}
\item
  sender doesn't know what happened at receiver!
\item
  can't just retransmit: possible duplicate
\end{itemize}

handling duplicates:

\begin{itemize}
\item
  sender retransmits current pkt if ACK/NAK corrupted
\item
  sender adds \emph{sequence number} to each pkt
\item
  receiver discards (doesn't deliver up) duplicate pkt
\end{itemize}

stop and wait

\begin{itemize}
\item
  sender sends one packet, then waits for receiver response
\end{itemize}

3.34

rdt2.1: sender, handles garbled ACK/NAKs

\begin{figure}
\centering
\includegraphics{/Users/whichway/workspace/E_ESSEX/CE265/MD/CE265.assets/image-20210329155004114.png}
\caption{}
\end{figure}

3.35

rdt2.1: receiver, handles garbled ACK/NAKs

\begin{figure}
\centering
\includegraphics{/Users/whichway/workspace/E_ESSEX/CE265/MD/CE265.assets/image-20210330080349636.png}
\caption{}
\end{figure}

3.36

rdt2.1: discussion

sender:

\begin{itemize}
\item
  seq \# added to pkt
\item
  two seq. \#'s (0,1) will suffice. Why?
\item
  must check if received ACK/NAK corrupted
\item
  twice as many states
\item
  \begin{itemize}
  \item
    state must ``remember'' whether ``expected'' pkt should have seq \#
    of 0 or 1
  \end{itemize}
\end{itemize}

receiver:

\begin{itemize}
\item
  must check if received packet is duplicate
\item
  \begin{itemize}
  \item
    state indicates whether 0 or 1 is expected pkt seq \#
  \end{itemize}
\item
  note: receiver can \emph{not} know if its last ACK/NAK received OK at
  sender
\end{itemize}

3.37

rdt2.2: a NAK-free protocol

\begin{itemize}
\item
  same functionality as rdt2.1, using ACKs only
\item
  instead of NAK, receiver sends ACK for last pkt received OK
\item
  \begin{itemize}
  \item
    receiver must \emph{explicitly} include seq \# of pkt being ACKed
  \end{itemize}
\item
  duplicate ACK at sender results in same action as NAK:
  \emph{retransmit current pkt}
\end{itemize}

3.38

rdt2.2: sender, receiver fragments

\begin{figure}
\centering
\includegraphics{/Users/whichway/workspace/E_ESSEX/CE265/MD/CE265.assets/image-20210330080531991.png}
\caption{}
\end{figure}

3.39

rdt3.0: channels with errors \emph{and} loss

new assumption: underlying channel can also lose packets (data, ACKs)

\begin{itemize}
\item
  \begin{itemize}
  \item
    checksum, seq. \#, ACKs, retransmissions will be of help \ldots{}
    but not enough
  \end{itemize}
\end{itemize}

approach: sender waits ``reasonable'' amount of time for ACK

\begin{itemize}
\item
  retransmits if no ACK received in this time
\item
  if pkt (or ACK) just delayed (not lost):
\item
  \begin{itemize}
  \item
    retransmission will be duplicate, but seq. \#'s already handles this
  \item
    receiver must specify seq \# of pkt being ACKed
  \end{itemize}
\item
  requires countdown timer
\end{itemize}

3.40

rdt3.0 sender

\begin{figure}
\centering
\includegraphics{/Users/whichway/workspace/E_ESSEX/CE265/MD/CE265.assets/image-20210330080646078.png}
\caption{}
\end{figure}

3.41

rdt3.0 in action

\begin{figure}
\centering
\includegraphics{/Users/whichway/workspace/E_ESSEX/CE265/MD/CE265.assets/image-20210330080718567.png}
\caption{}
\end{figure}

\begin{figure}
\centering
\includegraphics{/Users/whichway/workspace/E_ESSEX/CE265/MD/CE265.assets/image-20210330080728338.png}
\caption{}
\end{figure}

3.42

rdt3.0 in action

\begin{figure}
\centering
\includegraphics{/Users/whichway/workspace/E_ESSEX/CE265/MD/CE265.assets/image-20210330080810057.png}
\caption{}
\end{figure}

\begin{figure}
\centering
\includegraphics{/Users/whichway/workspace/E_ESSEX/CE265/MD/CE265.assets/image-20210330080824721.png}
\caption{}
\end{figure}

3.43

Performance of rdt3.0

\begin{itemize}
\item
  rdt3.0 is correct, but performance stinks
\item
  e.g.: 1 Gbps link, 15 ms prop. delay, 8000 bit packet:
\end{itemize}

\begin{figure}
\centering
\includegraphics{/Users/whichway/workspace/E_ESSEX/CE265/MD/CE265.assets/image-20210406084106023.png}
\caption{}
\end{figure}

\begin{itemize}
\item
  U sender: \emph{utilization} -- fraction of time sender busy sending
\end{itemize}

\begin{figure}
\centering
\includegraphics{/Users/whichway/workspace/E_ESSEX/CE265/MD/CE265.assets/image-20210406084159831.png}
\caption{}
\end{figure}

\begin{itemize}
\item
  \begin{itemize}
  \item
    if RTT=30 msec, 1KB pkt every 30 msec: 267Kb/sec thruput over 1 Gbps
    link
  \end{itemize}
\item
  network protocol limits use of physical resources!
\end{itemize}

3.44

rdt3.0: stop-and-wait operation

\begin{figure}
\centering
\includegraphics{/Users/whichway/workspace/E_ESSEX/CE265/MD/CE265.assets/image-20210406084634167.png}
\caption{}
\end{figure}

first packet bit transmitted, t = 0

last packet bit transmitted, t = L / R

first packet bit arrives

last packet bit arrives, send ACK

ACK arrives, send next packet, t = RTT + L / R

\begin{figure}
\centering
\includegraphics{/Users/whichway/workspace/E_ESSEX/CE265/MD/CE265.assets/image-20210406084418391.png}
\caption{}
\end{figure}

3.45

Pipelined protocols

pipelining: sender allows multiple, ``in-flight'',
yet-to-be-acknowledged pkts

\begin{itemize}
\item
  \begin{itemize}
  \item
    range of sequence numbers must be increased
  \item
    buffering at sender and/or receiver
  \end{itemize}
\end{itemize}

\begin{figure}
\centering
\includegraphics{/Users/whichway/workspace/E_ESSEX/CE265/MD/CE265.assets/image-20210412135559908.png}
\caption{}
\end{figure}

\begin{itemize}
\item
  two generic forms of pipelined protocols: \emph{go-Back-N, selective
  repeat}
\end{itemize}

3.46

Pipelining: increased utilization

\begin{figure}
\centering
\includegraphics{/Users/whichway/workspace/E_ESSEX/CE265/MD/CE265.assets/image-20210412140823377.png}
\caption{}
\end{figure}

3-packet pipelining increases utilization by a factor of 3!

\[U_{sender} = \frac{\frac{3L}{R}}{RTT+\frac{L}{R}}=\frac{0.0024}{30.008} = 0.00081\]

3.47

Pipelined protocols

pipelining: sender allows multiple, ``in-flight'',
yet-to-be-acknowledged pkts

\begin{itemize}
\item
  \begin{itemize}
  \item
    range of sequence numbers must be increased
  \item
    buffering at sender and/or receiver
  \end{itemize}
\end{itemize}

\begin{figure}
\centering
\includegraphics{/Users/whichway/workspace/E_ESSEX/CE265/MD/CE265.assets/image-20210412135559908.png}
\caption{}
\end{figure}

\begin{itemize}
\item
  two generic forms of pipelined protocols: \emph{go-Back-N, selective
  repeat}
\end{itemize}

3.48

Pipelined protocols: overview

Go-back-N:

\begin{itemize}
\item
  sender can have up to N unacked packets in pipeline
\item
  receiver only sends \emph{cumulative ack}
\item
  \begin{itemize}
  \item
    doesn't ack packet if there's a gap
  \end{itemize}
\item
  sender has timer for oldest unacked packet
\item
  \begin{itemize}
  \item
    when timer expires, retransmit \emph{all} unacked packets
  \end{itemize}
\end{itemize}

Selective Repeat:

\begin{itemize}
\item
  sender can have up to N unack'ed packets in pipeline
\item
  rcvr sends \emph{individual ack} for each packet
\item
  sender maintains timer for each unacked packet
\item
  \begin{itemize}
  \item
    when timer expires, retransmit only that unacked packet
  \end{itemize}
\end{itemize}

3.49

Why is window up to N?

Go-Back-N: sender

\begin{itemize}
\item
  k-bit seq \# in pkt header
\item
  ``window'' of up to N, consecutive unack'ed pkts allowed
\end{itemize}

\begin{figure}
\centering
\includegraphics{/Users/whichway/workspace/E_ESSEX/CE265/MD/CE265.assets/image-20210412141227983.png}
\caption{}
\end{figure}

\begin{itemize}
\item
  ACK(n): ACKs all pkts up to, including seq \# n - \emph{``cumulative
  ACK''}
\item
  \begin{itemize}
  \item
    may receive duplicate ACKs (see receiver)
  \end{itemize}
\item
  timer for oldest in-flight pkt
\item
  \emph{timeout(n):} retransmit packet n and all higher seq \# pkts in
  window
\end{itemize}

3.50 GBN: sender extended FSM

\begin{quote}
有限状态机,上层是状态,下层是动作 SR 分组长度的大小和动作
\end{quote}

\begin{figure}
\centering
\includegraphics{/Users/whichway/workspace/E_ESSEX/CE265/MD/CE265.assets/image-20210412141457461.png}
\caption{}
\end{figure}

3.51 GBN: receiver extended FSM

\begin{figure}
\centering
\includegraphics{/Users/whichway/workspace/E_ESSEX/CE265/MD/CE265.assets/image-20210412141516789.png}
\caption{}
\end{figure}

Why not buffering?

ACK-only: always send ACK for correctly-received pkt with highest
\emph{in-order} seq \#

\begin{itemize}
\item
  \begin{itemize}
  \item
    may generate duplicate ACKs
  \item
    need only remember \textbf{expectedseqnum}
  \end{itemize}
\item
  out-of-order pkt:
\item
  \begin{itemize}
  \item
    discard (don't buffer): \emph{no receiver buffering!}
  \item
    re-ACK pkt with highest in-order seq \#
  \end{itemize}
\end{itemize}

3.52

GBN in action

\begin{figure}
\centering
\includegraphics{/Users/whichway/workspace/E_ESSEX/CE265/MD/CE265.assets/image-20210412141602041.png}
\caption{}
\end{figure}

\hypertarget{35-connection-oriented-transport-tcp-segment-structure}{%
\subsection{3.5 connection-oriented transport: TCP segment
structure}\label{35-connection-oriented-transport-tcp-segment-structure}}

\begin{itemize}
\item
  \begin{enumerate}
  \def\labelenumi{\arabic{enumi}.}
  \item
    segment structure
  \item
    reliable data transfer
  \item
    flow control
  \item
    connection management
  \end{enumerate}
\end{itemize}

3.65

TCP: Overview RFCs: 793,1122,1323, 2018, 2581

\begin{itemize}
\item
  point-to-point:
\item
  \begin{itemize}
  \item
    one sender, one receiver
  \end{itemize}
\item
  reliable, in-order \emph{byte steam:}
\item
  \begin{itemize}
  \item
    no ``message boundaries''
  \end{itemize}
\item
  pipelined:
\item
  \begin{itemize}
  \item
    TCP congestion and flow control set window size
  \end{itemize}
\item
  full duplex data:
\item
  \begin{itemize}
  \item
    bi-directional data flow in same connection
  \item
    MSS: maximum segment size
  \end{itemize}
\item
  connection-oriented:
\item
  \begin{itemize}
  \item
    handshaking (exchange of control msgs) inits sender, receiver state
    before data exchange

    \begin{quote}
    TCP被称为是面向连接的(conn ection-oriented), 这是因为在一个
    应用进程 可以开始 向另 一 个应用 进 程发送数据之前,这两
    个进程必须先相互 ``握手'',即它们必须相互发送
    某些预备报文段,以建立确保数据传输的参数。作为TCP连接建立的一部分,连接的双方
    都将初始化 与TCP连接相关的许多TCP状态变量
    \end{quote}
  \end{itemize}
\item
  flow controlled:
\item
  \begin{itemize}
  \item
    sender will not overwhelm receiver
  \end{itemize}
\end{itemize}

\begin{quote}
TCP是
因特网运输层的面向连接的可靠的运输协议。我们在本节中将看到,为了提供可靠数据传
输,T C
P依赖于前一节所讨论的许多基本原理,其中包括差错检测、重传、累积确认、定
时器以及用于序号和确认号的首部字段。TCP定义在RFC 793、RFC 1122、RFC
1323、 RFC 2018 以及 RFC 2581 中。
\end{quote}

3.66

\hl{TCP segment structure}

\begin{figure}
\centering
\includegraphics{/Users/whichway/workspace/E_ESSEX/CE265/MD/CE265.assets/image-20210330081350540.png}
\caption{}
\end{figure}

3.67

TCP seq. numbers, ACKs

sequence numbers:

\begin{itemize}
\item
  \begin{itemize}
  \item
    byte stream ``number'' of first byte in segment's data
  \end{itemize}
\end{itemize}

acknowledgements:

\begin{itemize}
\item
  \begin{itemize}
  \item
    seq \# of next byte expected from other side
  \item
    cumulative ACK
  \end{itemize}
\end{itemize}

Q: how receiver handles out-of-order segments

\begin{itemize}
\item
  \begin{itemize}
  \item
    A: TCP spec doesn't say, - up to implementor
  \end{itemize}
\end{itemize}

\begin{figure}
\centering
\includegraphics{/Users/whichway/workspace/E_ESSEX/CE265/MD/CE265.assets/image-20210330081724253.png}
\caption{}
\end{figure}

3.68

TCP seq. numbers, ACKs

\begin{figure}
\centering
\includegraphics{/Users/whichway/workspace/E_ESSEX/CE265/MD/CE265.assets/image-20210330081931407.png}
\caption{}
\end{figure}

3.69

\hl{TCP round trip time, timeout}

\hl{Q: how to set TCP timeout value?}

\begin{enumerate}
\def\labelenumi{\arabic{enumi}.}
\item
  longer than RTT
\item
  \begin{itemize}
  \item
    but RTT varies
  \end{itemize}
\item
  \emph{too short:} premature timeout, unnecessary retransmissions
\item
  \emph{too long:} slow reaction to segment loss
\end{enumerate}

\hl{Q: how to estimate RTT?}

\begin{enumerate}
\def\labelenumi{\arabic{enumi}.}
\item
  \textbf{SampleRTT}: measured time from segment transmission until ACK
  receipt
\item
  \begin{itemize}
  \item
    ignore retransmissions
  \end{itemize}
\item
  \textbf{SampleRTT} will vary, want estimated RTT ``smoother''
\item
  \begin{itemize}
  \item
    average several \emph{recent} measurements, not just current
    \textbf{SampleRTT}
  \end{itemize}
\end{enumerate}

3.70

\hl{TCP round trip time, timeout}

\[EstimatedRTT = (1- α)*EstimatedRTT + α*SampleRTT\]

\begin{itemize}
\item
  {[}平均值{]}exponential weighted moving a\hl{verage}
\item
  influence of past sample decreases exponentially fast
\item
  typical value: α \textbf{=} 0.125
\end{itemize}

\begin{figure}
\centering
\includegraphics{/Users/whichway/workspace/E_ESSEX/CE265/MD/CE265.assets/image-20210330093040209.png}
\caption{}
\end{figure}

3.71

TCP round trip time, timeout

\begin{itemize}
\item
  timeout interval: \textbf{EstimatedRTT} plus ``safety margin''
\item
  \begin{itemize}
  \item
    large variation in \textbf{EstimatedRTT -\textgreater{}} larger
    safety margin
  \end{itemize}
\item
  estimate SampleRTT deviation from EstimatedRTT:
\end{itemize}

\[DevRTT = (1-β)*DevRTT +
             β*|SampleRTT-EstimatedRTT|\]

\textbf{(typically,} β \textbf{= 0.25)}

\[TimeoutInterval = EstimatedRTT + 4*DevRTT\]

\begin{figure}
\centering
\includegraphics{/Users/whichway/workspace/E_ESSEX/CE265/MD/CE265.assets/image-20210330093144967.png}
\caption{}
\end{figure}

\hypertarget{35-connection-oriented-transport-tcp-reliable-data-transfer}{%
\subsection{3.5 connection-oriented transport: TCP reliable data
transfer}\label{35-connection-oriented-transport-tcp-reliable-data-transfer}}

3.73

TCP reliable data transfer

\begin{itemize}
\item
  TCP creates rdt service on top of IP's unreliable service
\item
  \begin{itemize}
  \item
    pipelined segments
  \item
    cumulative acks
  \item
    single retransmission timer
  \end{itemize}
\item
  retransmissions triggered by:
\item
  \begin{itemize}
  \item
    timeout events
  \item
    duplicate acks
  \end{itemize}
\item
  let's initially consider simplified TCP sender:
\item
  \begin{itemize}
  \item
    ignore duplicate acks
  \item
    ignore flow control, congestion control
  \end{itemize}
\end{itemize}

3.74

\hl{TCP sender events:}

\emph{data rcvd from app:}

\begin{itemize}
\item
  create segment with seq
\item
  seq is byte-stream number of first data byte in segment
\item
  start timer if not already running
\item
  \begin{itemize}
  \item
    think of timer as for oldest unacked segment
  \item
    expiration interval: \textbf{TimeOutInterval}
  \end{itemize}
\end{itemize}

\emph{timeout:}

\begin{enumerate}
\def\labelenumi{\arabic{enumi}.}
\item
  retransmit segment that caused timeout
\item
  restart timer
\end{enumerate}

\emph{ack rcvd:}

\begin{itemize}
\item
  if ack acknowledges previously unacked segments
\item
  \begin{itemize}
  \item
    update what is known to be ACKed
  \item
    start timer if there are still unacked segments
  \end{itemize}
\end{itemize}

3.75

\hl{TCP sender (simplified)}

\begin{figure}
\centering
\includegraphics{/Users/whichway/workspace/E_ESSEX/CE265/MD/CE265.assets/image-20210330093637889.png}
\caption{}
\end{figure}

3.76

TCP: retransmission scenarios

lost ACK scenario

\begin{figure}
\centering
\includegraphics{/Users/whichway/workspace/E_ESSEX/CE265/MD/CE265.assets/image-20210330094040988.png}
\caption{}
\end{figure}

premature{[}过早地{]} timeout

\begin{figure}
\centering
\includegraphics{/Users/whichway/workspace/E_ESSEX/CE265/MD/CE265.assets/image-20210330094103520.png}
\caption{}
\end{figure}

cumulative ACK

\begin{figure}
\centering
\includegraphics{/Users/whichway/workspace/E_ESSEX/CE265/MD/CE265.assets/image-20210330094300490.png}
\caption{}
\end{figure}

3.78

\hl{TCP ACK generation {[}RFC 1122, RFC 2581{]}}

\begin{longtable}[]{@{}ll@{}}
\toprule
\emph{event at receiver} & \emph{TCP receiver action}\tabularnewline
\midrule
\endhead
1.1 arrival of in-order segment with with expected seq \# & 1.1 delayed
ACK. Wait up to 500 msdelayed ACK.\tabularnewline
1.2 All data up toexpected seq \#, already ACKed & 1.2 If no next
segment,for next segment.\tabularnewline
2.1 arrival of in-order segment with expected seq \# & 2.1 immediately
send single cumulative ACK\tabularnewline
2.2 One other segment has ACK pending & 2.2 ACKing both in-order
segments\tabularnewline
3 arrival of out-of-order segment higher-than-expect seq \# Gap detected
& 3. immediately send \hl{duplicate ACK,} indicating seq. \# of next
expected byte\tabularnewline
4. arrival of segment that partially or completely fills gap & 4.
immediate send ACK, provided that segment starts at lower end of
gap\tabularnewline
\bottomrule
\end{longtable}

3.79

\hl{TCP fast retransmit{[}快速🔜重传, 三次重复的ACK(fast retransmit after
sender receipt of triple duplicate ACK){]}}

\begin{itemize}
\item
  time-out period often relatively long:
\item
  \begin{itemize}
  \item
    long delay before resending lost packet
  \end{itemize}
\item
  detect lost segments via duplicate ACKs.
\item
  \begin{itemize}
  \item
    sender often sends many segments back-to-back
  \item
    if segment is lost, there will likely be many duplicate ACKs.
  \end{itemize}
\end{itemize}

if sender receives 3 ACKs for same data (``triple duplicate ACKs''),
resend unacked segment with smallest seq likely that unacked segment
lost, so don't wait for timeout

3.80

TCP fast retransmit

fast retransmit after sender receipt of triple duplicate ACK

\hypertarget{35-tcp-flow-controlux6d41ux91cfux63a7ux5236}{%
\subsection{3.5 TCP: flow
control{[}流量控制{]}}\label{35-tcp-flow-controlux6d41ux91cfux63a7ux5236}}

\begin{quote}
一种是接收方接受不过来的拥塞控制 还有🔟网络不行的流量控制
\end{quote}

3.82 TCP: flow control

\begin{figure}
\centering
\includegraphics{/Users/whichway/workspace/E_ESSEX/CE265/MD/CE265.assets/image-20210413082716221.png}
\caption{}
\end{figure}

3.83 TCP: flow control

\begin{enumerate}
\def\labelenumi{\arabic{enumi}.}
\item
  receiver "advertises" free buffer space by including \(rwnd\) value in
  TCP header of reciver-to-sender segments

  \begin{enumerate}
  \def\labelenumii{\arabic{enumii}.}
  \item
    1 \(RcvBuffer\) size set via socket option ( typical default is 4096
    bytes )
  \item
    2 many operating systems autoadjust \(RcvBuffer\)
  \end{enumerate}
\item
  Sender Limits amount of unacked ( "in-flight" ) data to reciver
  \(rwnd value\)
\item
  guarantees receive buffer will not overflow
\end{enumerate}

\begin{figure}
\centering
\includegraphics{/Users/whichway/workspace/E_ESSEX/CE265/MD/CE265.assets/image-20210413083128032.png}
\caption{}
\end{figure}

\hypertarget{35-tcp-connection-managementux94feux63a5ux7ba1ux7406}{%
\subsection{3.5 TCP: connection
management{[}链接管理{]}}\label{35-tcp-connection-managementux94feux63a5ux7ba1ux7406}}

3.85

\hl{Connection Management}

before exhanging data, sender/receiver "handshake"

\begin{enumerate}
\def\labelenumi{\arabic{enumi}.}
\item
  agree to establish ocnnection ( each knowing the other willing to
  establish connection )
\item
  agree on connection parameters
\end{enumerate}

\begin{figure}
\centering
\includegraphics{/Users/whichway/workspace/E_ESSEX/CE265/MD/CE265.assets/image-20210413084602148.png}
\caption{}
\end{figure}

3.86

Complexity of Protocol

\begin{figure}
\centering
\includegraphics{/Users/whichway/workspace/E_ESSEX/CE265/MD/CE265.assets/image-20210413084654479.png}
\caption{}
\end{figure}

3.87

Agreeing to establish a connection

2-way handshake

\begin{figure}
\centering
\includegraphics{/Users/whichway/workspace/E_ESSEX/CE265/MD/CE265.assets/image-20210413084915452.png}
\caption{}
\end{figure}

\hl{Q: Will 2-way handshake always work in network?}

\begin{enumerate}
\def\labelenumi{\arabic{enumi}.}
\item
  {[} variable delays{]} variable delays{[}变化的延迟{]}
\item
  {[} \hl{retransmitted message loss} {]} retransmitted messages ( e.g.
  \(req-conn(x) \) ) due to message loss
\item
  {[} mesage reordering {]} message reordering{[}重新安排{]}
\item
  {[} cannot see {]} cannot "see" other sider
\end{enumerate}

3.88

2-way handshake failure scenarios:

\begin{enumerate}
\def\labelenumi{\arabic{enumi}.}
\item
  half open connection! (no client!)
\end{enumerate}

\begin{quote}
提前判断超时, 重发 \(req\_conn(x)\) 导致 server 没发 \(acc\_conn(x)\)
\end{quote}

\begin{figure}
\centering
\includegraphics{/Users/whichway/workspace/E_ESSEX/CE265/MD/CE265.assets/image-20210413085407538.png}
\caption{}
\end{figure}

\begin{quote}
提前判断超时, 重发的 \(req\_conn(x)\) 导致 server 没发 \(acc\_conn(x)\)
\end{quote}

\begin{figure}
\centering
\includegraphics{/Users/whichway/workspace/E_ESSEX/CE265/MD/CE265.assets/image-20210413085752637.png}
\caption{}
\end{figure}

3.89

\hypertarget{tcp-3-way-handshake}{%
\subsubsection{\texorpdfstring{\hl{⭐️TCP 3-way
Handshake}}{⭐️TCP 3-way Handshake}}\label{tcp-3-way-handshake}}

\begin{figure}
\centering
\includegraphics{/Users/whichway/workspace/E_ESSEX/CE265/MD/CE265.assets/image-20210413091842901.png}
\caption{}
\end{figure}

⭐️

\begin{figure}
\centering
\includegraphics{/Users/whichway/workspace/E_ESSEX/CE265/MD/CE265.assets/image-20210413091920532.png}
\caption{}
\end{figure}

\begin{figure}
\centering
\includegraphics{/Users/whichway/workspace/E_ESSEX/CE265/MD/CE265.assets/image-20210413091928233.png}
\caption{}
\end{figure}

3.90

\hypertarget{tcp-3-way-handshake-fsm--finite-state-machine}{%
\subsubsection{\texorpdfstring{\hl{⭐️⭐️TCP 3-way Handshake: FSM ( Finite
State
Machine)}}{⭐️⭐️TCP 3-way Handshake: FSM ( Finite State Machine)}}\label{tcp-3-way-handshake-fsm--finite-state-machine}}

\begin{figure}
\centering
\includegraphics{/Users/whichway/workspace/E_ESSEX/CE265/MD/CE265.assets/image-20210413092030800.png}
\caption{}
\end{figure}

\begin{figure}
\centering
\includegraphics{/Users/whichway/workspace/E_ESSEX/CE265/MD/CE265.assets/image-20210413092043577.png}
\caption{}
\end{figure}

3.91 \#\#\# \hl{⭐️ TCP: closing a connection}

\begin{quote}
4次挥手🙋拆除链接🔗
\end{quote}

\begin{enumerate}
\def\labelenumi{\arabic{enumi}.}
\item
  Client, server each close their side of connection

  \begin{enumerate}
  \def\labelenumii{\arabic{enumii}.}
  \item
    send TCP segment with FIN bit = 1
  \end{enumerate}
\item
  respond to received FIN with ACK

  \begin{enumerate}
  \def\labelenumii{\arabic{enumii}.}
  \item
    on receiving FIN, ACK can be combined with own FIN
  \end{enumerate}
\item
  simultaneous FIN exchanges can be handled
\end{enumerate}

3.92

\begin{figure}
\centering
\includegraphics{/Users/whichway/workspace/E_ESSEX/CE265/MD/CE265.assets/image-20210413092647558.png}
\caption{}
\end{figure}

\begin{figure}
\centering
\includegraphics{/Users/whichway/workspace/E_ESSEX/CE265/MD/CE265.assets/image-20210413092734482.png}
\caption{}
\end{figure}

\begin{figure}
\centering
\includegraphics{/Users/whichway/workspace/E_ESSEX/CE265/MD/CE265.assets/image-20210413092750629.png}
\caption{}
\end{figure}

\hypertarget{36-principles-of-congestion-control}{%
\subsection{\texorpdfstring{3.6 Principles of congestion control
}{3.6 Principles of congestion control }}\label{36-principles-of-congestion-control}}

\begin{quote}
三种情况,四个代价,

四个代价

\begin{enumerate}
\def\labelenumi{\arabic{enumi}.}
\item
\item
  可能会丢包
\item
  可能会引起不必要的重传
\item
  多跳传输,所有上游所作的会作废
\end{enumerate}

三种情况

1.

\begin{enumerate}
\def\labelenumi{\arabic{enumi}.}
\item
  吞吐量严重下降
\item
\end{enumerate}

TCP

\begin{quote}
慢启动 + 拥塞避免 + 快速恢复
\end{quote}

ATM 网络控制
\end{quote}

3.93 Congestion:

\begin{quote}
有太多资源在网络上传输难以被控制 与流量控制不同 表现 manifestations

\begin{enumerate}
\def\labelenumi{\arabic{enumi}.}
\item
  丢包
\item
  长时间
\end{enumerate}
\end{quote}

\begin{enumerate}
\def\labelenumi{\arabic{enumi}.}
\item
  informally: ``too many sources sending too much data too fast for
  network to handle''
\item
  different from flow control!
\end{enumerate}

\begin{enumerate}
\def\labelenumi{\arabic{enumi}.}
\item
  manifestations{[}表现{]}:

  \begin{enumerate}
  \def\labelenumii{\arabic{enumii}.}
  \item
    1 lost packets (buffer overflow at routers)
  \item
    2 long delays (queueing in router buffers)
  \end{enumerate}
\end{enumerate}

\begin{enumerate}
\def\labelenumi{\arabic{enumi}.}
\item
  a top-10 problem!
\end{enumerate}

3.94

Principles of congestion control

\emph{congestion}:

\begin{itemize}
\item
  informally: ``too many sources sending too much data too fast for
  \emph{network} to handle''
\item
  different from flow control!
\item
  manifestations:
\item
  \begin{itemize}
  \item
    lost packets (buffer overflow at routers)
  \item
    long delays (queueing in router buffers)
  \end{itemize}
\item
  a top-10 problem!
\end{itemize}

3.95

Causes/costs of congestion: scenario 1

\begin{itemize}
\item
  two senders, two receivers
\item
  one router, infinite buffers
\item
  output link capacity: R
\item
  no retransmission
\end{itemize}

\begin{figure}
\centering
\includegraphics{/Users/whichway/workspace/E_ESSEX/CE265/MD/CE265.assets/image-20210419142432800.png}
\caption{}
\end{figure}

\begin{figure}
\centering
\includegraphics{/Users/whichway/workspace/E_ESSEX/CE265/MD/CE265.assets/image-20210419142444926.png}
\caption{}
\end{figure}

3.96

Causes/costs of congestion: scenario 2

\begin{itemize}
\item
  one router, \emph{finite} buffers
\item
  sender retransmission of timed-out packet
\item
  \begin{itemize}
  \item
    application-layer input = application-layer output: λin = λout
  \item
    transport-layer input includes \emph{retransmissions} : λin λin
  \end{itemize}
\end{itemize}

\begin{figure}
\centering
\includegraphics{/Users/whichway/workspace/E_ESSEX/CE265/MD/CE265.assets/image-20210419142536103.png}
\caption{}
\end{figure}

3.97

Causes/costs of congestion: scenario 2

idealization: perfect knowledge sender sends only when router buffers
available

\begin{figure}
\centering
\includegraphics{/Users/whichway/workspace/E_ESSEX/CE265/MD/CE265.assets/image-20210419142622693.png}
\caption{}
\end{figure}

\begin{figure}
\centering
\includegraphics{/Users/whichway/workspace/E_ESSEX/CE265/MD/CE265.assets/image-20210419142630439.png}
\caption{}
\end{figure}

3.98

Causes/costs of congestion: scenario 2

\emph{Idealization:} \emph{known loss} packets can be lost, dropped at
router due to full buffers

\begin{itemize}
\item
  sender only resends if packet \emph{known} to be lost
\end{itemize}

\begin{figure}
\centering
\includegraphics{/Users/whichway/workspace/E_ESSEX/CE265/MD/CE265.assets/image-20210419142717678.png}
\caption{}
\end{figure}

3.99

Causes/costs of congestion: scenario 2

\emph{Idealization:} \emph{known loss} packets can be lost, dropped at
router due to full buffers

\begin{itemize}
\item
  sender only resends if packet \emph{known} to be lost
\end{itemize}

when sending at R/2, some packets are retransmissions but asymptotic
goodput is still R/2 (why?)

\begin{figure}
\centering
\includegraphics{/Users/whichway/workspace/E_ESSEX/CE265/MD/CE265.assets/image-20210419142802328.png}
\caption{}
\end{figure}

\begin{figure}
\centering
\includegraphics{/Users/whichway/workspace/E_ESSEX/CE265/MD/CE265.assets/image-20210419142810477.png}
\caption{}
\end{figure}

3.100

Causes/costs of congestion: scenario 2

\emph{Realistic:} \emph{duplicates}

\begin{itemize}
\item
  packets can be lost, dropped at router due to full buffers
\item
  sender times out prematurely, sending \emph{two} copies, both of which
  are delivered
\end{itemize}

when sending at R/2, some packets are retransmissions including
duplicated that are delivered!

\begin{figure}
\centering
\includegraphics{/Users/whichway/workspace/E_ESSEX/CE265/MD/CE265.assets/image-20210419143014517.png}
\caption{}
\end{figure}

\begin{figure}
\centering
\includegraphics{/Users/whichway/workspace/E_ESSEX/CE265/MD/CE265.assets/image-20210419143032441.png}
\caption{}
\end{figure}

3.101

Causes/costs of congestion: scenario 2

when sending at R/2, some packets are retransmissions including
duplicated that are delivered!

\begin{figure}
\centering
\includegraphics{/Users/whichway/workspace/E_ESSEX/CE265/MD/CE265.assets/image-20210419143136103.png}
\caption{}
\end{figure}

\emph{Realistic:} \emph{duplicates}

\begin{itemize}
\item
  packets can be lost, dropped at router due to full buffers
\item
  sender times out prematurely, sending \emph{two} copies, both of which
  are delivered
\end{itemize}

``costs'' of congestion:

\begin{itemize}
\item
  more work (retrans) for given ``goodput''
\item
  unneeded retransmissions: link carries multiple copies of pkt
\item
  \begin{itemize}
  \item
    decreasing goodput
  \end{itemize}
\end{itemize}

3.102

Causes/costs of congestion: scenario 3

\begin{itemize}
\item
  four senders
\item
  multihop paths
\item
  timeout/retransmit
\end{itemize}

\hl{Q: what happens as λin and λin\textbf{'} increase ?}

\hl{A: as red λin' increases, all arriving blue pkts at upper queue are
dropped, blue throughput is 0.}

\begin{figure}
\centering
\includegraphics{/Users/whichway/workspace/E_ESSEX/CE265/MD/CE265.assets/image-20210419143344853.png}
\caption{}
\end{figure}

3.103

Causes/costs of congestion: scenario 3

\begin{figure}
\centering
\includegraphics{/Users/whichway/workspace/E_ESSEX/CE265/MD/CE265.assets/image-20210419143409235.png}
\caption{}
\end{figure}

another ``cost'' of congestion:

\begin{itemize}
\item
  when packet dropped, any ``upstream transmission capacity used for
  that packet was wasted!
\end{itemize}

3.104 \hl{Approaches towards congestion control{[}拥塞控制{]}}

two broad approaches towards congestion control:

A. \hl{{[}端对端拥塞控制{]} end-end congestion control}:

\begin{enumerate}
\def\labelenumi{\arabic{enumi}.}
\item
  \hl{{[} no feedback {]}} no explicit feedback from network
\item
  \hl{{[} loss delay {]}} congestion inferred{[}推断{]} from end-system
  observed loss, delay
\item
  \hl{{[} taken by TCP {]}} approach taken by TCP
\end{enumerate}

B. network-assisted congestion control:

\begin{itemize}
\item
  routers provide feedback to end systems
\item
  \begin{itemize}
  \item
    single bit indicating congestion (SNA, DECbit, TCP/IP ECN, ATM)
  \item
    explicit rate for sender to send at
  \end{itemize}
\end{itemize}

3.105

Case study: ATM ABR congestion control

\hl{ABR: available bit rate{[}可变位速率{]}:}

\begin{itemize}
\item
  ``elastic service''
\item
  if sender's path ``underloaded'':
\item
  \begin{itemize}
  \item
    sender should use available bandwidth
  \end{itemize}
\item
  if sender's path congested:
\item
  \begin{itemize}
  \item
    sender throttled to minimum guaranteed rate
  \end{itemize}
\end{itemize}

\hl{RM (resource management) cells:}

\begin{itemize}
\item
  sent by sender, interspersed with data cells
\item
  bits in RM cell set by switches (``\emph{network-assisted''})
\item
  \begin{itemize}
  \item
    \emph{NI bit:} no increase in rate (mild congestion) {[} 不增加速率
    {]}
  \item
    \emph{CI bit:} congestion indication {[} 拥塞发生 {]}
  \end{itemize}
\item
  RM cells returned to sender by receiver, with bits intact {[}
  控制分组位不变, 发回接收端 {]}
\end{itemize}

3.106

Case study: ATM ABR congestion control

\begin{figure}
\centering
\includegraphics{/Users/whichway/workspace/E_ESSEX/CE265/MD/CE265.assets/image-20210419144212249.png}
\caption{}
\end{figure}

\begin{itemize}
\item
  two-byte ER (explicit rate) field in RM cell
\item
  \begin{itemize}
  \item
    congested switch may lower ER value in cell
  \item
    senders' send rate thus max supportable rate on path
  \end{itemize}
\item
  EFCI bit in data cells: set to 1 in congested switch
\item
  \begin{itemize}
  \item
    if data cell preceding RM cell has EFCI set, receiver sets CI bit in
    returned RM cell
  \end{itemize}
\end{itemize}

\hypertarget{37-tcp-congestion-control}{%
\subsection{3.7 TCP congestion
control}\label{37-tcp-congestion-control}}

3.108 end-end control (no network assistance)

\begin{itemize}
\item
  sender limits transmission:
\end{itemize}

\textbf{LastByteSent-LastByteAcked} ≤ \textbf{Cwnd}

Roughly,

\[rate = \frac{cwnd}{RTT} Bytes/sec\]

How does sender perceive congestion?

\begin{enumerate}
\def\labelenumi{\arabic{enumi}.}
\item
  loss event = timeout \emph{or} 3 duplicate acks
\item
  TCP sender reduces rate (\textbf{Cwnd}) after loss event
\item
  \textbf{Cwnd} is dynamic, function of perceived network congestion
\end{enumerate}

three mechanisms:

\begin{enumerate}
\def\labelenumi{\arabic{enumi}.}
\item
  AIMD
\item
  slow start
\item
  conservative after timeout events
\end{enumerate}

3.109

TCP congestion control: additive increase multiplicative decrease

approach: sender increases transmission rate (window size), probing for
usable bandwidth, until loss occurs additive increase: increase cwnd by
1 MSS every RTT until loss detected multiplicative decrease: cut cwnd in
half after loss

AIMD saw tooth behavior: probing for bandwidth

\begin{figure}
\centering
\includegraphics{/Users/whichway/workspace/E_ESSEX/CE265/MD/CE265.assets/image-20210419144819082.png}
\caption{}
\end{figure}

3.110

TCP Congestion Control: details

\begin{figure}
\centering
\includegraphics{/Users/whichway/workspace/E_ESSEX/CE265/MD/CE265.assets/image-20210419150056262.png}
\caption{}
\end{figure}

sender limits transmission

\[LastByteSent - LastByteAcked \leq cwnd[congestion window]\]

cwnd is dynamic, function of perceived network congestion

TCP sending rate: roughly: send cwnd bytes, wait RTT for ACKS, then send
more bytes

\[rate \approx \frac{cwnd}{RTT} bytes/sec\]

3.111 TCP Slow Start

\begin{quote}
TCP 慢启动 当TCP链接启动时,速率指数增长直到第一个丢包 congestion
windows = 1 MSS 翻倍 congestion windows 在每一个 RTT
\end{quote}

\begin{enumerate}
\def\labelenumi{\arabic{enumi}.}
\item
  when connection begins, increase rate exponentially until first loss
  event:
\item
  initially cwnd = 1 MSS
\item
  double cwnd every RTT
\item
  done by incrementing cwnd for every ACK received
\end{enumerate}

summary: initial rate is slow but ramps up exponentially fast

\begin{figure}
\centering
\includegraphics{/Users/whichway/workspace/E_ESSEX/CE265/MD/CE265.assets/image-20210419150359285.png}
\caption{}
\end{figure}

3.112 TCP: detecting, reacting to loss

\begin{quote}
\end{quote}

loss indicated by timeout:

cwnd set to 1 MSS;

window then grows exponentially (as in slow start) to threshold, then
grows linearly

loss indicated by 3 duplicate ACKs: TCP RENO

dup ACKs indicate network capable of delivering some segments

ssthresh is cut in half window

cwnd set to ssthresh+ 3*MSS then grows linearly

TCP Tahoe always sets cwnd to 1 (timeout or 3 duplicate acks)

3.113

TCP: switching from slow start to CA

Q: when should the exponential increase switch to linear?

A: when cwnd gets to 1/2 of its value before timeout.

Implementation:

\begin{enumerate}
\def\labelenumi{\arabic{enumi}.}
\item
  variable ssthresh
\item
  on loss event, ssthresh is set to 1/2 of cwnd just before loss event
\end{enumerate}

\begin{figure}
\centering
\includegraphics{/Users/whichway/workspace/E_ESSEX/CE265/MD/CE265.assets/image-20210419150538081.png}
\caption{}
\end{figure}

3.114

Diffrent states

congestion avoidance

\begin{figure}
\centering
\includegraphics{/Users/whichway/workspace/E_ESSEX/CE265/MD/CE265.assets/image-20210419150805404.png}
\caption{}
\end{figure}

Fast recovery: cwnd = ssthresh + 3

\begin{figure}
\centering
\includegraphics{/Users/whichway/workspace/E_ESSEX/CE265/MD/CE265.assets/image-20210419150814575.png}
\caption{}
\end{figure}

3.115

Summary: TCP Congestion Control

\begin{figure}
\centering
\includegraphics{/Users/whichway/workspace/E_ESSEX/CE265/MD/CE265.assets/image-20210419150834987.png}
\caption{}
\end{figure}

3.116

TCP throughput

avg. TCP thruput as function of window size, RTT?

\begin{itemize}
\item
  ignore slow start, assume always data to send
\end{itemize}

W: window size (measured in bytes) where loss occurs

\begin{itemize}
\item
  avg. window size (\# in-flight bytes) is ¾ W
\item
  avg. thruput is 3/4W per RTT
\end{itemize}

\begin{figure}
\centering
\includegraphics{/Users/whichway/workspace/E_ESSEX/CE265/MD/CE265.assets/image-20210419150843451.png}
\caption{}
\end{figure}

3.117

TCP Futures: TCP over ``long, fat pipes''

example: 1500 byte segments, 100ms RTT, want 10 Gbps throughput

requires W = 83,333 in-flight segments

throughput in terms of segment loss probability, L {[}Mathis 1997{]}:

\[TCP throughput = \frac{1.22·MSS}{RTT \sqrt{L}}\]

➜ to achieve 10 Gbps throughput, need a loss rate of L = 2·10-10 -- a
very small loss rate!

new versions of TCP for high-speed

3.118

TCP Fairness

fairness goal: if K TCP sessions share same bottleneck link of bandwidth
R, each should have average rate of R/K

\begin{figure}
\centering
\includegraphics{/Users/whichway/workspace/E_ESSEX/CE265/MD/CE265.assets/image-20210419150905997.png}
\caption{}
\end{figure}

3.119

Why is TCP fair?

two competing sessions:

additive increase gives slope of 1, as throughout increases

multiplicative decrease decreases throughput proportionally

\begin{figure}
\centering
\includegraphics{/Users/whichway/workspace/E_ESSEX/CE265/MD/CE265.assets/image-20210419150934176.png}
\caption{}
\end{figure}

3.120

Fairness (more)

Fairness and UDP

multimedia apps often do not use TCP

\begin{itemize}
\item
  do not want rate throttled by congestion control
\end{itemize}

instead use UDP:

\begin{itemize}
\item
  send audio/video at constant rate, tolerate packet loss
\end{itemize}

Fairness, parallel TCP connections

\begin{enumerate}
\def\labelenumi{\arabic{enumi}.}
\item
  application can open multiple parallel connections between two hosts
\item
  web browsers do this
\item
  e.g., link of rate R with 9 existing connections:
\end{enumerate}

\begin{itemize}
\item
  \begin{enumerate}
  \def\labelenumi{\arabic{enumi}.}
  \item
    1 new app asks for 1 TCP, gets rate R/10
  \end{enumerate}
\item
  \begin{enumerate}
  \def\labelenumi{\arabic{enumi}.}
  \item
    2 new app asks for 11 TCPs, gets R/2
  \end{enumerate}
\end{itemize}

\hypertarget{chapter-3-summary}{%
\subsection{Chapter 3: summary}\label{chapter-3-summary}}

\begin{itemize}
\item
  principles behind transport layer services:
\item
  \begin{itemize}
  \item
    multiplexing, demultiplexing
  \item
    reliable data transfer
  \item
    flow control
  \item
    congestion control
  \end{itemize}
\item
  instantiation, implementation in the Internet
\item
  \begin{itemize}
  \item
    UDP
  \item
    TCP
  \end{itemize}
\end{itemize}

next:

\begin{itemize}
\item
  leaving the network ``edge'' (application, transport layers)
\item
  into the network ``core''
\end{itemize}

\hypertarget{chapter-4-network-layer}{%
\section{Chapter 4 Network Layer}\label{chapter-4-network-layer}}

4.2

Chapter goal

understand principles behind network layer services:

\begin{enumerate}
\def\labelenumi{\arabic{enumi}.}
\item
  netwrok layer service models {[}网络服务模型{]}
\item
  forwarding versus routing {[}转发路由选择{]}
\item
  how a router workd {[}路由器工作原理{]}
\item
  routing ( path selection ) {[}路由选择算法{]}
\item
  boradcast multicast {[}广播和多播{]}
\end{enumerate}

\hypertarget{44-ip-datagram-format}{%
\subsection{4.4 IP: Datagram Format}\label{44-ip-datagram-format}}

\begin{quote}
\end{quote}

\begin{enumerate}
\def\labelenumi{\arabic{enumi}.}
\item
  datagram format {[}数据报格式{]}
\end{enumerate}

\begin{quote}
约定的规矩 IP 的分片
片偏移是0,前面发了数据,片偏移量怎么计算,分片和重组{[}边缘主机处,尽可能减少路由器的负担{]}
\end{quote}

\begin{enumerate}
\def\labelenumi{\arabic{enumi}.}
\item
  IPV4 addressing {[}Internet Protocol version 4 选址{]}
\end{enumerate}

\begin{quote}
什么是子网 地址信息 路由聚合 最长前缀匹配
\end{quote}

\begin{enumerate}
\def\labelenumi{\arabic{enumi}.}
\item
  ICMP {[}Internet Control Message Protocol{]}
\end{enumerate}

\begin{quote}
\end{quote}

\begin{enumerate}
\def\labelenumi{\arabic{enumi}.}
\item
  IPV6 {[}Internet Protocol version 6{]}
\end{enumerate}

4.33

\begin{quote}
肖云:网络层三大组件
\end{quote}

host, router network layer functions:

\begin{figure}
\centering
\includegraphics{/Users/whichway/workspace/E_ESSEX/CE265/MD/CE265.assets/image-20210425081647742.png}
\caption{}
\end{figure}

\begin{quote}
ipv5 时间非常短 length 首部长度 20length +40options ( 0 - 60 bytes ),
最长可以达到60bytes ? 20 btes TTL time to live:
寿命。一直在找主机host, 会造成资源浪费, 设置最大的跳数。每过一个路由器
- 1, 到达最大跳数不在往前传了。 upper layer 上层协议, 6是UDP
\end{quote}

4.34

\begin{figure}
\centering
\includegraphics{/Users/whichway/workspace/E_ESSEX/CE265/MD/CE265.assets/image-20210425081847291.png}
\caption{}
\end{figure}

\begin{longtable}[]{@{}lllll@{}}
\toprule
& & & &\tabularnewline
\midrule
\endhead
& & & &\tabularnewline
& & & &\tabularnewline
& & & &\tabularnewline
& & & &\tabularnewline
& & & &\tabularnewline
& & & &\tabularnewline
\bottomrule
\end{longtable}

4.35

IP Fragmentation, reassembly {[} IP 数据报分片 {]}

\begin{enumerate}
\def\labelenumi{\arabic{enumi}.}
\item
  network lins have MTU ( maximum transfer size ) - largest possible
  link-level fram
\end{enumerate}

\begin{itemize}
\item
  different likn types, diffreent MTU
\end{itemize}

\begin{enumerate}
\def\labelenumi{\arabic{enumi}.}
\item
  Large IP datagram divided ( " fragmented" ) within net

  \begin{enumerate}
  \def\labelenumii{\arabic{enumii}.}
  \item
    one datagram becomes several datagrams
  \item
    "reassembled" only at final destination
  \item
    IP header bits used to identify, order related fragments
  \end{enumerate}
\end{enumerate}

4.36

IP fragmentation, reassembly {[} IP 数据报分片 {]}

\begin{figure}
\centering
\includegraphics{/Users/whichway/workspace/E_ESSEX/CE265/MD/CE265.assets/image-20210425083926933.png}
\caption{}
\end{figure}

\hypertarget{44-ipipv4-addressing--ip-ux5730ux5740-}{%
\subsection{4.4 IP:IPv4 Addressing {[} IP 地址
{]}}\label{44-ipipv4-addressing--ip-ux5730ux5740-}}

4.38

\begin{enumerate}
\def\labelenumi{\arabic{enumi}.}
\item
  IP Address: 32-bit identifier for host, router interface
\item
  Interface: connection between host/router and physical link

  \begin{enumerate}
  \def\labelenumii{\arabic{enumii}.}
  \item
    router's typically have multiple interfaces
  \item
    host typically has one or two interfaces ( e.g., wired Ethernet,
    wireless 802.11 )
  \end{enumerate}
\item
  IP addresses associated with each interface
\end{enumerate}

\begin{figure}
\centering
\includegraphics{/Users/whichway/workspace/E_ESSEX/CE265/MD/CE265.assets/image-20210425084550812.png}
\caption{}
\end{figure}

\begin{quote}
3个蓝色阴影 - 3个子网 ipv4 - 32bits \(2^{32}\) ipv6 - 40bits \(2^{40}\)
\end{quote}

\begin{figure}
\centering
\includegraphics{/Users/whichway/workspace/E_ESSEX/CE265/MD/CE265.assets/image-20210425084600870.png}
\caption{}
\end{figure}

4.43

IP addressing: CIDR{[} \textbf{Classless Interdomain Routing}{]}

\begin{figure}
\centering
\includegraphics{/Users/whichway/workspace/E_ESSEX/CE265/MD/CE265.assets/image-20210425100349596.png}
\caption{}
\end{figure}

4.44

IP addresses: how to get one?

Q: How does a \emph{host} get IP address?

\begin{itemize}
\item
  hard-coded by system admin in a file
\item
  \begin{itemize}
  \item
    Windows:
    control-panel-\textgreater network-\textgreater configuration-\textgreater tcp/ip-\textgreater properties
  \item
    UNIX: /etc/rc.config
  \end{itemize}
\item
  DHCP: Dynamic Host Configuration Protocol: dynamically get address
  from as server
\item
  \begin{itemize}
  \item
    ``plug-and-play''
  \end{itemize}
\end{itemize}

4.45

DHCP: Dynamic Host COnfiuration Protocol

\emph{goal:} allow host to \emph{dynamically} obtain its IP address from
network server when it joins network

\begin{itemize}
\item
  \begin{itemize}
  \item
    can renew its lease on address in use
  \item
    allows reuse of addresses (only hold address while connected/``on'')
  \item
    support for mobile users who want to join network (more shortly)
  \end{itemize}
\end{itemize}

\emph{DHCP overview:}

\begin{itemize}
\item
  \begin{itemize}
  \item
    host broadcasts ``DHCP discover'' msg {[}optional{]}
  \item
    DHCP server responds with ``DHCP offer'' msg {[}optional{]}
  \item
    host requests IP address: ``DHCP request'' msg
  \item
    DHCP server sends address: ``DHCP ack'' msg
  \end{itemize}
\end{itemize}

4.46

4.47

4.48

4.49

4.50

4.51

4.52

4.53

Hierarchical addressing: route aggregation

hierarchical addressing allows efficient advertisement of routing
information:

\begin{figure}
\centering
\includegraphics{/Users/whichway/workspace/E_ESSEX/CE265/MD/CE265.assets/image-20210425140610146.png}
\caption{}
\end{figure}

4.54

ISPs-R-Us has a more specific route to Organization 1

\begin{figure}
\centering
\includegraphics{/Users/whichway/workspace/E_ESSEX/CE265/MD/CE265.assets/image-20210425140635068.png}
\caption{}
\end{figure}

4.55

IP addressing: the last word...

\emph{Q:} how does an ISP get block of addresses?

\emph{A:} ICANN: Internet Corporation for Assigned

Names and Numbers \url{http://www.icann.org/}

\begin{itemize}
\item
  \begin{itemize}
  \item
    allocates addresses
  \item
    manages DNS
  \item
    assigns domain names, resolves disputes
  \end{itemize}
\end{itemize}

4.56

4.57

4.58

4.59

4.60

4.61

4.62

4.63

\hypertarget{44-ipicmp-ipv6}{%
\subsection{4.4 IP:ICMP, IPv6}\label{44-ipicmp-ipv6}}

\begin{quote}
解决IP不足的问题 网络地址转换:借助端口号维护从内到外的映射 IPV6
\end{quote}

4.65

ICMP: internet control message protocol

used by hosts \& routers to communicate network-level information error
reporting: unreachable host, network, port, protocol echo request/reply
(used by ping) network-layer ``above'' IP: ICMP msgs carried in IP
datagrams ICMP message: type, code plus first 8 bytes of IP datagram
causing error

\begin{longtable}[]{@{}l@{}}
\toprule
Type Code description\tabularnewline
\midrule
\endhead
0 0 echo reply (ping)\tabularnewline
3 0 dest. network unreachable\tabularnewline
3 1 dest host unreachable\tabularnewline
3 2 dest protocol unreachable\tabularnewline
3 3 dest port unreachable\tabularnewline
3 6 dest network unknown\tabularnewline
3 7 dest host unknown\tabularnewline
4 0 source quench (congestion\tabularnewline
control - not used)\tabularnewline
\bottomrule
\end{longtable}

4.66 Traceroute and ICMP

\begin{itemize}
\item
  source sends series of UDP segments to dest
\item
  \begin{itemize}
  \item
    first set has TTL =1
  \item
    second set has TTL=2, etc.
  \item
    unlikely port number
  \end{itemize}
\item
  when \emph{n}th set of datagrams arrives to nth router:
\item
  \begin{itemize}
  \item
    router discards datagrams
  \item
    and sends source ICMP messages (type 11, code 0)
  \item
    ICMP messages includes name of router \& IP address
  \end{itemize}
\item
  when ICMP messages arrives, source records RTTs
\end{itemize}

\emph{stopping criteria:}

\begin{itemize}
\item
  UDP segment eventually arrives at destination host
\item
  destination returns ICMP ``port unreachable'' message (type 3, code 3)
\item
  source stops
\end{itemize}

4.67

IPV6: motivation {[}研究动机{]}

\begin{quote}
没有首部检查和
\end{quote}

initial motivation: 32-bit address space soon to be completely
allocated.

additional motivation:

\begin{enumerate}
\def\labelenumi{\arabic{enumi}.}
\item
  header format helps speed processing/forwarding
\item
  header changes to facilitate QoS
\end{enumerate}

IPv6 datagram format:

\begin{enumerate}
\def\labelenumi{\arabic{enumi}.}
\item
  fixed-length 40 byte header
\item
  no fragmentation allowed
\end{enumerate}

4.68

IPv6 datagram format

priority: identify priority among datagrams in flow

flow Label: identify datagrams in same ``flow.'' (concept of``flow'' not
well defined).

next header: identify upper layer protocol for data

\begin{figure}
\centering
\includegraphics{/Users/whichway/workspace/E_ESSEX/CE265/MD/CE265.assets/image-20210425094707192.png}
\caption{}
\end{figure}

4.69

\begin{itemize}
\item
  \emph{checksum}: removed entirely to reduce processing time at each
  hop
\item
  \emph{options:} allowed, but outside of header, indicated by ``Next
  Header'' field
\item
  \emph{ICMPv6:} new version of ICMP
\item
  \begin{itemize}
  \item
    additional message types, e.g. ``Packet Too Big''
  \item
    multicast group management functions
  \end{itemize}
\end{itemize}

4.70

Transition from IPv4 to IPv6

\begin{quote}
固定长度: 40bytes

\begin{verbatim}
	IPv6 数据报更短一点
\end{verbatim}
\end{quote}

\begin{enumerate}
\def\labelenumi{\arabic{enumi}.}
\item
  not all routers can be upgraded simulateously

  \begin{enumerate}
  \def\labelenumii{\arabic{enumii}.}
  \item
    no flag days
  \item
    how will network operate with IPv4 to IPv6
  \end{enumerate}
\item
  tunelling: IPv6 datagram carried as payload in IPv4 datagram amoung
  IPv4 routers
\end{enumerate}

\begin{figure}
\centering
\includegraphics{/Users/whichway/workspace/E_ESSEX/CE265/MD/CE265.assets/image-20210426141352200.png}
\caption{}
\end{figure}

4.71

Tunneling

\begin{quote}
把数据报封装📦,甚至是倒着封装在上层的
\end{quote}

\begin{figure}
\centering
\includegraphics{/Users/whichway/workspace/E_ESSEX/CE265/MD/CE265.assets/image-20210426141340392.png}
\caption{}
\end{figure}

4.72

Tunneling

\begin{figure}
\centering
\includegraphics{/Users/whichway/workspace/E_ESSEX/CE265/MD/CE265.assets/image-20210426141539636.png}
\caption{}
\end{figure}

4.73

IPv6: adoption

\begin{itemize}
\item
  US National Institutes of Standards estimate {[}2013{]}:
\item
  \begin{itemize}
  \item
    \textasciitilde3\% of industry IP routers
  \item
    \textasciitilde11\% of US gov't routers
  \end{itemize}
\end{itemize}

\begin{itemize}
\item
  \emph{Long (long!) time for deployment, use}
\item
  \begin{itemize}
  \item
    20 years and counting!
  \item
    think of application-level changes in last 20 years: WWW, Facebook,
    \ldots{}
  \item
    \emph{Why?}
  \end{itemize}
\end{itemize}

\hypertarget{-45-routing-algorithms-link-state}{%
\subsection{⭐️ 4.5 Routing Algorithms: Link
State}\label{-45-routing-algorithms-link-state}}

\begin{quote}
不要在 Dijkstra 列 DV 的表 2 个算法迭代的公式是唯一的依据
简单计算,考察原理理解,不是考计算功力
\end{quote}

4.75

Interplay between routing, forwarding

\begin{figure}
\centering
\includegraphics{/Users/whichway/workspace/E_ESSEX/CE265/MD/CE265.assets/image-20210426142026440.png}
\caption{}
\end{figure}

4.76

Graph abstraction

graph: G = (N,E)

N = set of routers = \{ u, v, w, x, y, z \}

E = set of links =\{ (u,v), (u,x), (v,x), (v,w), (x,w), (x,y), (w,y),
(w,z), (y,z) \}

4.78

\begin{quote}
\end{quote}

4.82

\begin{quote}
N' {[}节点集合, 不是路径{]}

\begin{quote}
每个点的邻近一个路径
\end{quote}
\end{quote}

\begin{longtable}[]{@{}llllll@{}}
\toprule
N' {[}节点集合, 不是路径{]} & D(v) & D(w) & D(x) & D(y) &
D(z)\tabularnewline
\midrule
\endhead
\{ u \} & 7,u & \hl{3,u} & 5,u & \(\infty\) & \(\infty\)\tabularnewline
\{ uw \} ( 3, u ) & 6( 3+3 ) ,w & 3, u & 7( 3+4 ), w; \hl{\textbf{{[} 5,
u {]}}} \textless{} 7( 3+4 ), w & 11( 3+ 8 ), w &
\(\infty\)\tabularnewline
\{ uwx \} ( 5, u ) & \(\infty\); \hl{{[}6( 3+3 ) ,w{]}} \textless{}
\(\infty\) & 9 ( 5+4 ), x & 5, u & 12( 5 + 7 ), x; \hl{{[}11( 3+ 8 ),
w{]}} \textless{} 12( 5 + 7 ), x; & 14 ( 5+9 ), x\tabularnewline
& & & & &\tabularnewline
& & & & &\tabularnewline
& & & & &\tabularnewline
& & & & &\tabularnewline
& & & & &\tabularnewline
\bottomrule
\end{longtable}

\begin{figure}
\centering
\includegraphics{/Users/whichway/workspace/E_ESSEX/CE265/MD/CE265.assets/image-20210426150101598.png}
\caption{}
\end{figure}

\begin{enumerate}
\def\labelenumi{\arabic{enumi}.}
\item
  准备开始
\end{enumerate}

\begin{longtable}[]{@{}llllll@{}}
\toprule
N' {[}节点集合, 不是路径{]} & D(v) & D(w) & D(x) & D(y) &
D(z)\tabularnewline
\midrule
\endhead
\(\infty\) & \(\infty\) & \(\infty\) & \(\infty\) & \(\infty\) &
\(\infty\)\tabularnewline
\bottomrule
\end{longtable}

\begin{enumerate}
\def\labelenumi{\arabic{enumi}.}
\item
  第一个节点集合的最短路计算 ( 邻近1个节点,顺序可以打乱)
\end{enumerate}

1.1 第一个节点集合的最短路比较 ( 邻近1个节点,顺序可以打乱)

1.2 第一个节点集合的最短路结果 ( 邻近1个节点,顺序可以打乱)

1.2 第一个节点集合的最短路去除已经用的值 ( 邻近1个节点,顺序可以打乱)

\hypertarget{-45-routing-algorithms-distance-vector}{%
\subsection{⭐️ 4.5 Routing Algorithms: Distance
Vector}\label{-45-routing-algorithms-distance-vector}}

\begin{quote}
也是一个更新的公式 有初始化的过程 cost to 节点的费用 非邻居的是 无穷大∞
更新方式是BF方程
\end{quote}

\hypertarget{-45-routing-algorithmshiericial-rounting}{%
\subsection{⭐️ 4.5 Routing Algorithms:Hiericial
Rounting}\label{-45-routing-algorithmshiericial-rounting}}

\begin{quote}
收敛速度对比 鲁棒性的对比🆚
\end{quote}

\hypertarget{46-routing-in-the-internet-riprouting-information-protocol}{%
\subsection{4.6 routing in the Internet: RIP(Routing Information
Protocol)}\label{46-routing-in-the-internet-riprouting-information-protocol}}

\hypertarget{46-ospfopen-shortest-path-first}{%
\subsection{4.6 OSPF(Open Shortest Path
First)}\label{46-ospfopen-shortest-path-first}}

\hypertarget{46-bgpborder-gateway-protocol}{%
\subsection{4.6 BGP(Border Gateway
Protocol)}\label{46-bgpborder-gateway-protocol}}

\hypertarget{47-broadcast-and-multicast-routing}{%
\subsection{4.7 broadcast and multicast
routing}\label{47-broadcast-and-multicast-routing}}

\begin{quote}
生成树, 没有环路 广播📢, 多播
\end{quote}

\hypertarget{chapter-5-link-layer}{%
\section{Chapter 5 Link Layer}\label{chapter-5-link-layer}}

\begin{quote}
计算CRC校验码 和mod2运算
\end{quote}

\hypertarget{53-multiple-access-protocols}{%
\subsection{5.3 Multiple Access
Protocols}\label{53-multiple-access-protocols}}

5.19

Given: boradcast channel of rate R bps

desiderate

\begin{enumerate}
\def\labelenumi{\arabic{enumi}.}
\item
  {[}效率, 有效性{]} one note, R
\item
  {[}公平{]} M nodes, R/M
\item
  {[}分布式, 完全分散{]} Fully decentralized,

  \begin{quote}
  击鼓传花🥁, 没有一个特殊的节点协调传输,没有同步

  ❓ 非同步的 电路 有吗
  \end{quote}

  \begin{enumerate}
  \def\labelenumii{\arabic{enumii}.}
  \item
    no special node to coordineate transmissions
  \item
    no synchronization of clocks, slots
  \end{enumerate}
\item
  {[}简单性{]} Simple
\end{enumerate}

5.20

5.21

TDMA: time division multiple access

\begin{quote}
\begin{enumerate}
\def\labelenumi{\arabic{enumi}.}
\item
  只有一个节点时,没法全速传输
\item
  公平性满足
\item
  高度分散
\item
  很简单
\end{enumerate}
\end{quote}

\begin{enumerate}
\def\labelenumi{\arabic{enumi}.}
\item
  access to channel in "rounds"
\item
  each station gets fixed length slot
\item
  unused slots go idle
\item
  example: 6-station LAN, 1, 3, 4 have pkt, slots 2,5,6 idle
\end{enumerate}

5.22

FDMA: frequency division multiple access

\begin{enumerate}
\def\labelenumi{\arabic{enumi}.}
\item
  channel spectrum divided into frequency bands
\item
  each station assigned fixed frequency band
\item
  unused transmission time in frequency bands go idle
\item
  example:
\end{enumerate}

5.23

\begin{quote}
资源划分类的协议 资源不划分: 随机访问类的协议
\end{quote}

\begin{enumerate}
\def\labelenumi{\arabic{enumi}.}
\item
  {[}重传{]} two or more transmitting nods - collision
\item
  {[}判断有无传完{]}
\item
  {[}碰撞避免的{]}
\end{enumerate}

5.24

Slotted ALOHA{[}夏威夷打招呼,想发就发{]}

\begin{quote}
不满足完全分散 追求效率,导致分散性降低
\end{quote}

5.25 C: 碰撞时期 E: 空闲时期 S: 成功时期

5.26 Slotted ALOHA efficiency: long-run fraction of successful
slots(many nodes, all with many frames to send)

5.27 Pure (unslotted) ALOHA

\begin{quote}
产生冲突的概率增加了 不划分时期, 完全分散,追求分散性,导致效率降低
\end{quote}

5.29 CSMA(carier sense multiple access){[}载波侦听{]}

listen before transimic

5.30

collisions can still occur:

propagation delay means two nodes may not hear each other's transmission

5.31 CSMA/CD {[} 碰撞检测 的 载波侦听 {]}

\begin{quote}
必须掌握 载波侦听,看看对方有没有在发送 CD是因为CSMA还是🈶️碰撞
\end{quote}

\begin{enumerate}
\def\labelenumi{\arabic{enumi}.}
\item
  {[}碰撞检测{]} collisions detecte d within short time
\item
  {[}colliding transmissions aborted, reducing channel wastage
\end{enumerate}

1.1

\begin{quote}
有线: 检测信号强度
\end{quote}

1.2

\begin{quote}
无线: 在无线中信号衰减,一边发很难一边检测
\end{quote}

2.1 human analogy: the polite conversationalist

\begin{quote}
\end{quote}

5.32

\begin{quote}
充分让其他节点知道数据frame产生故障 collision detect/abort time
\end{quote}

5.33

Ethernet CSMA/CD algorithm

\begin{enumerate}
\def\labelenumi{\arabic{enumi}.}
\item
  {[}M次碰撞后,怎么去选择碰撞的时间{]} after aborting, NIC enters
  binary ( exponential) backoff:
\end{enumerate}

\begin{quote}
0-5152 去选择等待时间, 回退到第2步,
碰撞时间多,可选时间多了,选择时间短的先发, 长的后发。目的是避免冲突。
\end{quote}

\begin{itemize}
\item
  after mth collision, NIC chooses K at random from \{ 0, 1, 2, ...,
  \( 2^m - 1 \)\}. NIC waits K 512 bit times, returns to Step 2.
\item
\end{itemize}

5.34

CSMA/CD efficiency

\[efficiency = \frac{1}{1+\frac{5t_{prop}}{t_{trans}}}\]

\begin{quote}
一个节点做了载波侦听后传输一个无限大的比特,\(t_{trans}\) 无限大,效率==
1
\end{quote}

5.35

random access MAC protocols

\begin{itemize}
\item
  high load: collision overhead
\end{itemize}

\begin{quote}
碰撞负担就很大
\end{quote}

5.36

polling {[}轮循协议{]}

\begin{quote}
master 轮流询问,发 poll 报文 高负债,低负载效率都很好
\end{quote}

\begin{enumerate}
\def\labelenumi{\arabic{enumi}.}
\item
  master node ``invites'' slave nodes to transmit in turn
\item
\end{enumerate}

5.37

token

\begin{quote}
短的报文, 令牌环
\end{quote}

5.38

\begin{quote}
不同频段传输,划分时期
\end{quote}

5.40

\begin{quote}
channel partitioning
\end{quote}

\hypertarget{54-lan-addressing-arp}{%
\subsection{5.4 LAN: Addressing, ARP}\label{54-lan-addressing-arp}}

\begin{quote}
ARP 子网,局域网内部进行查询
\end{quote}

\hypertarget{54-lan-ethernet}{%
\subsection{5.4 LAN: Ethernet}\label{54-lan-ethernet}}

\hypertarget{54-lan-switches}{%
\subsection{5.4 LAN: Switches}\label{54-lan-switches}}

\begin{quote}
转发 过滤 自学习 - 即插即用 是怎么自学习的
\end{quote}

\hypertarget{54-lan-vlans}{%
\subsection{5.4 LAN: VLANS}\label{54-lan-vlans}}

\begin{quote}
基于端口的 多个交换机,干线链接起来 MAC,IP, 网络层的协议
\end{quote}

\hypertarget{55-link-virtualization-mpls}{%
\subsection{5.5 link virtualization:
MPLS}\label{55-link-virtualization-mpls}}

\begin{quote}
多协议标签交换 MPS 首部,分辨数据,决定路径
\end{quote}

\hypertarget{56-data-center-networking}{%
\subsection{5.6 data center
networking}\label{56-data-center-networking}}

\begin{quote}
层次架构 负载均衡器
\end{quote}

\hypertarget{57-a-day-in-the-life-of-a-web-request}{%
\subsection{5.7 a day in the life of a web
request}\label{57-a-day-in-the-life-of-a-web-request}}

\begin{quote}
把各层协议串了一遍
\end{quote}

\hypertarget{--discussion-problem-57-a-day-in-the-life-of-a-web-request}{%
\subsection{⭐️ {[} Discussion Problem {]}5.7 a day in the life of a web
request}\label{--discussion-problem-57-a-day-in-the-life-of-a-web-request}}

5.88

\emph{Synthesis:} a day in the life of a web request

\begin{itemize}
\item
  journey down protocol stack complete!
\item
  \begin{itemize}
  \item
    application, transport, network, link
  \end{itemize}
\item
  putting-it-all-together: synthesis!
\item
  \begin{itemize}
  \item
    \emph{goal:} identify, review, understand protocols (at all layers)
    involved in seemingly simple scenario: requesting www page
  \item
    \emph{scenario:} student attaches laptop to campus network,
    requests/receives \url{www.google.com}
  \end{itemize}
\end{itemize}

5.89

A day in the life: scenario

\begin{figure}
\centering
\includegraphics{/Users/whichway/workspace/E_ESSEX/CE265/MD/CE265.assets/image-20210519082048097.png}
\caption{}
\end{figure}

\hypertarget{ce265-chapter-6-wireless-mobility}{%
\section{CE265 Chapter 6 Wireless
Mobility}\label{ce265-chapter-6-wireless-mobility}}

\begin{quote}
我现在的要求呢

就是很简单

电子笔记每个都要有批注,方便进行理解和校准,

同时先过一遍中文书,再过一遍英文书,通读,skim,不求甚解

skimming 的同时补充以md中的笔记,用来应对ielts的备考

同时用以证明自己的理解
\end{quote}

6.2

\emph{Background:}

\begin{itemize}
\item
  \# wireless (mobile) phone subscribers now exceeds \# wired phone
  subscribers (5-to-1)!
\item
  \# wireless Internet-connected devices equals \# wireline
  Internet-connected devices
\item
  \begin{itemize}
  \item
    laptops, Internet-enabled phones promise anytime untethered Internet
    access
  \end{itemize}
\item
  two important (but different) challenges
\item
  \begin{itemize}
  \item
    \emph{wireless:} communication over wireless link
  \item
    \emph{mobility:} handling the mobile user who changes point of
    attachment to network
  \end{itemize}
\end{itemize}

\hypertarget{61-introduction}{%
\subsection{\texorpdfstring{6.1 Introduction
}{6.1 Introduction }}\label{61-introduction}}

6.4 Elements of a wireless network

\begin{figure}
\centering
\includegraphics{/Users/whichway/workspace/E_ESSEX/CE265/MD/CE265 Chapter 6.assets/image-20210524143620389.png}
\caption{}
\end{figure}

6.5 wireless hosts

wireless hosts

\begin{itemize}
\item
  laptop, smartphone
\item
  run applications
\item
  may be stationary (non-mobile) or mobile
\item
  \begin{itemize}
  \item
    wireless does \emph{not} always mean mobility
  \end{itemize}
\end{itemize}

\begin{figure}
\centering
\includegraphics{/Users/whichway/workspace/E_ESSEX/CE265/MD/CE265 Chapter 6.assets/image-20210524143847124.png}
\caption{}
\end{figure}

6.6 base station

base station

\begin{itemize}
\item
  typically connected to wired network
\item
  relay - responsible for sending packets between wired network and
  wireless host(s) in its ``area''
\item
  \begin{itemize}
  \item
    e.g., cell towers, 802.11 access points
  \end{itemize}
\end{itemize}

\begin{figure}
\centering
\includegraphics{/Users/whichway/workspace/E_ESSEX/CE265/MD/CE265 Chapter 6.assets/image-20210524144127010.png}
\caption{}
\end{figure}

6.7 wireless link

wireless link

\begin{itemize}
\item
  typically used to connect mobile(s) to base station
\item
  also used as backbone link
\item
  multiple access protocol coordinates link access
\item
  various data rates, transmission distance
\end{itemize}

\begin{figure}
\centering
\includegraphics{/Users/whichway/workspace/E_ESSEX/CE265/MD/CE265 Chapter 6.assets/image-20210524144150637.png}
\caption{}
\end{figure}

6.8 Characteristics of selected wireless links

\begin{figure}
\centering
\includegraphics{/Users/whichway/workspace/E_ESSEX/CE265/MD/CE265 Chapter 6.assets/image-20210524144211481.png}
\caption{}
\end{figure}

6.9 infrastructure mode

infrastructure mode

\begin{itemize}
\item
  base station connects mobiles into wired network
\item
  handoff: mobile changes base station providing connection into wired
  network
\end{itemize}

\begin{figure}
\centering
\includegraphics{/Users/whichway/workspace/E_ESSEX/CE265/MD/CE265 Chapter 6.assets/image-20210524144256823.png}
\caption{}
\end{figure}

6.10 ad hoc mode

ad hoc mode(点对点)

\begin{quote}
ad hoc( for this, for this purpose only ),

为某种目的设置的

省去ap base stations
\end{quote}

\begin{itemize}
\item
  no base stations
\item
  nodes can only transmit to other nodes within link coverage
\item
  nodes organize themselves into a network: route among themselves
\end{itemize}

\includegraphics{/Users/whichway/workspace/E_ESSEX/CE265/MD/CE265 Chapter 6.assets/image-20210524151915634.png}

6.11 Wireless network taxonomy{[}分类{]}

\begin{figure}
\centering
\includegraphics{/Users/whichway/workspace/E_ESSEX/CE265/MD/CE265 Chapter 6.assets/image-20210524152231876.png}
\caption{}
\end{figure}

\hypertarget{62-wireless-links-characteristics--cdma}{%
\subsection{6.2 Wireless links, characteristics,
CDMA}\label{62-wireless-links-characteristics--cdma}}

6.13 Wireless Link Characteristics (1)

\begin{quote}
衰减的信号强度

来自其他信号源的干扰

多路传播📣
\end{quote}

\emph{important} differences from wired link \ldots.

\begin{itemize}
\item
  \begin{itemize}
  \item
    \emph{decreased signal strength:} radio signal attenuates as it
    propagates through matter (path loss)
  \item
    \emph{interference from other sources:} standardized wireless
    network frequencies (e.g., 2.4 GHz) shared by other devices (e.g.,
    phone); devices (motors) interfere as well
  \item
    \emph{multipath propagation:} radio signal reflects off objects
    ground, arriving ad destination at slightly different times
  \end{itemize}
\end{itemize}

\ldots. make communication across (even a point to point) wireless link
much more ``difficult''

6.14 Wireless Link Characteristics (2)

\begin{quote}
SNR (Signal to Noise)
\end{quote}

\begin{itemize}
\item
  SNR: signal-to-noise ratio
\item
  \begin{itemize}
  \item
    larger SNR -- easier to extract signal from noise (a ``good thing'')
  \end{itemize}
\item
  \emph{SNR versus BER tradeoffs}
\item
  \begin{itemize}
  \item
    \emph{given physical layer:} increase power -\textgreater{} increase
    SNR-\textgreater decrease BER
  \item
    \emph{given SNR:} choose physical layer that meets BER requirement,
    giving highest thruput
  \item
    \begin{itemize}
    \item
      SNR may change with mobility: dynamically adapt physical layer
      (modulation technique, rate)
    \end{itemize}
  \end{itemize}
\end{itemize}

\begin{figure}
\centering
\includegraphics{/Users/whichway/workspace/E_ESSEX/CE265/MD/CE265 Chapter 6.assets/image-20210525000058641.png}
\caption{}
\end{figure}

6.15 Wireless network characteristics

Multiple wireless senders and receivers create additional problems
(beyond multiple access):

\begin{figure}
\centering
\includegraphics{/Users/whichway/workspace/E_ESSEX/CE265/MD/CE265 Chapter 6.assets/image-20210525000137324.png}
\caption{}
\end{figure}

\emph{Hidden terminal problem}

\begin{itemize}
\item
  B, A hear each other
\item
  B, C hear each other
\item
  A, C can not hear each other means A, C unaware of their interference
  at B
\end{itemize}

\emph{Signal attenuation:}

\begin{itemize}
\item
  B, A hear each other
\item
  B, C hear each other
\item
  A, C can not hear each other interfering at B
\end{itemize}

6.16 Code Division Multiple Access (CDMA)

\begin{itemize}
\item
  unique ``code'' assigned to each user; i.e., code set partitioning
\item
  \begin{itemize}
  \item
    all users share same frequency, but each user has own ``chipping''
    sequence (i.e., code) to encode data
  \item
    allows multiple users to ``coexist'' and transmit simultaneously
    with minimal interference (if codes are ``orthogonal'')
  \end{itemize}
\item
  \emph{encoded signal} = (original data) X (chipping sequence)
\item
  \emph{decoding:} inner-product of encoded signal and chipping sequence
\end{itemize}

6.17 CDMA encode/decode

\begin{figure}
\centering
\includegraphics{/Users/whichway/workspace/E_ESSEX/CE265/MD/CE265 Chapter 6.assets/image-20210525000223673.png}
\caption{}
\end{figure}

6.18 CDMA: two-sender interference

\begin{figure}
\centering
\includegraphics{/Users/whichway/workspace/E_ESSEX/CE265/MD/CE265 Chapter 6.assets/image-20210525000245731.png}
\caption{}
\end{figure}

\hypertarget{63-ieee-80211-wireless-lans-wi-fi}{%
\subsection{6.3 IEEE 802.11 wireless LANs
(``Wi-Fi'')}\label{63-ieee-80211-wireless-lans-wi-fi}}

6.20 IEEE 802.11 Wireless LAN

\begin{quote}
区分了 IEEE 802.11 小字母后缀的一些Wi-Fi频段,记忆为主

802.11b 2.4-5GHz

\begin{quote}
DSSS (direct sequence spread spectrum)
\end{quote}
\end{quote}

802.11b

\begin{itemize}
\item
  ❖2.4-5 GHz unlicensed spectrum
\item
  ❖up to 11 Mbps
\item
  ❖direct sequence spread spectrum (DSSS) in physical layer
\item
  \begin{itemize}
  \item
    all hosts use same chipping code
  \end{itemize}
\end{itemize}

802.11a

\begin{itemize}
\item
  \begin{itemize}
  \item
    5-6 GHz range
  \item
    up to 54 Mbps
  \end{itemize}
\end{itemize}

802.11g

\begin{itemize}
\item
  \begin{itemize}
  \item
    2.4-5 GHz range
  \item
    up to 54 Mbps
  \end{itemize}
\end{itemize}

802.11n: multiple antennae

\begin{itemize}
\item
  \begin{itemize}
  \item
    2.4-5 GHz range
  \item
    up to 200 Mbps
  \end{itemize}
\item
  all use CSMA/CA for multiple access
\item
  all have base-station and ad-hoc network versions
\end{itemize}

6.21 802.11 LAN architecture

\begin{itemize}
\item
  wireless host communicates with base station
\item
  \begin{itemize}
  \item
    base station = access point (AP)
  \end{itemize}
\item
  Basic Service Set (BSS) (aka ``cell'') in infrastructure mode
  contains:
\item
  \begin{itemize}
  \item
    wireless hosts
  \item
    access point (AP): base station
  \item
    ad hoc mode: hosts only
  \end{itemize}
\end{itemize}

\begin{figure}
\centering
\includegraphics{/Users/whichway/workspace/E_ESSEX/CE265/MD/CE265 Chapter 6.assets/image-20210524234846093.png}
\caption{}
\end{figure}

6.22 802.11: Channels, association

\begin{itemize}
\item
  802.11b: 2.4GHz-2.485GHz spectrum divided into 11 channels at
  different frequencies
\item
  \begin{itemize}
  \item
    AP admin chooses frequency for AP
  \item
    interference possible: channel can be same as that chosen by
    neighboring AP!
  \end{itemize}
\item
  host: must \emph{associate} with an AP
\item
  \begin{itemize}
  \item
    scans channels, listening for \emph{beacon frames} containing AP's
    name (SSID) and MAC address
  \item
    selects AP to associate with
  \item
    may perform authentication {[}Chapter 8{]}
  \item
    will typically run DHCP to get IP address in AP's subnet
  \end{itemize}
\end{itemize}

6.23 802.11: passive/active scanning

\emph{passive scanning:}

\begin{enumerate}
\def\labelenumi{\arabic{enumi}.}
\item
  beacon frames sent from APs
\item
  association Request frame sent: H1 to selected AP
\item
  association Response frame sent from selected AP to H1
\end{enumerate}

\emph{active scanning}:

\begin{enumerate}
\def\labelenumi{\arabic{enumi}.}
\item
  Probe Request frame broadcast from H1
\item
  Probe Response frames sent from APs
\item
  Association Request frame sent: H1 to selected AP
\item
  Association Response frame sent from selected AP to H1
\end{enumerate}

\begin{figure}
\centering
\includegraphics{/Users/whichway/workspace/E_ESSEX/CE265/MD/CE265 Chapter 6.assets/image-20210524234936378.png}
\caption{}
\end{figure}

6.24 IEEE 802.11: multiple access

\begin{itemize}
\item
  avoid collisions: 2+ nodes transmitting at same time
\item
  802.11: CSMA - sense before transmitting
\item
  \begin{itemize}
  \item
    don't collide with ongoing transmission by other node
  \end{itemize}
\item
  802.11: \emph{no} collision detection!
\item
  \begin{itemize}
  \item
    difficult to receive (sense collisions) when transmitting due to
    weak received signals (fading)
  \item
    can't sense all collisions in any case: hidden terminal, fading
  \item
    goal: \emph{avoid collisions**:} CSMA/C(ollision)A(voidance)
  \end{itemize}
\end{itemize}

\begin{figure}
\centering
\includegraphics{/Users/whichway/workspace/E_ESSEX/CE265/MD/CE265 Chapter 6.assets/image-20210524235015379.png}
\caption{}
\end{figure}

6.25

IEEE 802.11 MAC Protocol: CSMA/CA

\begin{quote}
CSMA载波侦听

不做碰撞的检测
\end{quote}

802.11 sender

1 if sense channel idle for DIFD then transmit entire frame

2 if senes channel busy then

2.1 start random backoff time

2.2 timer counts down while channel idle

2.3 transmit when timer expires

802.11 receiver

if frame received OK

return ACK after SIFS (ACK needed due to hidden terminal problem)

\begin{quote}
一次性传整个数据frame

SIFS, 状态切换, 从听转为发

基础模式,有可能碰撞💥
\end{quote}

\begin{figure}
\centering
\includegraphics{/Users/whichway/workspace/E_ESSEX/CE265/MD/CE265 Chapter 6.assets/image-20210524141744378.png}
\caption{}
\end{figure}

6.26

\begin{quote}
预约模式

利用预约的节点传输数据frame

避免长数据碰撞💥

RTS (Request to send) 这些请求frame是很短的,开销是可以忍的

CTS(Clear to send) 是 基站发出的, 利用广播的信道发送的

CTS其他节点会被告知,延迟发送RTS
\end{quote}

Avoiding collisions (more)

\emph{idea:} allow sender to ``reserve'' channel rather than random
access of data frames: avoid collisions of long data frames

\begin{itemize}
\item
  sender first transmits \emph{small} request-to-send (RTS) packets to
  BS using CSMA
\item
  \begin{itemize}
  \item
    RTSs may still collide with each other (but they're short)
  \end{itemize}
\item
  BS broadcasts clear-to-send CTS in response to RTS
\item
  CTS heard by all nodes
\item
  \begin{itemize}
  \item
    sender transmits data frame
  \item
    other stations defer transmissions
  \end{itemize}
\end{itemize}

\emph{avoid data frame collisions completely} \emph{using small
reservation packets!}

6.27

Collision Avoidance: RTS-CTS exchange

\begin{quote}
RTS 发送

B, CTS 响应

收到数据后,B,发ACK

❓ 为什么是折的,不是只有B发CTS(clear to send), ACK 吗

大,长的数据DATA, 发一个小的RTS预约模式还是划算的
\end{quote}

\begin{figure}
\centering
\includegraphics{/Users/whichway/workspace/E_ESSEX/CE265/MD/CE265 Chapter 6.assets/image-20210524142225180.png}
\caption{}
\end{figure}

\begin{quote}
四个节点
\end{quote}

\begin{figure}
\centering
\includegraphics{/Users/whichway/workspace/E_ESSEX/CE265/MD/CE265 Chapter 6.assets/5151621837595_.pic_hd.jpg}
\caption{}
\end{figure}

6.28

802.11 frame: addressing

\begin{quote}
少不了的是 payload 和 CRC

payload: 包含上层协议,通常长度是1500字节,

frame control 类型,子类型,RTS, CTS, DATA, ACK

地址1 目的地址

地址 2 源地址,

地址3 路由器MAC地址,跨越不同子网

地址4 ad hoc 地址
\end{quote}

\begin{figure}
\centering
\includegraphics{/Users/whichway/workspace/E_ESSEX/CE265/MD/CE265 Chapter 6.assets/image-20210524142701839.png}
\caption{}
\end{figure}

Address 1: MAC(media access control){[}媒体存取控制{]} address of
wireless host or AP to \hl{receive} this frame

\begin{quote}
\end{quote}

Address 2: MAC(media access control){[}媒体存取控制{]} address of
wireless host or AP to \hl{transmitting} this frame

6.29 802.11 frame: addressing

\begin{figure}
\centering
\includegraphics{/Users/whichway/workspace/E_ESSEX/CE265/MD/CE265 Chapter 6.assets/image-20210524144708521.png}
\caption{}
\end{figure}

6.30 802.11 frame: more

\begin{figure}
\centering
\includegraphics{/Users/whichway/workspace/E_ESSEX/CE265/MD/CE265 Chapter 6.assets/image-20210524144642736.png}
\caption{}
\end{figure}

6.31 802.11: mobility within same subnet

\begin{quote}
只有一台交换机连接,位于一个子网

移动时位于一个子网,IP不变,可持续发送数据

主机探测到AP2信号📶强,产生关联
\end{quote}

802.11: mobility within same subnet

\begin{itemize}
\item
  H1 remains in same IP subnet: IP address can remain same
\item
  switch: which AP is associated with H1?
\item
  \begin{itemize}
  \item
    self-learning (Ch. 5): switch will see frame from H1 and
    ``remember'' which switch port can be used to reach H1
  \end{itemize}
\end{itemize}

\begin{figure}
\centering
\includegraphics{/Users/whichway/workspace/E_ESSEX/CE265/MD/CE265 Chapter 6.assets/image-20210524144510672.png}
\caption{}
\end{figure}

6.32 802.11: advanced capabilities

\begin{quote}
速率自适应

拥塞控制
\end{quote}

802.11: advanced capabilities

\emph{Rate adaptation}

\begin{itemize}
\item
  base station, mobile dynamically change transmission rate (physical
  layer modulation technique) as mobile moves, SNR(信道比) varies
\end{itemize}

\begin{figure}
\centering
\includegraphics{/Users/whichway/workspace/E_ESSEX/CE265/MD/CE265 Chapter 6.assets/image-20210524145022829.png}
\caption{}
\end{figure}

\textbackslash1. SNR decreases, BER increase as node moves away from
base station

\textbackslash2. When BER becomes too high, switch to lower transmission
rate but with lower BER

6.33 802.11: advanced capabilities

\emph{power management}

\begin{quote}
AP 清楚哪些是休眠状态,就不传输数据了

250 微秒, 节点节省自己的能源

没有frame要发送和接受的99\%处于睡眠💤状态, 非常节省能源
\end{quote}

\begin{itemize}
\item
  node-to-AP(Access Point){[}接入点{]}: ``I am going to sleep until next
  beacon{[}信号灯🚥{]} frame''
\item
  \begin{itemize}
  \item
    AP knows not to transmit frames to this node
  \item
    node wakes up before next beacon frame
  \end{itemize}
\item
  beacon frame: contains list of mobiles with AP-to-mobile frames
  waiting to be sent
\item
  \begin{itemize}
  \item
    node will stay awake if AP-to-mobile frames to be sent; otherwise
    sleep again until next beacon frame
  \end{itemize}
\end{itemize}

6.34 802.15: personal area network

\begin{quote}
小范围用于 802.15
\end{quote}

\begin{itemize}
\item
  less than 10 m diameter
\item
  replacement for cables (mouse, keyboard, headphones)
\item
  ad hoc: no infrastructure
\item
  master/slaves:
\item
  \begin{itemize}
  \item
    slaves request permission to send (to master)
  \item
    master grants requests
  \end{itemize}
\item
  802.15: evolved from Bluetooth specification
\item
  \begin{itemize}
  \item
    2.4-2.5 GHz radio band
  \item
    up to 721 kbps
  \end{itemize}
\end{itemize}

\begin{figure}
\centering
\includegraphics{/Users/whichway/workspace/E_ESSEX/CE265/MD/CE265 Chapter 6.assets/image-20210524150543095.png}
\caption{}
\end{figure}

\hypertarget{64-cellular-internet-access}{%
\subsection{6.4 Cellular Internet
access}\label{64-cellular-internet-access}}

6.36 Components of cellular network architecture

\begin{figure}
\centering
\includegraphics{/Users/whichway/workspace/E_ESSEX/CE265/MD/CE265 Chapter 6.assets/image-20210524233808196.png}
\caption{}
\end{figure}

6.37 Cellular networks: the first hop

Two techniques for sharing mobile-to-BS radio spectrum

\begin{itemize}
\item
  combined FDMA/TDMA: divide spectrum in frequency channels, divide each
  channel into time slots
\item
  CDMA: code division multiple access
\end{itemize}

\begin{figure}
\centering
\includegraphics{/Users/whichway/workspace/E_ESSEX/CE265/MD/CE265 Chapter 6.assets/image-20210524233842589.png}
\caption{}
\end{figure}

6.38 2G (voice) network architecture

\begin{quote}
BTS( Base Transceiver station )

BSC( Base Station Controler )

MSC( Mobile Switching Center )
\end{quote}

\begin{figure}
\centering
\includegraphics{/Users/whichway/workspace/E_ESSEX/CE265/MD/CE265 Chapter 6.assets/image-20210524233905363.png}
\caption{}
\end{figure}

6.39 3G (voice+data) network architecture

\emph{Key insight:} new cellular data network operates \emph{in
parallel} (except at edge) with existing cellular voice network

\begin{itemize}
\item
  ❖ voice network unchanged in core
\item
  ❖ data network operates in parallel
\end{itemize}

\begin{figure}
\centering
\includegraphics{/Users/whichway/workspace/E_ESSEX/CE265/MD/CE265 Chapter 6.assets/image-20210524234223035.png}
\caption{}
\end{figure}

6.40 3G (voice+data) network architecture

\begin{quote}
射频

核心网

公众网络
\end{quote}

\begin{figure}
\centering
\includegraphics{/Users/whichway/workspace/E_ESSEX/CE265/MD/CE265 Chapter 6.assets/image-20210524234258370.png}
\caption{}
\end{figure}

\hypertarget{65-mobility-principles-addressing-and-routing-to-mobile-users}{%
\subsection{6.5 {[}Mobility{]} Principles: addressing and routing to
mobile
users}\label{65-mobility-principles-addressing-and-routing-to-mobile-users}}

6.42 What is mobility?

\begin{itemize}
\item
  spectrum{[}光谱{]} of mobility{[}移动性{]}, from the \emph{network}
  perspective:
\end{itemize}

\begin{figure}
\centering
\includegraphics{/Users/whichway/workspace/E_ESSEX/CE265/MD/CE265 Chapter 6.assets/image-20210524151539918.png}
\caption{}
\end{figure}

6.43 Mobility: vocabulary

\begin{quote}
Home network{[}家乡网络{]}, 永久固定的

Home agent{[}家乡代理{]}, 先转到代理路由器上
\end{quote}

\begin{figure}
\centering
\includegraphics{/Users/whichway/workspace/E_ESSEX/CE265/MD/CE265 Chapter 6.assets/image-20210524151659147.png}
\caption{}
\end{figure}

\begin{enumerate}
\def\labelenumi{\arabic{enumi}.}
\item
  Home network{[}家乡网络{]} \emph{home network:} permanent ``home'' of
  mobile

  (e.g., 128.119.40/24)
\item
  \emph{home agent:} \emph{entity that will perform mobility functions
  on behalf of mobile, when mobile is remote}
\item
  \emph{permanent address:} address in home network, \emph{can always}
  be used to reach mobile. e.g., 128.119.40.186
\end{enumerate}

6.46 Mobility: approaches

\begin{quote}
路由器工作量大,需要知道各个移动用户所在的位置,节点

网络的设计:边缘是复杂的,核心是简单的

无论直接路由,还是间接路由, 这种违反了设计
\end{quote}

\begin{itemize}
\item
  🚫 \emph{let routing handle it:} routers advertise permanent address of
  mobile-nodes-in-residence via usual routing table exchange.
\item
  \begin{itemize}
  \item
    routing tables indicate where each mobile located
  \item
    no changes to end-systems
  \end{itemize}
\item
  \emph{let end-systems handle it:}
\item
  \begin{itemize}
  \item
    \emph{indirect routing:} communication from correspondent to mobile
    goes through home agent, then forwarded to remote
  \item
    \emph{direct routing:} correspondent gets foreign address of mobile,
    sends directly to mobile
  \end{itemize}
\end{itemize}

6.47 not scalable to millions of mobiles

not scalable to millions of mobiles

6.48 Mobility: registration

\begin{figure}
\centering
\includegraphics{/Users/whichway/workspace/E_ESSEX/CE265/MD/CE265 Chapter 6.assets/image-20210524153159195.png}
\caption{}
\end{figure}

end result:

\begin{itemize}
\item
  foreign agent knows about mobile
\item
  home agent knows location of mobile
\end{itemize}

6.49 Mobility via indirect routing

\begin{figure}
\centering
\includegraphics{/Users/whichway/workspace/E_ESSEX/CE265/MD/CE265 Chapter 6.assets/image-20210524153632669.png}
\caption{}
\end{figure}

6.50 Indirect Routing: comments

\begin{quote}
三角网络

外部代理告知家乡网络
\end{quote}

\begin{itemize}
\item
  mobile uses two addresses:
\item
  \begin{itemize}
  \item
    permanent address: used by correspondent (hence mobile location is
    \emph{transparent} to correspondent)
  \item
    care-of-address: used by home agent to forward datagrams to mobile
  \end{itemize}
\item
  foreign agent functions may be done by mobile itself
\item
  triangle routing: correspondent-home-network-mobile
\item
  \begin{itemize}
  \item
    inefficient when
  \end{itemize}
\end{itemize}

correspondent, mobile are in same network

6.51 Indirect routing: moving between networks

\begin{quote}
记住家乡网络的ip

间接路由选择有一定透明性
\end{quote}

\begin{itemize}
\item
  suppose mobile user moves to another network
\item
  \begin{itemize}
  \item
    registers with new foreign agent
  \item
    new foreign agent registers with home agent
  \item
    home agent update care-of-address for mobile
  \item
    packets continue to be forwarded to mobile (but with new
    care-of-address)
  \end{itemize}
\item
  mobility, changing foreign networks transparent: \emph{on going
  connections can be maintained!}
\end{itemize}

6.52 Mobility via direct routing

\begin{figure}
\centering
\includegraphics{/Users/whichway/workspace/E_ESSEX/CE265/MD/CE265 Chapter 6.assets/image-20210524153936425.png}
\caption{}
\end{figure}

6.53 Mobility via direct routing: comments

\begin{quote}
克服三角路由

但不透明
\end{quote}

\begin{itemize}
\item
  overcome triangle routing problem
\item
  \emph{non-transparent to correspondent:} correspondent must get
  care-of-address from home agent
\item
  \begin{itemize}
  \item
    what if mobile changes visited network?
  \end{itemize}
\end{itemize}

\begin{figure}
\centering
\includegraphics{/Users/whichway/workspace/E_ESSEX/CE265/MD/CE265 Chapter 6.assets/image-20210524154207470.png}
\caption{}
\end{figure}

6.54 Accommodating mobility with direct routing

\begin{quote}
锚⚓️的外部代理

:为什么要有锚⚓️
\end{quote}

\begin{itemize}
\item
  anchor foreign agent: FA in first visited network
\item
  data always routed first to anchor FA
\item
  when mobile moves: new FA arranges to have data forwarded from old FA
  (chaining)
\end{itemize}

\begin{figure}
\centering
\includegraphics{/Users/whichway/workspace/E_ESSEX/CE265/MD/CE265 Chapter 6.assets/image-20210524154229149.png}
\caption{}
\end{figure}

\hypertarget{66-mobile-ip}{%
\subsection{6.6 Mobile IP}\label{66-mobile-ip}}

6.56 Mobile IP

\begin{itemize}
\item
  RFC 3344
\item
  has many features we've seen:
\item
  \begin{itemize}
  \item
    home agents, foreign agents, foreign-agent registration,
    care-of-addresses, encapsulation (packet-within-a-packet)
  \end{itemize}
\item
  three components to standard:
\item
  \begin{itemize}
  \item
    indirect routing of datagrams
  \item
    agent discovery
  \item
    registration with home agent
  \end{itemize}
\end{itemize}

6.57 Mobile IP: indirect routing

\begin{figure}
\centering
\includegraphics{/Users/whichway/workspace/E_ESSEX/CE265/MD/CE265 Chapter 6.assets/image-20210524234413387.png}
\caption{}
\end{figure}

6.58 Mobile IP: agent discovery

\begin{itemize}
\item
  \emph{agent advertisement:} foreign/home agents advertise service by
  broadcasting ICMP messages (typefield = 9)
\end{itemize}

\begin{figure}
\centering
\includegraphics{/Users/whichway/workspace/E_ESSEX/CE265/MD/CE265 Chapter 6.assets/image-20210524234503078.png}
\caption{}
\end{figure}

6.59 Mobile IP: registration example

\begin{figure}
\centering
\includegraphics{/Users/whichway/workspace/E_ESSEX/CE265/MD/CE265 Chapter 6.assets/image-20210524234517228.png}
\caption{}
\end{figure}

\hypertarget{67-handling-mobility-in-cellular-networks}{%
\subsection{6.7 Handling mobility in cellular
networks}\label{67-handling-mobility-in-cellular-networks}}

6.60 Components of cellular network architecture

\begin{quote}
蜂窝网络, 也是来自早期的电信网络
\end{quote}

\begin{figure}
\centering
\includegraphics{/Users/whichway/workspace/E_ESSEX/CE265/MD/CE265 Chapter 6.assets/image-20210524154546838.png}
\caption{}
\end{figure}

6.61 Handling mobility in cellular networks

\begin{itemize}
\item
  \emph{home network}: network of cellular provider you subscribe to
  (e.g., Sprint PCS, Verizon)
\item
  \begin{itemize}
  \item
    \emph{home location register (HLR):} database in home network
    containing permanent cell phone \#, profile information (services,
    preferences, billing), information about current location (could be
    in another network)
  \end{itemize}
\item
  \emph{visited network:} network in which mobile currently resides
\item
  \begin{itemize}
  \item
    \emph{visitor location register (VLR):} database with entry for each
    user currently in network
  \item
    could be home network
  \end{itemize}
\end{itemize}

\hypertarget{68-mobility-and-higher-layer-protocols}{%
\subsection{6.8 Mobility and higher-layer
protocols}\label{68-mobility-and-higher-layer-protocols}}

6.62 GSM: indirect routing to mobile

\begin{quote}
HLR (Home Location Register)

VLR (Visitor Location Register)

MSC (Mobile Switching Center)
\end{quote}

\begin{figure}
\centering
\includegraphics{/Users/whichway/workspace/E_ESSEX/CE265/MD/CE265 Chapter 6.assets/image-20210524233419012.png}
\caption{}
\end{figure}

6.63 GSM: handoff with common MSC

\begin{itemize}
\item
  \emph{handoff goal:} route call via new base station (without
  interruption)
\item
  reasons for handoff:
\item
  \begin{itemize}
  \item
    stronger signal to/from new BSS (continuing connectivity, less
    battery drain)
  \item
    load balance: free up channel in current BSS
  \item
    GSM doesnt mandate why to perform handoff (policy), only how
    (mechanism)
  \end{itemize}
\item
  handoff initiated by old BSS
\end{itemize}

\begin{figure}
\centering
\includegraphics{/Users/whichway/workspace/E_ESSEX/CE265/MD/CE265 Chapter 6.assets/image-20210524233431912.png}
\caption{}
\end{figure}

6.64 GSM: handoff with common MSC

\textbackslash1. old BSS informs MSC of impending handoff, provides list
of 1+ new BSSs

\textbackslash2. MSC sets up path (allocates resources) to new BSS

\textbackslash3. new BSS allocates radio channel for use by mobile

\textbackslash4. new BSS signals MSC, old BSS: ready

\textbackslash5. old BSS tells mobile: perform handoff to new BSS

\textbackslash6. mobile, new BSS signal to activate new channel

\textbackslash7. mobile signals via new BSS to MSC: handoff complete.
MSC reroutes call

8 MSC-old-BSS resources released

\begin{figure}
\centering
\includegraphics{/Users/whichway/workspace/E_ESSEX/CE265/MD/CE265 Chapter 6.assets/image-20210524233501261.png}
\caption{}
\end{figure}

6.65 GSM: handoff between MSCs

\begin{itemize}
\item
  \emph{anchor MSC:} first MSC visited during call
\item
  \begin{itemize}
  \item
    call remains routed through anchor MSC
  \end{itemize}
\item
  new MSCs add on to end of MSC chain as mobile moves to new MSC
\item
  optional path minimization step to shorten multi-MSC chain
\end{itemize}

\begin{figure}
\centering
\includegraphics{/Users/whichway/workspace/E_ESSEX/CE265/MD/CE265 Chapter 6.assets/image-20210524233521664.png}
\caption{}
\end{figure}

6.66 GSM: handoff between MSCs

\begin{itemize}
\item
  \emph{anchor MSC:} first MSC visited during call
\item
  \begin{itemize}
  \item
    call remains routed through anchor MSC
  \end{itemize}
\item
  new MSCs add on to end of MSC chain as mobile moves to new MSC
\item
  optional path minimization step to shorten multi-MSC chain
\end{itemize}

\begin{figure}
\centering
\includegraphics{/Users/whichway/workspace/E_ESSEX/CE265/MD/CE265 Chapter 6.assets/image-20210524233548387.png}
\caption{}
\end{figure}

6.67 Mobility: GSM versus Mobile IP

\begin{longtable}[]{@{}lll@{}}
\toprule
\textbf{GSM element} & \textbf{Comment on GSM element} & \textbf{Mobile
IP element}\tabularnewline
\midrule
\endhead
\textbf{Home system} & Network to which mobile user's permanent phone
number belongs & \textbf{Home network}\tabularnewline
\textbf{Gateway Mobile Switching Center, or ``home MSC''. Home Location
Register (HLR)} & Home MSC: point of contact to obtain routable address
of mobile user. HLR: database in home system containing permanent phone
number, profile information, current location of mobile user,
subscription information & \textbf{Home agent}\tabularnewline
\textbf{Visited System} & Network other than home system where mobile
user is currently residing & \textbf{Visited network}\tabularnewline
\textbf{Visited Mobile services Switching Center.} \textbf{Visitor
Location Record (VLR)} & Visited MSC: responsible for setting up calls
to/from mobile nodes in cells associated with MSC. VLR: temporary
database entry in visited system, containing subscription information
for each visiting mobile user & \textbf{Foreign agent}\tabularnewline
\textbf{Mobile Station Roaming Number (MSRN), or ``roaming number''} &
Routable address for telephone call segment between home MSC and visited
MSC, visible to neither the mobile nor the correspondent. &
\textbf{Care-of-address}\tabularnewline
\bottomrule
\end{longtable}

6.68 Wireless, mobility: impact on higher layer protocols

\begin{itemize}
\item
  logically, impact \emph{should} be minimal \ldots{}
\item
  \begin{itemize}
  \item
    best effort service model remains unchanged
  \item
    TCP and UDP can (and do) run over wireless, mobile
  \end{itemize}
\item
  \ldots{} but performance-wise:
\item
  \begin{itemize}
  \item
    packet loss/delay due to bit-errors (discarded packets, delays for
    link-layer retransmissions), and handoff
  \item
    TCP interprets loss as congestion, will decrease congestion window
    un-necessarily
  \item
    delay impairments for real-time traffic
  \item
    limited bandwidth of wireless links
  \end{itemize}
\end{itemize}

\hypertarget{69-summary}{%
\subsection{6.9 Summary}\label{69-summary}}

\emph{Wireless}

\begin{itemize}
\item
  wireless links:
\item
  \begin{itemize}
  \item
    capacity, distance
  \item
    channel impairments
  \item
    CDMA
  \end{itemize}
\item
  IEEE 802.11 (``Wi-Fi'')
\item
  \begin{itemize}
  \item
    CSMA/CA reflects wireless channel characteristics
  \end{itemize}
\item
  cellular access
\item
  \begin{itemize}
  \item
    architecture
  \item
    standards (e.g., GSM, 3G, 4G LTE)
  \end{itemize}
\end{itemize}

\emph{Mobility}

\begin{itemize}
\item
  principles: addressing, routing to mobile users
\item
  \begin{itemize}
  \item
    home, visited networks
  \item
    direct, indirect routing
  \item
    care-of-addresses
  \end{itemize}
\item
  case studies
\item
  \begin{itemize}
  \item
    mobile IP
  \item
    mobility in GSM
  \end{itemize}
\item
  impact on higher-layer protocols
\end{itemize}

\end{document}
